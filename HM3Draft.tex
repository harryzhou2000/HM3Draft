%% 
%% Copyright 2007-2020 Elsevier Ltd
%% 
%% This file is part of the 'Elsarticle Bundle'.
%% ---------------------------------------------
%% 
%% It may be distributed under the conditions of the LaTeX Project Public
%% License, either version 1.2 of this license or (at your option) any
%% later version.  The latest version of this license is in
%%    http://www.latex-project.org/lppl.txt
%% and version 1.2 or later is part of all distributions of LaTeX
%% version 1999/12/01 or later.
%% 
%% The list of all files belonging to the 'Elsarticle Bundle' is
%% given in the file `manifest.txt'.
%% 

%% Template article for Elsevier's document class `elsarticle'
%% with numbered style bibliographic references
%% SP 2008/03/01
%%
%% 
%%
%% $Id: elsarticle-template-num.tex 190 2020-11-23 11:12:32Z rishi $
%%
%%
\documentclass[preprint,12pt]{elsarticle}

%% Use the option review to obtain double line spacing
%% \documentclass[authoryear,preprint,review,12pt]{elsarticle}

%% Use the options 1p,twocolumn; 3p; 3p,twocolumn; 5p; or 5p,twocolumn
%% for a journal layout:
%% \documentclass[final,1p,times]{elsarticle}
%% \documentclass[final,1p,times,twocolumn]{elsarticle}
%% \documentclass[final,3p,times]{elsarticle}
%% \documentclass[final,3p,times,twocolumn]{elsarticle}
%% \documentclass[final,5p,times]{elsarticle}
%% \documentclass[final,5p,times,twocolumn]{elsarticle}

%% For including figures, graphicx.sty has been loaded in
%% elsarticle.cls. If you prefer to use the old commands
%% please give \usepackage{epsfig}

%% The amssymb package provides various useful mathematical symbols
\usepackage{amssymb}
\usepackage{amsmath}
%% The amsthm package provides extended theorem environments
%% \usepackage{amsthm}

%% The lineno packages adds line numbers. Start line numbering with
%% \begin{linenumbers}, end it with \end{linenumbers}. Or switch it on
%% for the whole article with \linenumbers.
%% \usepackage{lineno}

\usepackage{hyperref}
\hypersetup{
    colorlinks=true,
    linkcolor=blue,
    filecolor=magenta,      
    urlcolor=cyan,
    }

\journal{Journal's Name}

\begin{document}

\begin{frontmatter}

    %% Title, authors and addresses

    %% use the tnoteref command within \title for footnotes;
    %% use the tnotetext command for theassociated footnote;
    %% use the fnref command within \author or \address for footnotes;
    %% use the fntext command for theassociated footnote;
    %% use the corref command within \author for corresponding author footnotes;
    %% use the cortext command for theassociated footnote;
    %% use the ead command for the email address,
    %% and the form \ead[url] for the home page:
    \title{A Nested Implicit Runge-Kutta Method
        Based on Time Interpolation
        for High-order Finite Volume Method}
    % \tnotetext[label1]{}
    % \ead{email address}
    % \ead[url]{home page}
    % \fntext[label2]{}
    % \cortext[cor1]{}
    % \fntext[label3]{}



    %% use optional labels to link authors explicitly to addresses:
    %% \author[label1,label2]{}
    %% \affiliation[label1]{organization={},
    %%             addressline={},
    %%             city={},
    %%             postcode={},
    %%             state={},
    %%             country={}}
    %%
    %% \affiliation[label2]{organization={},
    %%             addressline={},
    %%             city={},
    %%             postcode={},
    %%             state={},
    %%             country={}}

    \author[THUDEM]{Hanyu Zhou}
    \author[THUDEM]{Yuxin Ren}

    \affiliation[THUDEM]
    {
        organization=
            { Department of Engineering Mechanices, Tsinghua Universiy},%Department and Organization
        addressline={},
        city={},
        postcode={},
        state={Beijing},
        country={China}}



    \begin{abstract}
        Text of abstract

    \end{abstract}

    % %%Graphical abstract
    % \begin{graphicalabstract}
    % %\includegraphics{grabs}
    % \end{graphicalabstract}

    %%Research highlights
    \begin{highlights}
        \item Research highlight 1
        \item Research highlight 2
    \end{highlights}

    \begin{keyword}
        %% keywords here, in the form: keyword \sep keyword

        %% PACS codes here, in the form: \PACS code \sep code

        %% MSC codes here, in the form: \MSC code \sep code
        %% or \MSC[2008] code \sep code (2000 is the default)

    \end{keyword}

\end{frontmatter}

%% \linenumbers

%% main text
\section{Introduction}
\label{sec:intro}

In computational fluid dynamics (CFD),
high-order numerical methods have the capability of
resolving complex flows effectively and efficiently,
which have been attracting great attention recently.
Popular high-order CFD methods, including discontinuous
Galerkin (DG) methods
\cite{reed1973triangularDG,
    BASSI1997251DG,
    BASSI1997267DG,
    cockburn1989DGII,
    cockburn2001rungeDG},
spectral volume
\cite{WANG2002210_SV}
and spectral difference
\cite{LIU2006780_SD} methods,
PnPm procedures
\cite{DUMBSER20088209_PNPM},
FR/CPR methods
\cite{huynh2007flux_FR,
    huynh2009reconstruction_FR,
    vincent2011new_FR,
    wang2009unifying_CPR},
and finite volume (FV) methods
\cite{wang2016compact_VR,
    wang2016compact1_VR,
    wang2017compact_VR,
    pan2018high_VR,
    zhang2019compact_VR,
    barth1990higher_FV,
    delanaye1999quadratic_FV,
    ollivier1997quasi_ENO,
    friedrich1998weighted_WENO,
    hu1999weighted_WENO,
    dumbser2007quadrature_WENO},
generally provides spacial discretization methods
which produce semi-discretized forms of the original
partial differential equations (PDEs).
The semi-discretized PDEs are
sets of first order ordinary
differential equations (ODEs),
which are usually further solved with
ODE integrators such as the
popular strong stability preserving
Runge-Kutta (SSPRK) methods
\cite{gottlieb2001strong_SSPRK}.

Although explicit ODE integrators
are simple and efficient in a wide range of
CFD problems,
the Courant-Friedrichs-Lewy (CFL) constraint
that limits physical time step in explicit
methods could make them inefficient in
notably inhomogeneous or anisotropic
transient flows,
such as wall-bounded turbulence.
Such inefficiency could be overcome by
applying implicit ODE methods with
sufficient stability.
Due to Dahlquist's second barrier
\cite{dahlquist1963special},
only second and first order multi-step ODE methods
could achieve $A$-stability, and
trapezoid rule not being $L$-stable,
only the $L$-stable
second-order backward differentiation formula (BDF2)
is extensively adopted in transient CFD problems.
Different from multi-step methods,
the single-step implicit Runge-Kutta (IRK) methods
with multiple internal stages
can achieve higher order of accuracy while
preserving stability \cite{butcher2016ODEBook}.
Among the IRK methods, fully coupled IRK methods
could achieve optimal order given number of stages,
but they require the solution of a nonlinear
system with its dimension multiple times larger
than the ODE, which could be especially
troublesome for its implementation in
CFD schemes.
Pazner and Persson
\cite{pazner2017stage}
made effort in efficiently solving
fully IRK methods with DG, and
Jameson \cite{jameson2017evaluation}
discussed how to adopt dual time stepping
into fully IRK.
As a result of the difficulties in the solution
of fully IRK methods,
singly diagonally implicit Runge-Kutta (SDIRK)
methods are more commonly used in CFD, for example in
\cite{wang2017compact_VR}.
SDIRK methods have lower-triangular butcher tableau,
enabling the stage values to be solved in a sequence.
As a special case of SDIRK,
ESDIRK methods are SDIRK with an explicit first stage,
which are constructed to be
stiffly accurate and $L$-stable
\cite{kennedy2003additiveARK,kvaerno2004singly}.
High-order ESDIRK schemes and
BDF2 have been tested in
\cite{
    bijl2002implicitBDFvESDIRK,
    wang2007implicitDGTests}
to solve flow problems,
whose results illustrate
better accuracy and higher efficiency
of high-order ESDIRK methods
compared to second order BDF2.

Apart from fully IRK and SDIRK methods,
another class of implicit RK methods,
mono implicit RK methods (MIRK)
\cite{cash1975classMIRKOrig}
have explicit internal stages and
puts all the implicitness into the final
stage, allowing the implicit system to
remain have the same number of dimensions
as the original ODE.
Cash and Signhal derived examples of
$A$-stable and $L$-stable high-order
MIRK methods in
\cite{cash1977clasMIRK1,cash1982monoMIRK2}.
Kulikov and Shindin presented a similar type
of method called nested implicit RK (NIRK)
\cite{kulikov2006familyNIRKOrig}.
Discussion was made on symmetry, stiff accuracy and
other advantageous properties of a series of Gauss type
NIRK methods.
\cite{kulikov2009adaptive}.
The major drawback of NIRK or MIRK schemes is that
the Jacobian matrices of the nonlinear algebraic problems
for each NIRK or MIRK step are
polynomials of the Jacobian of ODE's right hand side.
Cash and Singhal proposed to find MIRK methods whose
Jacobian could be factorized or approximate the Jacobian
with a factorization,
so that the Newton iteration
step could be solved by solving a series of successive linear
problems \cite{cash1982monoMIRK2}.
Kulikov and Shindin found certain factorization
approximation could sabotage the stability of
the method in
\cite{kulikov2009adaptive},
and analyzed how to choose the approximation of
Jacobian in
\cite{kulikov2007asymptotic}.
MIRK and NIRK methods are more attractive than SDIRK methods
for they require fewer inner stage solving.
Typical fourth order
stiffly accurate ESDIRK requires 5 implicit internal stages to be
solved,
while MIRK and NIRK could
solve only one implicit stage.


MIRK and NIRK methods have implicit stages than SDIRK methods,
which gives them potential to gain better efficiency in
large scale problems.
However, current literature has seldom explored
the application of MIRK or NIRK methods in
PDE solving.
Therefore, this paper aims to
derive an NIRK type ODE method suitable for high-order
CFD implementation,
and apply the method to flow problems.

Paper's structure:...%TODO


\section{Hermite Interpolation Nested Implicit Runge Kutta Method}
\label{sec:HINIRK}

The acquiring of an NIRK method with
certain order of accuracy is usually based on
a certain integration method and
corresponding taylor expansion analysis,
as described in \cite{kulikov2006familyNIRKOrig}.
The present paper proposes a type of NIRK
derived with Hermite interpolation
(thus called HINIRK),
which simplifies the analysis on
stage accuracy and provides an approach to
modify the scheme.

\newcommand{\uu}{\mathbf{u}}
\newcommand{\R}{\mathbf{R}}
\newcommand{\inc}{\mathrm\Delta}

Considering a general first order ODE:
\begin{equation}
    \frac{d\uu}{dt} = \R(t, \uu)
\end{equation}
with $\uu^n$ as the known numerical solution at $t=t^n$ and
$t^{n+1} = t^{n} + \inc t$,
the HINIRK method has the form:
\begin{subequations}
    \label{eq:HM3}
    \begin{align}
        \uu^{n+1} & = \uu^{n} +
        \inc t
        \left(
        b_1\R(t^{n,1}, \uu^n) +
        b_2\R(t^{n,2}, \uu^*) +
        b_3\R(t^{n,3}, \uu^{n+1})
        \right)   \label{eq:HM3-1} \\
        \uu^{*}   & =
        a_1\uu^{n} +
        a_2\uu^{n+1} +
        \inc t
        \left(
        d_1\R(t^{n,1}, \uu^n) +
        d_2\R(t^{n,3}, \uu^{n+1})
        \right) \label{eq:HM3-2}
    \end{align}
\end{subequations}
where $t^{n,i}=t^n+c_i\inc t$, and $c_1 = 0, c_3 = 1, c_2\in(0,1)$.

The only stage value besides endpoints is $\uu^*$.
The internal stage $\uu^*$ can be explicitly
calculated with \eqref{eq:HM3-2}, which is derived
directly from the cubic Hermite interpolation of $\uu$
on interval $[t^n,t^{n+1}]$ evaluated at
$t^{n,2}$, which result in the interpolation relation:
\begin{equation}
    \begin{aligned}
        a_2 & = 1 - a_1 = 3{c_2}^2 - 2 {c_2}^3 \\
        d_1 & = {c_2} - 2 {c_2}^2 + {c_2}^3    \\
        d_2 & = - {c_2}^2 + {c_2}^3            \\
    \end{aligned}
    \label{eq:interp}
\end{equation}
The first equation \eqref{eq:HM3-1} is
a numeric integration on interval $[t^n,t^{n+1}]$ with
polynomial nodes
$t^{n,1}=t^n,t^{n,2}=t^n+c_2\inc t,t^{n,3}=t^{n+1}$, which demands:
\begin{equation}
    \begin{aligned}
        b_1 & = \frac{1}{2} - \frac{1}{6{c_2}}     \\
        b_2 & = \frac{1}{6{c_2}(1-{c_2})}          \\
        b_3 & = \frac{1}{2} - \frac{1}{6(1-{c_2})} \\
    \end{aligned}
    \label{eq:integ}
\end{equation}
The Hermite interpolation gives \eqref{eq:HM3-2}
local truncation error $O(\inc t^4)$,
and the numeric integration gives \eqref{eq:HM3-1}
local truncation error $O(\inc t^3)$, therefore the
classic order of accuracy of HMLB \eqref{eq:HM3} is
3. If $c_2=1/2$, then the numeric integration \eqref{eq:HM3-2}
becomes a Gauss-Lobatto quadrature rule, which yields
a local truncation error of $O(\inc t^4)$, making
the scheme 4th order accurate.
Also, from the interpolation, no matter
the choice of $c_2$, HINIRK has a stage order of
3, making it stiffly accurate.
As there is a family of HINIRK methods decided by
a single parameter $c_2$,
here we denote them as HINIRK($c_2$).


Reformulating \eqref{sec:HINIRK} into
a standard Runge-Kutta form yields a Butcher
tableau:
\begin{table}[htbp]
    \centering
    \begin{tabular}{c|ccc}
        0     & 0              & 0        & 0              \\
        $c_2$ & $d_1 + a_2b_1$ & $a_2b_2$ & $d_2 + a_2b_3$ \\
        1     & $b_1$          & $b_2$    & $b_3$          \\ \hline
              & $b_1$          & $b_2$    & $b_3$
    \end{tabular}
    \caption{Butcher tableau of \eqref{eq:HM3}}
    \label{tab:HM3Butcher}
\end{table}

From table \ref{tab:HM3Butcher} with the coefficients
decided with \eqref{eq:interp} and \eqref{eq:integ},
one can find that the 4th order accurate
HINIRK($1/2$) is indeed the Lobatto IIIA method
of order 4 \cite{wanner1996solving}.
The classic order and stage order of HINIRK could
also be evaluated from table \ref{tab:HM3Butcher} via
the simplifying assumptions.
Despite the formulation above only
provides an order degraded modification 
of Lobatto IIIA type IRK method, 
this modification will prove useful 
in latter discussion.

If the quadrature rule in HINIRK is 
replaced with Gauss-Legendre rule, 
the method immediately becomes a special case of the 
Gauss type NIRK method of order 4 
described in \cite{kulikov2006familyNIRKOrig}. 
For large scale CFD application, 
using 2 point Gauss-Legendre rule 
would mean deriving the right hand side at 
3 different stages iteratively. 
Thus, the current HINIRK method only considers 
a 3 point Gauss-Lobatto type quadrature 
(with the middle abscissa moved and order of accuracy reduced).   



% \begin{equation}
%     \label{eq:HM3R}
%     \begin{aligned}
%         \uu^{n+1} & = \uu^{n} +
%         \inc t
%         \left(
%         b_1\R(t^{n,1}, \uu^n) +
%         b_2\R(t^{n,2}, \uu^*) +
%         b_3\R(t^{n,3}, \uu^{n+1})
%         \right) \\
%         \uu^{*}   & =
%         \uu^{n}  + 
%         \inc t
%         \left(
%         (d_1 + a_2b_1)\R(t^{n,1}, \uu^n) +
%         a_2b_2\R(t^{n,2}, \uu^*) +
%         (d_2 + a_2b_3)\R(t^{n,3}, \uu^{n+1})
%         \right)
%     \end{aligned}
% \end{equation}

\subsection{Linear Stability}

Following standard analysis based on Dahlquist's equation
 $\frac{dy}{dt} = \lambda y$ \cite{wanner1996solving}, 
the stability function giving $y^{1}=R(h\lambda)y^0$ 
when applying HINIRK is in the form:
\begin{equation}
    R(z) = -\frac{4\,z-2\,c_{2}\,z-c_{2}\,z^2+z^2+6}{2\,z+2\,c_{2}\,z-c_{2}\,z^2-6}
\end{equation}
which becomes the (2,2)-Pad{\'e} approximation when $c_2=1/2$ and HINIRK
becomes Lobatto IIIA. The limit 
\begin{equation}
    \lim_{z\rightarrow\infty}R(z) = \frac{1-c_2}{c_2}
\end{equation}
gives that a necessary condition for $A$-stability of 
HINIRK is $c_2 \in [1/2,1)$, and shows that HINIRK 
is unable to achieve $L$-stability.

HINIRK($1/2$) or Lobatto IIIA method is symmetric, 
which is a preferable property when integrating 
reversible systems, 
but the symmetry could be considered harmful in CFD application.
Most CFD systems of interest are physically dissipative,
while for symmetric RK methods 
$R(z) \rightarrow 1$ when $z \rightarrow \infty$, 
which is more likely to preserve 
spurious modes arising from spacial discretization.  
Moreover, 

\subsection{Alternating Solving Strategy}

\section{Spacial Discretization Method}


\section{Numerical Tests}

\section{Conclusion}


%% The Appendices part is started with the command \appendix;
%% appendix sections are then done as normal sections
%% \appendix

%% \section{}
%% \label{}

%% If you have bibdatabase file and want bibtex to generate the
%% bibitems, please use
%%
\bibliographystyle{elsarticle-num}
\bibliography{HM3DraftRefs.bib}

%% else use the following coding to input the bibitems directly in the
%% TeX file.

% \begin{thebibliography}{00}

%     %% \bibitem{label}
%     %% Text of bibliographic item

%     \bibitem{}

% \end{thebibliography}


\end{document}
\endinput
%%
%% End of file `elsarticle-template-num.tex'.
