\documentclass[10pt]{article}
\usepackage[a4paper,left=2.54cm,top=2.54cm,right=2.54cm,bottom=2.54cm]{geometry}
\usepackage{fancyhdr}
\setlength{\headsep}{1.cm} % Adjust the space after the header
\usepackage{afterpage}
\usepackage{setspace}
\usepackage{bibspacing}
\usepackage{float}
\singlespacing

%%%% YOU CAN PUT YOUR OWN DEFINITIONS HERE
\newfont{\toto}{msbm10 at 12 pt}
\newfont{\ithd}{cmr9}


\newcommand{\U}{\mathbf{U}}
\newcommand{\F}{\mathbf{F}}
\newcommand{\x}{\mathbf{x}}

\newcommand{\OO}{\mathbf{\Omega}}
\newcommand{\UM}{\overline{\U}}
\newcommand{\Fn}{\tilde{\F}}
\newcommand{\n}{\mathbf{n}}
\newcommand{\uu}{\overline{\mathbf{U}}}
\newcommand{\R}{\mathbf{R}}
\newcommand{\inc}{\mathrm\Delta}
\newcommand{\Tau}{\mathrm{T}}
\newcommand{\Res}{\mathcal{R}}
\newcommand{\Jres}{\mathcal{J}}
\newcommand{\eye}{\mathbf{I}}
\newcommand{\J}{\mathbf{J}}
\usepackage{subcaption}

%%%% END OF YOUR DEFINITIONS 

\pagestyle{fancyplain}
\renewcommand{\headrulewidth}{0pt}

\usepackage{amsmath,amsthm,amsfonts,amssymb}
\usepackage[pdftex]{graphicx}
\usepackage[T1]{fontenc}

%%%% CONFERENCE HEADER. REPLACE xxxx WITH 4-DIGIT PAPER NUMBER ASSIGNED BY CONFERENCE COMMITTEE.

\rhead{\ithd{\bf ICCFD12-2024-xxxx\\  \   \\}}
\lhead{\ithd{\bf Twelfth International Conference on \\      
Computational Fluid Dynamics (ICCFD12), \\
Kobe, Japan, July 14-19, 2024
}}


\usepackage{titling}
\setlength{\droptitle}{0em}  
\pretitle{\vspace{-4em}\begin{center}\LARGE}
\posttitle{\end{center}\vspace{-1em}}
\preauthor{\begin{center}\large}
\postauthor{\end{center}\vspace{-6em}}


\title{
\bf 
An Efficient Implicit Runge Kutta Method for High-order Finite Volume Method 
Based on Temporal Interpolation 
}
\author{
Hanyu Zhou$^{*}$, Yuxin Ren$^{*}$ \\
$^{*}$ Tsinghua University, Beijing, China.
}
\date{}

\begin{document}

%%%% TITLE
\maketitle
\afterpage{\fancyhead{}}

%%%% ABSTRACT AND KEYWORDS
%\vskip0.5cm
\centerline{
}
\vskip0.5cm

%%%% MAIN PART
\section{Introduction}
% High-order numerical methods for spacial discretization are typically 
% combined with high-order time marching schemes. 
% Meanwhile, for stiff problems including wall-bounded flow 
% or reaction flow, an implicit ODE method is 
% more preferable. 
% Due to Dahlquist's second barrier
% \cite{dahlquist1963special}, 
% high-order linear multi-step methods 
% can not acquire $A$-stability, 
% resulting in the development of 
% high-order implicit single step methods,
% which mainly consist of implicit Runge-Kutta (IRK)
% methods \cite{butcher2016ODEBook}. 
% When applied to time marching for CFD systems, 
% fully IRK methods are efficient but
% require additional effort in the solution 
% of implicit algebraic systems \cite{pazner2017stage, jameson2017evaluation}. 
% Singly diagonally implicit Runge-Kutta (SDIRK)
% methods, including SDIRK with an explicit first 
% stage (ESDIRK) \cite{kennedy2003additiveARK,kvaerno2004singly},  
% are more prevalent due to their 
% simplicity in solving the stage values and nice stability.

The present research proposes a new type of implicit time marching
scheme based on temporal Hermite interpolation with adjustable
stability.
The scheme is equivalent with a
3rd to 4th order IRK
with high stage order.
A simple but efficient solving strategy
dedicated to application of the new scheme
in high-order finite volume
method is developed, and several numerical
tests are conducted which shows its
high efficiency and accuracy.

\section{Hermite Interpolation Implicit Runge-Kutta Method}

For clarity, suppose we aim to solve
the semi-discretized ODE system arising from high-order
finite volume method
\begin{equation}
  \frac{d\uu}{dt} = \R(t, \uu)
\end{equation}
The Hermite Interpolation Implicit Runge-Kutta Method (HIRK)
is defined as
\begin{subequations}
  \label{eq:HM3}
  \begin{align}
    \uu^{n+1} & = \uu^{n} +
    \inc t
    \left[
      b_1\R(t^{n,1}, \uu^n) +
      b_2\R(t^{n,2}, \uu^*) +
      b_3\R(t^{n,3}, \uu^{n+1})
    \right]   \label{eq:HM3-1} \\
    \uu^{*}   & =
    a_1\uu^{n} +
    a_2\uu^{n+1} +
    \inc t
    \left[
      d_1\R(t^{n,1}, \uu^n) +
      d_2\R(t^{n,3}, \uu^{n+1})
      \right] \label{eq:HM3-2}
  \end{align}
\end{subequations}
where $t^{n,i}=t^n+c_i\inc t$, and $c_1 = 0, c_3 = 1, c_2\in(0,1)$,
and $\uu^*$ is the stage value at $c_2$.
The first equation \eqref{eq:HM3-1} is
a quadrature rule for $t\in[t^n,t^{n+1}]$,
while the second equation \eqref{eq:HM3-2} is a Hermite interpolation
evaluated at $t=t^n + c_2\inc t$.
Due to these relations, coefficients in \eqref{eq:HM3}
could be determined by the only free parameter $c_2$.
The quadrature rule yields
\(
b_1  = \frac{1}{2} - \frac{1}{6{c_2}},\ \
b_2  = \frac{1}{6{c_2}(1-{c_2})},\ \
b_3  = \frac{1}{2} - \frac{1}{6(1-{c_2})}
\),
and the interpolation yields
\(
a_2 = 1 - a_1 = 3{c_2}^2 - 2 {c_2}^3,\ \
d_1  = {c_2} - 2 {c_2}^2 + {c_2}^3   ,\ \
d_2  = - {c_2}^2 + {c_2}^3
\).

It can be proved when $c_2=1/2$, the scheme \eqref{eq:HM3}
is 4th-order accurate, and 3rd-order accurate otherwise.
Also, linear stability analysis provides when $c_2=1/2$, the
scheme is $A$-stable and when $c_2>1/2$, the scheme's
stability improves for large eigenvalues.

In order to solve each implicit time stepping reliably and efficiently
when conducting time marching on high-order CFD systems,
a successive solving strategy is proposed and tested.
The strategy is defined by iteratively conducted two
successive updating:
\begin{equation}
  \label{eq:intersolve}
  \begin{aligned}
    \uu^{*,m+1}   & = \uu^{*,m} - \left[
      \eye - \inc t \frac{\partial \Res^{*'}(\uu^{n+1,m},\uu^{*,m})}
      {\partial \uu^{*,m}}
      + \inc t \Tau^{-1}
      \right]
    \backslash \Res^{*'}(\uu^{n+1,m},\uu^{*,m}) \\
    \uu^{n+1,m+1} & = \uu^{n+1,m} - \left[
      \eye - \inc t \frac{\partial \Res^{}(\uu^{n+1,m},\uu^{*,m+1})}
      {\partial \uu^{n+1,m}}
      + \inc t \Tau^{-1}
      \right]
    \backslash \Res^{}(\uu^{n+1,m},\uu^{*,m+1}) \\
  \end{aligned}
\end{equation}
where the residuals are
\begin{equation}
  \begin{aligned}
    \Res(\uu^{n+1},\uu^{*})   & \doteq \uu^{n+1} - \left(
    \uu^{n} +
    \inc t
    \left[
      b_1\R(t^{n,1}, \uu^n) +
      b_2\R(t^{n,2}, \uu^*) +
      b_3\R(t^{n,3}, \uu^{n+1})
      \right]
    \right)                                               \\
    \Res^*(\uu^{n+1},\uu^{*}) & \doteq\uu^{*} - \left(
    a_1\uu^{n} +
    a_2\uu^{n+1} +
    \inc t
    \left[
      d_1\R(t^{n,1}, \uu^n) +
      d_2\R(t^{n,3}, \uu^{n+1})
      \right]
    \right)
  \end{aligned}
\end{equation}
and
\(
  \Res^{*'}(\uu^{n+1},\uu^{*}) = \Res^*(\uu^{n+1},\uu^{*}) + \beta\Res(\uu^{n+1},\uu^{*})
\).
The parameter $\beta$ is a preconditioning coefficient,
and $\Tau^{-1}$ is the diagonal matrix of pseudo time step.
The $\mathbf{A}\backslash \mathbf{b}$
notation means to solve the linear problem $\mathbf{A}\mathbf{x}=\mathbf{b}$.
The operations defined in \eqref{eq:intersolve}
are very similar to those used in BDF methods, and can be easily implemented
in a Jacobian-free manner.

\section{Numerical Tests}

Tests of HIRK are conducted solving Euler equation with $p=3$ variational
reconstruction finite volume method \cite{wang2017compact_VR}.
Cases shown here solve the isentropic vortex problem \cite{hu1999weighted_WENO}.
The results are compared with classic 2nd order backward difference formula (BDF2)
and 4th order singly diagonally implicit Runge-Kutta (ESDIRK4) \cite{kennedy2003additiveARK}.

\begin{figure}[htbp]
  \centering
  \begin{subfigure}{0.25\textwidth}
    \includegraphics[width=\textwidth]{../pics/IV40_10steps_HM3LBT.png}
    \caption[]{HIRK $c_2=0.5$}
    \label{sfig:IV10Step_HM3LBT}
  \end{subfigure}\hfill
  \begin{subfigure}{0.25\textwidth}
    \includegraphics[width=\textwidth]{../pics/IV40_10steps_HM3.png}
    \caption[]{HIRK $c_2=0.55$}
    \label{sfig:IV10Step_HM3}
  \end{subfigure}\hfill %\\
  \begin{subfigure}{0.25\textwidth}
    \includegraphics[width=\textwidth]{../pics/IV40_10steps_ESDIRK4.png}
    \caption[]{ESDIRK4}
    \label{sfig:IV10Step_ESDIRK4}
  \end{subfigure}\hfill
  \begin{subfigure}{0.25\textwidth}
    \includegraphics[width=\textwidth]{../pics/IV40_10steps_BDF2.png}
    \caption[]{BDF2}
    \label{sfig:IV10Step_BDF2}
  \end{subfigure}
  \caption{Density distribution of isentropic vortex problem, with $\inc t = 1$ at $t=10$}
  \label{fig:IV10Step}
\end{figure}

Figure \ref{fig:IV10Step} illustrates the results on a $40\times40$ square mesh using
large time step size $\inc t = 1$, which corresponds to a CFL number roughly $11$.
The high-order time marching methods preserves some vortex characteristics, while
the 2nd order BDF2 smears the vortex completely. HIRK with $c_2 = 0.5$ generates
additional numeric oscillation, while ESDIRK and HIRK $c_2 = 0.55$ suppress them
better.


\begin{figure}[htbp]
  \centering
  \begin{subfigure}{0.4\textwidth}
    \includegraphics[width=\textwidth]{../pics/HM3_IV160_fig_1.png}
    \caption[]{Density error vs. time step size }
    \label{sfig:IVTests_Conv}
  \end{subfigure}\hfill
  \begin{subfigure}{0.4\textwidth}
    \includegraphics[width=\textwidth]{../pics/HM3_IV160_fig_2.png}
    \caption[]{Density error vs. CPU time}
    \label{sfig:IVTests_Eff}
  \end{subfigure}
  \caption[]{Convergence and efficiency test with isentropic vortex problem}
  \label{fig:IVTests}
\end{figure}

Convergence and efficiency tests are conducted on a fine $160\times160$ mesh with
negligible spacial error, and density error from analytic solution at $t=2$ is evaluated.
Figure \ref{sfig:IVTests_Conv} indicates all the time marching methods
roughly reach their temporal orders of accuracy. When time step sizes are
the same, ESDIRK4 is the most accurate, while HIRK methods are
quite close.
Figure \ref{sfig:IVTests_Eff} offers a insight of efficiency. With
the same CPU time consumed, HIRK methods are more accurate than
ESDIRK4, indicating HIRK's higher efficiency.




%%%% BIBLIOGRAPHY
\bibspacing=\dimen 100
\bibliographystyle{unsrt}
\bibliography{biblio}

\end{document}
