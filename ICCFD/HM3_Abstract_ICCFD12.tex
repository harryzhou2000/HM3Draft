\documentclass[10pt]{article}
\usepackage[a4paper,left=2.54cm,top=2.54cm,right=2.54cm,bottom=2.54cm]{geometry}
\usepackage{fancyhdr}
\setlength{\headsep}{1.cm} % Adjust the space after the header
\usepackage{afterpage}
\usepackage{setspace}
\usepackage{bibspacing}
\usepackage{float}
\usepackage{physics}
\singlespacing

%%%% YOU CAN PUT YOUR OWN DEFINITIONS HERE
\newfont{\toto}{msbm10 at 12 pt}
\newfont{\ithd}{cmr9}


\newcommand{\U}{\mathbf{U}}
\newcommand{\F}{\mathbf{F}}
\newcommand{\SRC}{\mathbf{S}}
\newcommand{\FF}{\mathbf{F}}
\newcommand{\x}{\mathbf{x}}

\newcommand{\OO}{\mathbf{\Omega}}
\newcommand{\UM}{\overline{\U}}
\newcommand{\Fn}{\tilde{\F}}
\newcommand{\n}{\mathbf{n}}
\newcommand{\uu}{\overline{\mathbf{U}}}
\newcommand{\R}{\mathbf{R}}
\newcommand{\inc}{\mathrm\Delta}
\newcommand{\Tau}{\mathrm{T}}
\renewcommand{\Res}{\mathcal{R}}
\newcommand{\Jres}{\mathcal{J}}
\newcommand{\eye}{\mathbf{I}}
\newcommand{\J}{\mathbf{J}}
\usepackage{subcaption}

%%%% END OF YOUR DEFINITIONS 

\pagestyle{fancyplain}
\renewcommand{\headrulewidth}{0pt}

\usepackage{amsmath,amsthm,amsfonts,amssymb}
\usepackage[pdftex]{graphicx}
\usepackage[T1]{fontenc}

%%%% CONFERENCE HEADER. REPLACE xxxx WITH 4-DIGIT PAPER NUMBER ASSIGNED BY CONFERENCE COMMITTEE.

\rhead{\ithd{\bf ICCFD12-2024-xxxx\\  \   \\}}
\lhead{\ithd{\bf Twelfth International Conference on \\      
Computational Fluid Dynamics (ICCFD12), \\
Kobe, Japan, July 14-19, 2024
}}


\usepackage{titling}
\setlength{\droptitle}{0em}  
\pretitle{\vspace{-4em}\begin{center}\LARGE}
\posttitle{\end{center}\vspace{-1em}}
\preauthor{\begin{center}\large}
\postauthor{\end{center}\vspace{-6em}}


\title{
\bf 
High-order Accurate Implicit Scheme Based on 
Temporal Reconstruction for Solving Compressible 
Navier-Stokes Equations
}
\author{
Hanyu Zhou$^{*}$, Yu-Xin Ren$^{*}$ \\
$^{*}$ Department of Engineering Mechanics, Tsinghua University, Beijing, China.
}
\date{}

\begin{document}

%%%% TITLE
\maketitle
\afterpage{\fancyhead{}}

%%%% ABSTRACT AND KEYWORDS
%\vskip0.5cm
\centerline{
}
\vskip 0.5 cm

%%%% MAIN PART
\section{Introduction}
% High-order numerical methods for spatial discretization are typically 
% combined with high-order time marching schemes. 
% Meanwhile, for stiff problems including wall-bounded flow 
% or reaction flow, an implicit ODE method is 
% more preferable. 
% Due to Dahlquist's second barrier
% \cite{dahlquist1963special}, 
% high-order linear multi-step methods 
% can not acquire $A$-stability, 
% resulting in the development of 
% high-order implicit single step methods,
% which mainly consist of implicit Runge-Kutta (IRK)
% methods \cite{butcher2016ODEBook}. 
% When applied to time marching for CFD systems, 
% fully IRK methods are efficient but
% require additional effort in the solution 
% of implicit algebraic systems \cite{pazner2017stage, jameson2017evaluation}. 
% Singly diagonally implicit Runge-Kutta (SDIRK)
% methods, including SDIRK with an explicit first 
% stage (ESDIRK) \cite{kennedy2003additiveARK,kvaerno2004singly},  
% are more prevalent due to their 
% simplicity in solving the stage values and nice stability.

% The present research proposes a new type of implicit time marching
% scheme based on temporal Hermite interpolation with adjustable
% stability.
% The scheme is equivalent with a
% 3rd to 4th order IRK
% with high stage order.
% A simple but efficient solving strategy
% dedicated to application of the new scheme
% in high-order finite volume
% method is developed, and several numerical
% tests are conducted which shows its 
% high efficiency and accuracy.

A new high-order accurate implicit scheme for
solving the compressible Navier-Stokes (NS) equations is proposed.
This scheme is a Direct Integration method based on Temporal Reconstruction,
which is called DITR method for short.
The DITR methods represent a general class of implicit scheme for solving
both the ordinary and partial differential equations.
Advantages of the DITR method are the simplicity in its construction
and the capability to handle governing equations
with unsteady source terms.
In this paper, we focus on single step 3rd and 4th order accurate
Direct Integration methods based on Temporal
Hermite Reconstruction (DITHR).
DITHR's accuracy and stability properties are studied,
and an efficient iteration technique is proposed.
Numerical tests show that the proposed schemes have 
similar stability dealing with a large CFL number 
compared with the 4th order ESDIRK scheme \cite{kennedy2003additiveARK}. %,kvaerno2004singly
However, DITHR is more efficient in the sense that it spends less CPU time 
than ESDIRK4 for achieving the same accuracy
due to less stage require by DITHR.

\section{Hermite DITR method}

DITR schemes can be applied to any spatial discretization including
finite volume, discontinuous Galerkin and so on.
The present research practices DITR upon the high-order compact FV
scheme based on variational reconstruction \cite{wang2017compact_VR}.
When the method of line is used to solve the compressible
NS equations,
the following systems of first-order nonlinear ODEs
will emerge after spatial discretization
\begin{equation}
  \label{eq:ODE}
  \dv{\uu}{t} = \R(t, \uu)+\SRC(t, \uu)\,,
\end{equation}
where $\uu$ is a real vector consisting of cell averaged quantities of
the entire field, and $\R(t, \uu),\SRC(t, \uu)$ are discretized flux and source terms respectively.
Basic idea of the new implicit schemes is to perform
a temporal polynomial reconstruction using $\uu$ and $\dv{\uu}{t}$
at $t=t^{n+1}, t^{n}, t^{n-1}...$ Next, Eq. \eqref{eq:ODE} is integrated
directly to obtain 
\(
  \uu^{n+1} - \uu^{n} = \int_{t^n}^{t^{n+1}}{\R(t,\uu(t))+\SRC(t,\uu(t))\dd{t}}
\).
% Given $\uu$ and $\dv{\uu}{t}$ at $t=t^{n+1}, t^{n}, t^{n-1}...$,
% a temporal polynomial reconstruction of $\uu(t)$ could be 
% obtained. With this temporal reconstruction, a numeric approximation of 
% temporal integration 
% \(
%   \label{eq:intGeneral}
%   \uu^{n+1} - \uu^{n} = \int_{t^n}^{t^{n+1}}{\R(t,\uu(t))\dd{t}}
% \)
% finally defines the implicit scheme.
% Choosing of different reconstruction 
% and integration schemes offers
% DITR ability to achieve high-order 
% with few steps or stages and versatility
% in handling stiff source terms. 
Different temporal reconstructions and numerical quadrature rules
result in DITR schemes with different accuracy and stability properties.
One advantage of this approach is that DITR can be designed 
to have fewer stages than traditional SDIRK methods.
Moreover, $\SRC(t,\uu(t))$ can be
integrated differently from $\R(t,\uu(t))$ to account for possible 
stiffness in the source terms.
% When the NS equations without source term 
% is solved, integration \eqref{eq:intGeneral}
% could be approximated by simple quadrature rules
% including Gauss-Legendre and Gauss-Lobatto rules. 
% When a flow system with unsteady source term 
% such as external force or chemical reaction is 
% involved, splitting the right-hand-side (RHS) 
% into flux term and source term as 
% \(\R = \FF + \SRC\), the stiff part of RHS
% $\SRC$ could be integrated  with refined time steps.
% Therefore, DITRs could provide a source term
% treatment method different from the commonly practiced
% splitting method. 
% A special case of DITR is the 
% Adams-Moulton methods, which is DITR
% using Lagrange interpolation and 
% quadrature rule based on nodal values.
% Meanwhile, second order trapezoid rule and implicit midpoint
% methods are DITR using linear reconstruction with 
% trapezoid quadrature rule or midpoint quadrature rule.

The present paper 
studies DITHR which is constructed with 
Hermite interpolation and a modified Gauss-Lobatto quadrature rule.
% , 
% applied to NS equations with flux terms only 
The Hermite interpolation only requires 
$\uu$ and $\dv{\uu}{t}$ at $t=t^{n+1}, t^{n}$, 
making DITHR a single step method.
% The middle abscissa of Gauss-Lobatto rule
% can be modified to trade accuracy for stability.
% For clarity, suppose we aim to solve
% the semi-discretized ODE system arising from high-order
% finite volume method
% \begin{equation}
%   \frac{d\uu}{dt} = \R(t, \uu)
% \end{equation}
% The Hermite Interpolation Implicit Runge-Kutta Method
% is defined as
DITHR can be summarized in the form of 
\begin{subequations}
  \label{eq:HM3}
  \begin{align}
    \uu^{n+1} & = \uu^{n} +
    \inc t
    \left[
      b_1\R(t^n, \uu^n) +
      b_2\R(t^{n,2}, \uu^*) +
      b_3\R(t^{n+1}, \uu^{n+1})
    \right]   \label{eq:HM3-1}\,, \\
    \uu^{*}   & =
    a_1\uu^{n} +
    a_2\uu^{n+1} +
    \inc t
    \left[
      d_1\R(t^{n}, \uu^n) +
      d_2\R(t^{n+1}, \uu^{n+1})
      \right] \label{eq:HM3-2}\,,
  \end{align}
\end{subequations}
where $t^{n,2}=t^n+c_2\inc t$, $c_2\in(0,1)$,
and $\uu^*$ is the interpolated value at $c_2$.
The first equation \eqref{eq:HM3-1} is
a quadrature rule for $t\in[t^n,t^{n+1}]$,
while the second equation \eqref{eq:HM3-2} is a Hermite interpolation
evaluated at $t=t^n + c_2\inc t$.
Coefficient $c_2$ is the only free parameter and other coefficients are 
determined by
% Due to these relations, coefficients in \eqref{eq:HM3}
% could be determined by the only free parameter $c_2$.
% The quadrature rule yields
\(
b_1  = \frac{1}{2} - \frac{1}{6{c_2}},\ \
b_2  = \frac{1}{6{c_2}(1-{c_2})},\ \
b_3  = \frac{1}{2} - \frac{1}{6(1-{c_2})}
\),
% and the interpolation yields
\(
a_2 = 1 - a_1 = 3{c_2}^2 - 2 {c_2}^3,\ \
d_1  = {c_2} - 2 {c_2}^2 + {c_2}^3   ,\ \
d_2  = - {c_2}^2 + {c_2}^3
\). When $c_2=1/2$, the scheme Eq. \eqref{eq:HM3}
is 4th-order accurate, and 3rd-order accurate otherwise.
Linear stability analysis provides when $c_2\geq1/2$, the
scheme is $A$-stable and when $c_2$ increases from $1/2$, 
DITHR's stability is improved.

To solve this implicit scheme efficiently,
a successive iteration strategy is proposed and tested.
The successive iteration strategy only requires 
matrix-free operations, making its implementation
essentially the same as BDF and ESDIRK schemes.

% The strategy is defined by iteratively conducted two
% successive updating:
% \begin{equation}
%   \label{eq:intersolve}
%   \begin{aligned}
%     \uu^{*,m+1}   & = \uu^{*,m} - \left[
%       \eye - \inc t \frac{\partial \Res^{*'}(\uu^{n+1,m},\uu^{*,m})}
%       {\partial \uu^{*,m}}
%       + \inc t \Tau^{-1}
%       \right]
%     \backslash \Res^{*'}(\uu^{n+1,m},\uu^{*,m}) \\
%     \uu^{n+1,m+1} & = \uu^{n+1,m} - \left[
%       \eye - \inc t \frac{\partial \Res^{}(\uu^{n+1,m},\uu^{*,m+1})}
%       {\partial \uu^{n+1,m}}
%       + \inc t \Tau^{-1}
%       \right]
%     \backslash \Res^{}(\uu^{n+1,m},\uu^{*,m+1}) \\
%   \end{aligned}
% \end{equation}
% where the residuals are
% \begin{equation}
%   \begin{aligned}
%     \Res(\uu^{n+1},\uu^{*})   & \doteq \uu^{n+1} - \left(
%     \uu^{n} +
%     \inc t
%     \left[
%       b_1\R(t^{n,1}, \uu^n) +
%       b_2\R(t^{n,2}, \uu^*) +
%       b_3\R(t^{n,3}, \uu^{n+1})
%       \right]
%     \right)                                               \\
%     \Res^*(\uu^{n+1},\uu^{*}) & \doteq\uu^{*} - \left(
%     a_1\uu^{n} +
%     a_2\uu^{n+1} +
%     \inc t
%     \left[
%       d_1\R(t^{n,1}, \uu^n) +
%       d_2\R(t^{n,3}, \uu^{n+1})
%       \right]
%     \right)
%   \end{aligned}
% \end{equation}
% and
% \(
% \Res^{*'}(\uu^{n+1},\uu^{*}) = \Res^*(\uu^{n+1},\uu^{*}) + \beta\Res(\uu^{n+1},\uu^{*})
% \).
% The parameter $\beta$ is a preconditioning coefficient,
% and $\Tau^{-1}$ is the diagonal matrix of pseudo time step.
% The $\mathbf{A}\backslash \mathbf{b}$
% notation means to solve the linear problem $\mathbf{A}\mathbf{x}=\mathbf{b}$.
% The operations defined in \eqref{eq:intersolve}
% are very similar to those used in BDF methods, and can be easily implemented
% in a Jacobian-free manner.

\section{Numerical Tests}

Tests are conducted solving NS and Euler equations with 4th order variational
reconstruction finite volume method \cite{wang2017compact_VR}. 
Convergence tests using isotropic vortex problem 
indicate that DITHR schemes can reach 3rd and 4th order
of accuracy. 
% Cases shown here solve the isotropic vortex problem \cite{hu1999weighted_WENO},
% and a $Re$ 1200 two dimensional vortex shedding of a cylinder.
% The results are compared with classic 2nd order backward difference formula (BDF2)
% and 4th order singly diagonally implicit Runge-Kutta (ESDIRK4) \cite{kennedy2003additiveARK}.

\begin{figure}[htbp]
  \centering
  % \begin{subfigure}{0.3\textwidth}
  %   \includegraphics[width=\textwidth]{../pics/HM3_IV160_fig_1.pdf}
  %   \caption[]{Error vs. time step size, isotropic vortex}
  %   \label{sfig:IVTests_Conv}
  % \end{subfigure}\hfill
  \begin{subfigure}{0.4\textwidth}
    \includegraphics[width=\textwidth]{../pics/HM3_IV160_fig_2.pdf}
    \caption[]{Error vs. CPU time, isotropic vortex}
    \label{sfig:IVTests_Eff}
  \end{subfigure}\hfill
  \begin{subfigure}{0.4\textwidth}
    \includegraphics[width=\textwidth]{../pics/Cylinder_fig4.pdf}
    \caption[]{Error vs. Consumption, vortex shedding}
    \label{sfig:CyTests_Eff}
  \end{subfigure}
  \caption[]{Efficiency test with isotropic vortex and vortex shedding problems}
  \label{fig:IVTests}
\end{figure}

DITHR's performance is compared with BDF2 and ESDIRK4 \cite{kennedy2003additiveARK}.
Efficiency of different schemes is compared by 
recording their time marching error at different CPU time consumption,
as presented in Figure \ref{fig:IVTests}. 
The $Re=1200$ 
circular cylinder external flow is used as the vortex shedding problem.
Isotropic vortex uses density for error indication, while the 
vortex shedding problem uses transient $y$ momentum right-hand-side.
Figure \ref{fig:IVTests} indicates that with the same computational 
consumption, DITHR schemes can achieve the highest accuracy, 
and DITHR $c_2=0.5$ is better than $c_2=0.55$.
Having 5 implicit stages to solve each step makes ESDIRK4 even less efficient than 
the 3rd order DITHR $c_2=0.55$.

% Convergence and efficiency tests are conducted with the isentropic vortex 
% problem solved on fine $160\times160$ mesh which introduces
% negligible spatial error, and density errors at $t=2$ is evaluated.
% Figure \ref{sfig:IVTests_Conv} indicates all the time marching methods
% roughly reach their theoretical temporal orders of accuracy. 
% When time step sizes are the same, ESDIRK4 is the most accurate, while DITHR methods are only slightly less accurate.
% Figure \ref{sfig:IVTests_Eff} offers a insight of efficiency. With
% the same CPU time consumed, DITHR methods are more accurate than
% ESDIRK4, indicating DITHR's higher efficiency.
% Meanwhile, similar efficiency analysis is also conducted on $Re$ 1200 two dimensional cylinder vortex shedding 
% problem 
% initialized with a fully developed solution,
% using $y$-direction momentum RHS as the error indicator.
% % Vortex shedding problems use a fully developed numeric solution as initial condition,
% % whose results are compared with a reference solution calculated using ESDIRK4 with very 
% % small time step $\inc t = 1.25\times10^{-3}$. 
% % Solution of $t\in[0,10]$ is calculated to evaluate 
% % error induced by temporal discretization.
% Efficiency results of the vortex shedding problem are shown in figure \ref{sfig:CyTests_Eff},
% indicating larger difference in efficiency between schemes while the qualitative 
% results are identical with   the isentropic vortex problem. 
% % In figure \ref{sfig:CyTests_Eff}, DITHR schemes have best efficiency, 
% % while DITHR $c_2=0.5$ is more efficient than DITHR $c_2=0.55$ and ESDIRK4.

\begin{figure}[htbp]
  \centering
  \begin{subfigure}{0.25\textwidth}
    \includegraphics[width=\textwidth]{../pics/CylinderB1_Re1200_TH9_BDF.png}
    \caption[]{BDF2}
    \label{sfig:Cytests_BDF}
  \end{subfigure}\hfill
  \begin{subfigure}{0.25\textwidth}
    \includegraphics[width=\textwidth]{../pics/CylinderB1_Re1200_TH9_ESDIRK.png}
    \caption[]{ESDIRK4}
    \label{sfig:Cytests_ESDIRK}
  \end{subfigure}\hfill
  \begin{subfigure}{0.25\textwidth}
    \includegraphics[width=\textwidth]{../pics/CylinderB1_Re1200_TH9_HM3.png}
    \caption[]{DITHR $c_2=0.55$}
    \label{sfig:Cytests_HM3}
  \end{subfigure}\hfill
  \begin{subfigure}{0.25\textwidth}
    \includegraphics[width=\textwidth]{../pics/CylinderB1_Re1200_TH9_REF_TH1.png}
    \caption[]{Reference Solution}
    \label{sfig:Cytests_ref}
  \end{subfigure}
  \caption[]{Vorticity distribution of different schemes at $t=10$}
  \label{fig:Cytests}
\end{figure}

Figure \ref{fig:Cytests} visualizes the vortex shedding problem's $z$-vorticity computed with different schemes using $\inc t = 0.1$ where $CFL$ is 
approximately $1000$. 
The reference 
result is computed using ESDIRK4 with very small time step $\inc t = 1.25\times10^{-3}$.
Results of high-order schemes in Figure \ref{sfig:Cytests_ESDIRK}, \ref{sfig:Cytests_HM3}
are almost identical to the reference solution in Figure \ref{sfig:Cytests_ref}, 
while 2nd order BDF2 generates a vortex street with observably smaller vorticity peak and 
different vortex positions in Figure \ref{sfig:Cytests_BDF}.




%%%% BIBLIOGRAPHY
\bibspacing=\dimen 100
\bibliographystyle{unsrt}
\bibliography{biblio}

\end{document}
