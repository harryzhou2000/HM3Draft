% !TeX root = ./HM3Draft.tex

\newcommand{\trans}{^\mathrm{T}}

\newcommand{\U}{\mathbf{U}}
\newcommand{\F}{\mathbf{F}}
\newcommand{\x}{\mathbf{x}}

\newcommand{\OO}{\mathbf{\Omega}}
\newcommand{\UM}{\overline{\U}}
\newcommand{\Fn}{\tilde{\F}}
\newcommand{\n}{\mathbf{n}}
\newcommand{\uu}{\overline{\mathbf{U}}}
\newcommand{\R}{\mathbf{R}}
\newcommand{\inc}{\mathrm\Delta}
\newcommand{\Tau}{\mathrm{T}}
\renewcommand{\real}{\mathrm{Re}}
\newcommand{\imag}{\mathrm{Im}}

\newcommand{\CFLt}{\text{CFL}_t}
\newcommand{\CFLtau}{\text{CFL}_\tau}

\newcommand{\eeqref}[1]{Eq.\eqref{#1}}
\newcommand{\us}{\mathbf{u}}


\renewcommand{\Res}{\mathcal{R}}
\newcommand{\Jres}{\mathcal{J}}
\newcommand{\eye}{\mathbf{I}}
\newcommand{\J}{\mathbf{J}}
\newcommand{\FF}{\mathcal{F}}
\newcommand{\Pm}{\mathbf{P}}
\newcommand{\Pe}{\mathrm{P}}

\newcommand{\imagUnit}{\mathrm{i}}

\newcommand{\GG}{\mathcal{G}}


\newcommand{\A}{\mathbf{A}}
\newcommand{\B}{\mathbf{B}}
\newcommand{\C}{\mathbf{C}}

%% main text
\section{Introduction}
\label{sec:intro}

In computational fluid dynamics (CFD),
high-order numerical methods have the capability of
resolving complex flows effectively and efficiently,
which have been attracting great attention recently.
Popular high-order CFD methods, including discontinuous
Galerkin (DG) methods
\cite{reed1973triangularDG,
    BASSI1997251DG,
    BASSI1997267DG,
    cockburn1989DGII,
    cockburn2001rungeDG},
spectral volume
\cite{WANG2002210_SV}
and spectral difference
\cite{LIU2006780_SD} methods,
PnPm procedures
\cite{DUMBSER20088209_PNPM},
FR/CPR methods
\cite{huynh2007flux_FR,
    huynh2009reconstruction_FR,
    vincent2011new_FR,
    wang2009unifying_CPR},
and finite volume (FV) methods
\cite{wang2016compact_VR,
    wang2016compact1_VR,
    wang2017compact_VR,
    pan2018high_VR,
    zhang2019compact_VR,
    barth1990higher_FV,
    delanaye1999quadratic_FV,
    ollivier1997quasi_ENO,
    friedrich1998weighted_WENO,
    hu1999weighted_WENO,
    dumbser2007quadrature_WENO},
generally adopted the method of line,
in which the governing equations 
are spatially discretized at first
resulting in a system of ODEs.
The system of ODEs is then 
solved with
ODE integrators such as the 
popular strong stability preserving
Runge-Kutta (SSPRK) methods
\cite{gottlieb2001strong_SSPRK}.

Although explicit ODE integrators
are simple and efficient for a wide range of
CFD problems,
the Courant-Friedrichs-Lewy (CFL) condition
that limits physical time step in explicit
methods could make them inefficient in
notably inhomogeneous or anisotropic
transient flows,
such as wall-bounded turbulence.
Such inefficiency could be overcome by
applying implicit ODE solvers which
usually have better stability attributes.
Due to Dahlquist's second barrier
\cite{dahlquist1963special},
only second and first order linear multistep ODE methods
could achieve $A$-stability.
Therefore,
the $L$-stable
second-order backward differentiation formula (BDF2)
is extensively adopted in solving transient CFD problems.
Different from multistep methods,
the single-step implicit Runge-Kutta (IRK) methods
with multiple internal stages
can achieve higher order of accuracy while
preserving stability \cite{butcher2016ODEBook}.
Among the IRK methods, fully coupled IRK methods
could achieve optimal order
for given number of stages,
but they require the solution of a nonlinear
system with its dimension several times larger
than that of the ODE.
The enlarged algebraic system
could be especially troublesome for its implementation in
CFD solvers.
Pazner and Persson
\cite{pazner2017stage}
made effort in efficiently solving
fully IRK methods with DG, and
Jameson \cite{jameson2017evaluation}
discussed how to adopt dual time stepping
into fully IRKs.
Due to the difficulties in solving
fully IRK methods,
singly diagonally implicit Runge-Kutta (SDIRK)
methods are more commonly used in CFD, for example in
\cite{wang2017compact_VR}.
SDIRK methods have lower-triangular butcher tableau,
enabling the stage values to be solved in a sequence.
As a special case of SDIRK,
ESDIRK methods are SDIRK with an explicit first stage,
which are constructed to have
second stage order while being $L$-stable
\cite{kennedy2003additiveARK,kvaerno2004singly}.
High-order ESDIRK schemes and
BDF2 have been tested in
\cite{
    bijl2002implicitBDFvESDIRK,
    wang2007implicitDGTests}
to solve flow problems,
whose results illustrate
better accuracy and higher efficiency
of high-order ESDIRK methods
compared to second order BDF2.

Apart from fully IRK and SDIRK methods,
another class of implicit RK methods,
mono implicit RK methods (MIRK)
\cite{cash1975classMIRKOrig}
with explicit internal stages and
an implicit final
stage,
allow the implicit system to
have the same number of dimensions
as the original ODE.
Cash and Signhal derived examples of
$A$-stable and $L$-stable high-order
MIRK methods in
\cite{cash1977clasMIRK1,cash1982monoMIRK2}.
Kulikov and Shindin presented a similar type
of method called nested implicit RK (NIRK)
\cite{kulikov2006familyNIRKOrig}.
Discussion was made on symmetry, stiff accuracy and
other advantageous properties of a series of Gauss type
NIRK methods.
\cite{kulikov2009adaptive}.
The major drawback of NIRK or MIRK schemes is that
the Jacobian matrices of the nonlinear algebraic problems
for each NIRK or MIRK step are
polynomials of the Jacobian of the ODE's right-hand-side.
Cash and Singhal proposed to find MIRK methods whose
Jacobian could be approximately factorized,
so that the Newton iteration
could be realized approximately by solving a series of successive linear
problems \cite{cash1982monoMIRK2}.
Kulikov and Shindin found certain factorization
approximation could sabotage the stability of
the method in
\cite{kulikov2009adaptive},
and analyzed how to choose the approximation of
Jacobian in
\cite{kulikov2007asymptotic}.
MIRK and NIRK methods are more attractive than SDIRK methods
for they require fewer inner stages.
Typical fourth order
stiffly accurate ESDIRK requires 5 implicit internal stages to be
solved,
while MIRK and NIRK could
solve only one implicit stage.


MIRK and NIRK methods have less
implicit stages than SDIRK methods,
which gives them potential to gain better efficiency in
large scale problems.
However, current literature has seldom explored
the application of MIRK or NIRK methods in
solving PDEs.
The fully IRK methods, on the other hand,
have been practiced in
finite volume \cite{jameson2017evaluation} and
DG \cite{pazner2017stage}
as mentioned before,
but their implementations
in CFD methods are significantly more
complex compared to DIRK methods.

Different from previous approaches,
the current paper aims to construct implicit
time marching schemes based on temporal
reconstruction.
By combining polynomial approximation and
numerical quadrature rules in the direction of time,
stable,
high-order accurate and efficient implicit ODE methods
termed as DITR could be
obtained straightforwardly.
The basic idea of the DITR scheme will be presented
in Section \ref{sec:HIRK}.
Briefly speaking, the construction of the DITR
schemes consists of the following two steps.
The first step is the direct integration of the
system of first-order ODEs.
When the right-hand side is integrated using the
three point Gauss-Lobatto rule, the resulting DITR
schemes have two stages and are at most temporally fourth-order accurate.
The actual order of accuracy is then determined by
the second step which is the temporal reconstruction.
When the temporal reconstruction uses only the information of n
and n+1 steps, the DITR schemes are single step schemes
similar to the MIRK or NIRK schemes.
When the temporal reconstruction uses the information of n-1, n and n+1 steps,
the resulting DITR schemes are multistep methods that are
different from the common implicit linear multistep
methods such as
the Adams-Moulton and BDF schemes \cite{wanner1996solving}.
%! actually, the method was described as a Multistep Collocation Method 
%! actually, DITR is a collocation method
The stability of the DITR
schemes is mainly determined by the temporal
reconstruction step.
When adopting proper quadrature rules
and temporal reconstructions in these two steps, single-step
and multistep methods with arbitrary orders of temporal
accuracy can be constructed. Therefore, the DITR approaches
are new methodologies for designing implicit time marching
schemes that are different from the IRK and multistep
methods.

When applying DITR schemes to solving compressible Euler and
Navier-Stokes equations, the approximate factorization
proposed by Kulikov and Shindin \cite{kulikov2007asymptotic}
does not work well.
In the present paper,
a new iteration scheme called the stage-decoupled iteration
is proposed.
Stability analysis
confirm the advantage of the proposed procedure.
It will be shown that the new methods can be implemented
in a matrix-free style, guaranteeing their portability.
The new implicit methods are further applied to
high-order compact finite volume method to
test their capabilities in flow problems. Numerical results
illustrate that the new methods are high-order accurate
and competitive with the 4th order ESDIRK method in efficiency.

% Paper's structure:...%TODO

The rest of the paper is organized as follows.
High-order compact finite volume spatial  discretization schemes
are introduced in Section \ref{sec:CFV}.
The construction and analyses of DITR methods are presented
in Section \ref{sec:HIRK}.
Section \ref{sec:Solving} introduces the methods
of solving DITR and discusses their stability attributes.
Section \ref{sec:Results} illustrates some numerical results of
DITR solving the Euler and N-S equations, combined with
spatial  discretization from Section \ref{sec:CFV}.
Last, Section \ref{sec:Conc} gives the concluding remarks.


\section{High-order compact finite volume method}
\label{sec:CFV}



The new time-marching method developed in
the current paper can be applied with
any spatial discretizations when solving
time-dependent PDEs such as the compressible Euler and Navier-Stokes
equations.
However, their actual performance will be affected by
the spatial discretizations.
Therefore,
this section will provide a brief description of
finite volume spatial  discretization,
and additional details
of the high-order finite volume scheme
used in numerical tests are specified.

\subsection{Governing equations}
\label{ssec:GovEq}

The compressible Navier-Stokes equations has the conservative form:
\begin{equation}
    \label{eq:NS}
    \frac{\partial \U}{\partial t} +
    \mathbf\nabla \cdot (\F - \F_v)  = 0
\end{equation}
where $\U$ is vector of conservative variables and
$\F=[\F_1,\F_2,\F_3]$,
$\F_v=[\F_{v,1},\F_{v,2},\F_{v,3}]$
are
inviscid and viscous tensors of
their flux.
In Cartesian coordinates $x_k, k=1,2,3$, the components are
\begin{equation}
    \U = \begin{bmatrix}
        \rho \\ \rho u_1 \\ \rho u_2 \\ \rho u_3 \\ E
    \end{bmatrix},\ \
    \F_j = \begin{bmatrix}
        \rho u_j                   \\
        \rho u_1u_j + p\delta_{1j} \\
        \rho u_2u_j + p\delta_{2j} \\
        \rho u_3u_j + p\delta_{3j} \\
        (E+p)u_j                   \\
    \end{bmatrix},\ \
    \F_{v,j} = \begin{bmatrix}
        0                                \\
        \tau_{1j}                        \\
        \tau_{2j}                        \\
        \tau_{3j}                        \\
        \sum_{k=1}^3{u_k\tau_{kj}} - K_j \\
    \end{bmatrix}
\end{equation}
where $\rho$ is density,
$u_i, i=1,2,3$ are velocity components,
$p$ is pressure,
$E$ is total energy per unit volume,
$\tau_{ij}, i,j=1,2,3$ are viscous stress tensor components
and
$K_i, i=1,2,3$ are heat flux components.
$\delta_{ij}$ is
Kronecker delta.
With ideal gas equation of state,
Newtonian viscosity and Fourier
heat conduction, additional relations
\begin{equation}
    \begin{aligned}
        E         & = \frac{p}{\gamma -1 } + \frac{1}{2}\rho\sum_{k=1}^{3}(u_ku_k),  \\
        p         & =\rho R_g T                                              ,       \\
        \tau_{ij} & =
        \mu\left(\frac{\partial u_i}{\partial x_j} + \frac{\partial u_j}{\partial x_i}\right)
        -
        \frac{2}{3}\mu \delta_{ij}\sum_{k=1}^{3}{\frac{\partial u_k}{\partial x_k}}, \\
        K_i       & = - \kappa \frac{\partial T}{\partial x_i}
    \end{aligned}
\end{equation}
are used to close the equations
with $T$ being temperature, $\gamma$ being specific heat ratio,
$R_g$ being  gas constant, $\mu$ being dynamic viscosity and
$\kappa$ being thermal conductivity.
Specific heat ratio $\gamma$ is fixed to $1.4$ in this paper.
The current paper only
considers using simple gas property
with $\kappa = \mu c_p / Pr$,
and $\mu=\mu_{\infty}$ being a constant,
while $c_p$ is special heat capacity
at constant pressure and
Prandtl number $Pr$ is fixed to 0.71 in this paper.
When $\mu=0$,
$\F_v=0$ and
equation \eqref{eq:NS} becomes Euler equation.
The equations discussed above are in 3D form, and
assuming constant distribution of values over $x_3$
yields the 2D version of NS and Euler equations.

\subsection{High-order finite volume spatial discretization}
\label{ssec:FV}




This subsection provides a general framework of
high-order
finite volume discretization.
The computational domain $\OO$ is divided
into $N_{cell}$ cells $\OO_i, i=1,2,...N_{cell}$ which
are non-overlapping, forming a mesh.
An averaging of conservative quantities
over each cell is
\begin{equation}
    \label{eq:FVMean}
    \UM_i = \frac{1}{\overline{\OO}_i}\int_{\OO_i}\U(\x)\dd \Omega
\end{equation}
where $\overline{\OO}_i$ is the volume of $\OO_i$.

Next, a degree $k$ piecewise polynomial reconstruction is
conducted to approximate the distribution of
quantities
\begin{equation}
    \label{eq:FVRec}
    \U_i(\x) = \UM_i + \sum_{l=1}^{\mathrm{NDOF}(k)}{\U_i^l\varphi_{i,l}(\x) }
\end{equation}
in which $\U_j(\x)$ is the local polynomial distribution on cell $\OO_{j}$,
and
$\varphi_{j,i}(\x)$ are
polynomial basis functions.
\eeqref{eq:FVRec} can also be considered as a scalar field
\begin{equation}
    \label{eq:FVRecScalar}
    u_i(\x) = \overline{u}_i + \sum_{l=1}^{\mathrm{NDOF}(k)}{u_i^l\varphi_{i,l}(\x) }
\end{equation}
where $u_i(\x)$ is one of $\rho(\x), \rho u_1(\x)\dots$ and should not
be confused with the velocity components $u_1, u_2, u_3$
which was used in the governing equations.

There are $\mathrm{NDOF}(k)$ polynomial bases for reconstruction on
each cell.
The current paper uses zero-mean Taylor basis
for reconstruction \cite{wang2017compact_VR}.
Given a specific reconstruction method,
the reconstruction coefficients $\U_i^l$ (or $u_i^l$ for
its components) can be determined using
the distribution of mean value $\UM_i, i=1,2...N_{cell}$
and boundary conditions.

With the piecewise polynomial approximation,
the PDEs \eqref{eq:NS} can therefore become ODEs
with cell averaging applied
\begin{equation}
    \label{eq:FVInt}
    \frac{\dd\UM_i}{\dd t}
    +\sum_{j\in S_i, j\neq i}{\left(
        \int_{f_{i,j}}{
            [\Fn(\U_i,\U_j) - \Fn_v(\U_i,\U_j, \nabla \U_i, \nabla \U_j)] \cdot \n  \dd A
        }\right)
    }
    = 0
\end{equation}
where $\Fn(\U_i,\U_j)$ and $\Fn_v(\U_i,\U_j, \nabla \U_i, \nabla \U_j)$ are approximations
of exact fluxes on the cell interfaces,
and $f_{i,j}=\OO_i \cap \OO_j$ is the interface between $i,j$ cells.
Set $S_i$ denotes the compact stencil of cell $\OO_{i}$, consisting
of cell $\OO_{i}$ and its direct face neighboring cells.
The piecewise polynomial approximation \eqref{eq:FVRec} does not
guarantee a continuous distribution on interfaces,
thus the numerical fluxes
$\Fn(\U_i,\U_j)$, $\Fn_v(\U_i,\U_j, \nabla \U_i, \nabla \U_j)$ are
functions of the approximate field on both sides. Inviscid numerical
flux $\Fn(\U_i,\U_j)$ is typically computed with an approximate Riemann solver which
will be specified for each numerical test in the paper.
Numerical viscous flux $\Fn_v(\U_i,\U_j, \nabla \U_i, \nabla \U_j)$
in this paper follows the practice of Wang \cite{wang2017compact_VR}.
The integration of flux terms on the cell interfaces
are conducted using numerical quadrature rules with
enough algebraic precision.

As the spatial derivatives are approximated in \eqref{eq:FVInt},
it is referred to as the semi-discretized form of finite volume method.
The inviscid term has a truncation error of $O(h^{k+1})$ for smooth problems,
with $h$ being the size of mesh. As the approximate fields $\U_i$ are functions
of average values $\UM_i$, \eqref{eq:FVInt} can be rearranged into the
assembled form
\begin{equation}
    \label{eq:FVODE}
    \frac{d\uu}{dt} = \R(t, \uu)
\end{equation}
with

\begin{equation}
    \uu = \begin{bmatrix}
        \UM_1 \\
        \dots \\
        \UM_{N_{cell}}
    \end{bmatrix},\ \
    \R = \begin{bmatrix}
        -\sum_{j\in S_1, j\neq 1}{\left(
            \int_{f_{1,j}}{
                [\Fn - \Fn_v] \cdot \n  dA
            }\right)
        }     \\
        \dots \\
        -\sum_{j\in S_{N_{cell}}, j\neq N_{cell}}{\left(
            \int_{f_{N_{cell},j}}{
                [\Fn - \Fn_v] \cdot \n  dA
            }\right)
        }
    \end{bmatrix}.
\end{equation}

The arguments in the numerical fluxes are omitted.
The right-hand-side vector $\R$ is also a function of $t$
because boundary conditions or additional source terms
could depend on $t$ generally speaking.




\subsection{Variational reconstruction}
\label{ssec:VR}
In order to determine the coefficients of polynomial bases $\U_i^l$
(or $u_i^l$) in
\eeqref{eq:FVRec}, a reconstruction method needs to be specified.
Traditional second-order FV methods for unstructured grid
needs only to reconstruct a $k=1$ polynomial, namely linear
distribution on each cell.
The variational reconstruction \cite{wang2017compact_VR}
is a compact high-order
reconstruction scheme, which features high-order
accuracy achieved on a compact stencil.
The current section will explain the variational reconstruction
and specify relevant details.

The current paper uses local zero-mean Taylor basis, similar to
\cite{wang2017compact_VR}.
Taking 2 dimensional polynomials as an example:
\begin{equation}
    \begin{aligned}
        \varphi_{i,l} & =
        \left(\frac{x - x_{c,i}}{\inc x_i}\right)^{p_l}
        \left(\frac{y - y_{c,i}}{\inc y_i}\right)^{q_l}
        -
        \overline{
            \left(\frac{x - x_{c,i}}{\inc x_i}\right)^{p_l}
            \left(\frac{y - y_{c,i}}{\inc y_i}\right)^{q_l}
        }
    \end{aligned}
\end{equation}
where $[x_{c,j},y_{c,j}]\trans = \x_{c,j}$ are barycenters of cell $\OO_{j}$,
$\inc x_i,\inc y_i$ are the cell's length scale, and $p_l,q_l$ are
powers of the bases. For $k=3$ basis used in two-dimensional cases,
$p_l=[1,0,2,1,0,3,2,1,0]_l$,
and $q_l=[0,1,0,1,2,0,1,2,3]_l$.
The mean value term makes the basis have zero mean values,
which is calculated with:
\begin{equation}
    \overline{
        \left(\frac{x - x_{c,i}}{\inc x_i}\right)^{p_l}
        \left(\frac{y - y_{c,i}}{\inc y_i}\right)^{q_l}
    }
    =
    \frac{1}{\overline{\OO}_j}\int_{\OO_j}{
        \left(\frac{x - x_{c,i}}{\inc x_i}\right)^{p_l}
        \left(\frac{y - y_{c,i}}{\inc y_i}\right)^{q_l}
    }\dd \Omega .
\end{equation}

The length scales are chosen to be the largest distance of nodes from the
barycenter, which has the from
\begin{equation}
    \inc x_j = \inc y_j = \max_k(\|\x_{j,k} - \x_{c,j}\|_2)
\end{equation}
where $\x_{j,k}$ are coordinates of nodes belonging to cell $\OO_{j}$.

The variational reconstruction defines the solution of reconstruction coefficients
to be the minimum point of a globally defined functional $I$
\begin{equation}
    I = \sum{I_f}
\end{equation}
where $I_f$ are interface jump integrations (IJI) defined on each cell interface $f$.
Hereafter, we use $f=f_{ij}$ to denote the interface of cell $\OO_{i}$ and $\OO_{j}$ by default.
Using the two-dimensional case as an example, the current paper uses IJI in the form of
\begin{equation}
    I_f = \omega_f^G\int_{\OO_i \cap \OO_j}{
        \sum_{p+q=0}^{p + q\leq k}\left[
            \omega_f^D(p,q)
            d_{ij}^{p+q}
            \left(
            \partialderivative{^{p+q}u_i(x,y)}{x^p\partial y^q}
            -
            \partialderivative{^{p+q}u_j(x,y)}{x^p\partial y^q}
            \right)
            \right]^2
        \dd f
    }
\end{equation}
where $\omega_f^G$ is geometric weight and
$\omega_f^D$ is derivative weight.
The facial length scale  $d_{ij}$
is taken as
\begin{equation}
    d_{ij} = \|\x_{c,i}-\x_{c,j}\|_2
\end{equation}
which is the distance between the barycenters of the cells.
The IJI represents facial discontinuities of reconstruction polynomials using
function values and their various partial derivatives on the interface.
For 2-D simulation and cubic (with $k=3$)
reconstruction, the current paper uses
\begin{equation}
    \begin{aligned}
        \omega_f^D(0,0) & = 1\times \omega_D(0)                                                                                    \\
        \omega_f^D(1,0) & = \omega_f^D(0,1) = 1\times \omega_D(1)                                                                  \\
        \omega_f^D(2,0) & = \omega_f^D(0,2) = 1\times \omega_D(2),\ \omega_f^D(1,1) = \sqrt{2}\times \omega_D(2)                   \\
        \omega_f^D(3,0) & = \omega_f^D(0,3) = 1\times \omega_D(3),\ \omega_f^D(1,2) = \omega_f^D(2,1) = \sqrt{3}\times \omega_D(3) \\
    \end{aligned}
    \label{eq:wdRotRatio}
\end{equation}
with
\begin{equation}
    \omega_D(0) = 1, \omega_D(1) = 0.5925, \omega_D(2) = \omega_D(3) = 0.2117
    \label{eq:wdHQMOPT}
\end{equation}
as the derivative weight.
This form of derivative weight in \eeqref{eq:wdRotRatio} is rotational invariant and
isotropic.
The coefficients in \eeqref{eq:wdHQMOPT} are taken from \cite{huang2022high}, which were
obtained through the
optimization of dissipation and dispersion relations.

The geometric weight uses the form
\begin{equation}
    \omega_f^G = \sqrt{S^{\frac{1}{d-1}}d_{ij}^{-1}}
\end{equation}
where $d$ is the number of spatial dimension and $S$ is the area of the interface.
This form of
geometric weight is also taken from \cite{huang2022high}.

The minimization problem
\begin{equation}
    u_i^l = \mathop{\arg \min}(I)
\end{equation}
defines the solution to the variational reconstruction system.
It is observed that $I$ is a quadratic function of $u_i^l, i = 1...N_{cell}$ and
$I\geq 0$,
therefore $I$ is a positive semi-definite quadratic form about the global vector of reconstruction
coefficients. Consequently, the minimization problem converts to a linear equation
\begin{equation}
    \partialderivative{I}{u_i^l} = 0, l=1,2,\dots \mathrm{NDOF}(k),\ \ i=1,2,\dots N_{cell} .
\end{equation}
This linear equation can be arranged in a cell-block form:
\begin{equation}
    \label{eq:vrBlockEq}
    \sum_{f=f_{ij},j\in S_i, j\neq i}{ \mathbf{A}_{ij} \us_i}
    =
    \sum_{f=f_{ij},j\in S_i, j\neq i}{ \mathbf{B}_{ij} \us_j + \mathbf{b}_{ij} (\overline{u}_j - \overline{u}_i)}
\end{equation}
where $S_i$ is the compact stencil of $i$ composed of cell $\OO_{i}$ and its face neighbors.
Local reconstruction coefficient vector is $\us_i = [u_i^1,u_i^2\dots]\trans$.
The reconstruction coefficient matrices and vectors in \eeqref{eq:vrBlockEq}
are
\begin{equation}
    \label{eq:vrCoeffs}
    \begin{aligned}
        \mathbf{A}_{ij} = &
        \left[
            \left({
                \omega_f^G\int_{\OO_i \cap \OO_j}{
                    \sum_{p+q=0}^{p + q\leq k}
                    \omega_f^D(p,q)^2
                    d_{ij}^{2(p+q)}
                    \partialderivative{^{p+q}\varphi_{i,r}}{x^p\partial y^q}
                    \partialderivative{^{p+q}\varphi_{i,l}}{x^p\partial y^q}
                    \dd f}
            }\right)_{lr}
        \right]_{\mathrm{NDOF}(k) \times \mathrm{NDOF}(k)}, \\
        \mathbf{B}_{ij}=  &
        \left[
            \left({
                \omega_f^G\int_{\OO_i \cap \OO_j}{
                    \sum_{p+q=0}^{p + q\leq k}
                    \omega_f^D(p,q)^2
                    d_{ij}^{2(p+q)}
                    \partialderivative{^{p+q}\varphi_{j,r}}{x^p\partial y^q}
                    \partialderivative{^{p+q}\varphi_{i,l}}{x^p\partial y^q}
                    \dd f}
            }\right)_{lr}
        \right]_{\mathrm{NDOF}(k) \times \mathrm{NDOF}(k)}, \\
        \mathbf{b}_{ij}=  &
        \left[
            \left({
                    \omega_f^G\omega_f^D(0,0)^2
                    \int_{\OO_i \cap \OO_j}{
                        \varphi_{i,l}
                        \dd f}
                }\right)_{l}
        \right]_{\mathrm{NDOF}(k) \times 1}.                \\
    \end{aligned}
\end{equation}

The local coefficients matrices and vectors \eeqref{eq:vrCoeffs} can be
calculated at the start of computation.
The linear reconstruction system \eeqref{eq:vrBlockEq}
is solved with block-SOR method.
The details of linear solving of variational reconstruction and the treatments of
boundary conditions can be found in \cite{wang2017compact_VR}.
The fact to be emphasized here is that
the reconstruction is implicit, and
the reconstruction coefficients must be
obtained through a series of
iterations.
To ensure the efficiency of the numerical method,
the variational reconstruction is used
together with temporally implicit discretization
of the governing equations.
During an implicit solving procedure,
the updating of mean values and reconstruction
coefficients are decoupled
and executed alternately, with the block-SOR
iteration carried out only once for every iteration
of the implicit marching scheme.
As each block-SOR updating is compact in the sense of data dependency,
the variational reconstruction is able to have better data locality and
less communication during the iteration process.



% TODO
% \#\#\#


\section{Direct integration with temporal reconstruction (DITR)
  methods for high order time marching}
\label{sec:HIRK}

\subsection{General ideas}

Considering the first order ODE equation \eqref{eq:FVODE} arising
from high-order finite volume method
with
$t\in[0,\infty)$ and $\uu\in \mathbb{R}^N$:
\begin{equation*}
    \frac{\dd \uu}{\dd t} = \R(t, \uu)
\end{equation*}
which can be
considered a more general first order ODE with $\uu$ being
a general solution vector.
For example, in finite difference schemes,
$\uu$ represents the vector of grid point values.
And for finite volume schemes, $\uu$ is the vector
of all cell averages.
A direct integration of equation \eqref{eq:FVODE}
leads to a time-marching relation
\begin{equation}
    \uu^{n+1} = \uu^{n} + \int_{t^n}^{t^{n+1}}{
    \R(t, \uu) \dd t
    }
    \label{eq:DI}
\end{equation}

In order to acquire the integration result in equation
\eqref{eq:DI}, a numeric quadrature rule in the interval
$[t^n, t^{n+1}]$ is used. For example, the current paper
uses three point polynomial quadrature rule
\begin{equation}
    \label{eq:Quad3}
    \begin{aligned}
        \frac{\uu^{n+1} - \uu^{n}}{\inc t^n} = & \frac{1}{\inc t^n}
        \int_{t^n}^{t^{n+1}}{
        \R(t, \uu) \dd t
        }                                                           \\ \approx &
        b_1\R(t^n, \uu(t^n))
        +
        b_2\R(t^{n+c_2}, \uu(t^{n+c_2}))
        +
        b_3\R(t^{n+1}, \uu(t^{n+1}))
    \end{aligned}
\end{equation}
where $t^{n+c_2} = t^{n} + c_2 (t^{n+1} - t^n)$ and $c_2\in(0,1)$.
$\inc t^n=t^{n+1} - t^n$ is
the time step size.
The parameter $c_2$ represents the relative place of the second abscissa
in the quadrature rule, where the first and third fixed at $t^{n}$ and $t^{n + 1}$.
Using quadratic polynomial interpolation,
the weights of the quadrature rule will be:
\begin{equation}
    \begin{aligned}
        b_1 & = \frac{1}{2} - \frac{1}{6{c_2}},     \\
        b_2 & = \frac{1}{6{c_2}(1-{c_2})},          \\
        b_3 & = \frac{1}{2} - \frac{1}{6(1-{c_2})}. \\
    \end{aligned}
    \label{eq:integ0}
\end{equation}
The quadrature rule defined with
\eeqref{eq:Quad3} and \eeqref{eq:integ0} has algebraic precision
of degree 2. When $c_2=1/2$, the quadrature rule has algebraic precision of
degree 3 and becomes the three point Gauss-Lobatto rule.
The numeric integration process used in equation \eqref{eq:Quad3}
is referred to as a direct integration process because it is
directly derived from the ODE.


% The precise value of $\uu(t)$ at each $t\in[t^n, t^{n+1}]$
% is unknown, thus an approximation is required for numeric calculation.
% The current paper proposes to use conditions at
% time steps only, namely $t \in \{t^{n+1}, t^n, t^{n-1}\dots\}$,
% in acquiring the polynomial reconstruction.
% Since the step values $\uu^n, \uu^{n-1}\dots$ and
% their temporal derivatives $\R^n, \R^{n-1}\dots$ are known,
% the addition of the unknown latest step $\uu^{n+1}, \R^{n+1}$
% makes the method implicit.

Besides $\uu^{n+1}$, $\uu(t^{n+c_2})$ is also unknown.
To make \eeqref{eq:integ0}
solvable, the temporal reconstruction is introduced in the
present paper.
The temporal reconstruction is to reconstruct $\uu(t)$
from known $\uu$ at previous time steps and the desired
solution $\uu^{n+1}$ of the numerical schemes. Since $\frac{\dd\uu}{\dd t}=\R$
can be computed from  $\uu$, R at previous time steps as well
as  $\R^{n+1}$ can be also used in the temporal reconstruction.
The inclusion of $\uu^{n+1}$ and $\R^{n+1}$ in the reconstruction
makes the scheme implicit.
Using the reconstructed $\uu(t)$, $\uu(t^{n+c_2})$
can be evaluated, making \eeqref{eq:integ0} solvable.
The current paper has only considered using a subset of
$\uu(t), \R(t, \uu(t))$ at $t \in \{t^{n+1}, t^n, t^{n-1}\}$
as conditions of the polynomial reconstruction.
Generally, the polynomial interpolation
of $\uu(t)$ could be expressed as
\begin{equation}
    \begin{aligned}
        \label{eq:TR}
        \uu(t) & \approx
        A^n_0(t)\uu^{n - 1} +
        A^n_1(t)\uu^{n} +
        A^n_2(t)\uu^{n + 1}
        \\ & +
        \inc t^n D^n_0(t)\R^{n - 1} +
        \inc t^n D^n_1(t)\R^{n} +
        \inc t^n D^n_2(t)\R^{n + 1}
    \end{aligned}
\end{equation}
where $A^n_i(t), D^n_i(t), i=0,1,2$
are polynomial base functions. Due to the precision of quadrature rule,
the polynomial reconstruction is only expected to reach 3rd order
constructing a 4th order accurate scheme.
Consequently, at most 4 out of the
6 conditions would be used at the same time.
The polynomial interpolation in the direction of time in \eeqref{eq:TR}
is referred to as temporal reconstruction, for
a continuous distribution of $\uu$ is reconstructed with point values,
similar to the finite volume reconstruction.

Combining the direct integration in \eeqref{eq:Quad3}
and a  temporal reconstruction in \eeqref{eq:TR},
a Direct Integration with Temporal Reconstruction (DITR) method is
determined.
The current paper sticks to the same formula of direct integration
in \eeqref{eq:Quad3}, while experimenting on different
forms of temporal reconstruction in \eeqref{eq:TR}.

As a remark, the Adams-Moulton scheme which is one of the
implicit linear multistep method also use the direct
integration approach.
However, it relies on the
reconstruction of $\R(t)$ to make the numerical quadrature
similar to the right-hand side of \eeqref{eq:integ0} computable.
For the third order scheme, quadratic reconstruction in terms
of $\R^{n+1},\R^{n},\R^{n-1}$ is used.
It is well known that the high-order
Adams-Moulton schemes have rather poor stability
property. The present DITR scheme reconstructs U rather
than R. According to \eeqref{eq:TR},
more information can be used
in the reconstruction, making the present method more
compact and flexible. The present approach can achieve
high-order of accuracy and better stability property when
the reconstruction is designed properly.

If the quadrature rule in direct integration is replaced with
midpoint rule, and the temporal reconstruction uses linear
reconstruction, the implicit midpoint method for ODE can be derived,
which is 2nd order accurate. A high order accurate DITR method could
only be obtained with a high order accurate quadrature rule and
high degree temporal reconstruction.

Assuming the quadrature rule has algebraic precision
of degree $m$, and the polynomial interpolation is of degree $n$,
a straightforward analysis on local truncation error could be conducted.
Approximation \eqref{eq:Quad3} yields a
truncation error of $O((\inc t)^{m+2} )$ expressed in equation \eqref{eq:Quad3Err}
\begin{equation}
    \label{eq:Quad3Err}
    \begin{aligned}
        \uu^{n+1} - \uu^{n} = & \int_{t^n}^{t^{n+1}}{
        \R(t, \uu)\dd t}                              \\  = &
        {\inc t}^{n}
        \left[
            b_1\R(t^n, \uu(t^n))
            +
            b_2\R(t^{n+c_2}, \uu(t^{n+c_2}))
            +
            b_3\R(t^{n+1}, \uu(t^{n+1}))
            \right]
        \\ + &
        O((\inc t^{n})^{m+2} )
    \end{aligned}
\end{equation}
due to the precision degree of quadrature rule.
Approximation
\eeqref{eq:TR} has a truncation error of $O((\inc t)^{n+1})$
expressed in \eeqref{eq:TRErr}
\begin{equation}
    \begin{aligned}
        \label{eq:TRErr}
        \uu(t^{n+c_2}) & =
        A^n_0(t^{n+c_2})\uu^{n - 1} +
        A^n_1(t^{n+c_2})\uu^{n} +
        A^n_2(t^{n+c_2})\uu^{n + 1}
        \\ & +
        \inc t^n D^n_0(t^{n+c_2})\R^{n - 1} +
        \inc t^n D^n_1(t^{n+c_2})\R^{n} +
        \inc t^n D^n_2(t^{n+c_2})\R^{n + 1}
        \\ & +
        O((\inc t^{n})^{n+1} )
    \end{aligned}
\end{equation}
as a result of polynomial degree.
Substituting \eeqref{eq:TRErr} into the $t^{n+c_2}$ stage
of \eeqref{eq:Quad3Err}, the truncation error
of the entire scheme becomes $O((\inc t)^{n+2}) + O((\inc t)^{m+2})$:
\begin{equation}
    \begin{aligned}
        \uu^{n+1} = & \uu^{n} + {\inc t}^{n}
        \left[
            b_1\R(t^n, \uu^n)
            +
            b_2\R(t^{n+c_2}, \uu^{n+c_2} + O((\inc t^{n})^{n+1} ))
            +
            b_3\R(t^{n+1}, \uu^{n+1})
            \right]
        \\ + &
        O((\inc t^{n})^{m+2} )               \\
        =           &
        \uu^{n} + {\inc t}^{n}
        \left[
            b_1\R(t^n, \uu^n)
            +
            b_2\R(t^{n+c_2}, \uu^{n+c_2})
            +
            b_3\R(t^{n+1}, \uu^{n+1})
            \right]
        \\ + &
        O((\inc t^{n})^{m+2}  + O((\inc t^{n})^{n+2} ))
    \end{aligned}
    \label{eq:fullLTE}
\end{equation}
where $\uu^{n+1}, \uu^{n}$ are step values assumed to be accurate here, and
$\uu^{n+c_2}$ is the approximate
$c_2$ stage value determined by the temporal reconstruction.
Equation \eqref{eq:fullLTE} assumes
$\R(t, \uu)$ to be sufficiently regular and therefore does not
change the order of error.
Therefore, for smooth problems the
order of accuracy of a DITR
method is theoretically $\min(m,n) + 1$.

The following sections will illustrate some
practical DITR methods based on equations \eqref{eq:Quad3} and \eqref{eq:TR}.


\subsection{Variants of DITR methods}



\subsubsection{The DITR U2R2 method}

To make the method single-step and 4th order accurate,
we choose $\uu^{n},\uu^{n+1}$, $\R^{n},\R^{n+1}$ as
the interpolation conditions, making the interpolation
basically cubic Hermite interpolation.
Therefore, the interpolation of $\uu(t^{n+c_2})$
is:
\begin{equation}
    \begin{aligned}
        \label{eq:TRU2R2}
        \uu^{n+c_2} & =
        a_{1,U2R2}\uu^{n} +
        a_{2,U2R2}\uu^{n + 1}
        \\ & +
        \inc t^n d_{1,U2R2}\R^{n} +
        \inc t^n d_{2,U2R2}\R^{n + 1}
    \end{aligned}
\end{equation}
with $\uu^{n+c_2}$ being
the numerical approximation
of $\uu(t^{n+c_2})$
and the interpolation bases at $c_2$ node being:
\begin{equation}
    \begin{aligned}
        a_{1,U2R2} & = 1 - (3{c_2}^2 - 2 {c_2}^3) , \\
        a_{2,U2R2} & = 3{c_2}^2 - 2 {c_2}^3 ,       \\
        d_{1,U2R2} & = {c_2} - 2 {c_2}^2 + {c_2}^3, \\
        d_{2,U2R2} & = - {c_2}^2 + {c_2}^3 .        \\
    \end{aligned}
    \label{eq:interpU2R2}
\end{equation}

The temporal reconstruction of \eeqref{eq:TRU2R2}
combined with direct integration equation \eqref{eq:Quad3}
forms the DITR U2R2 method.
Here U2R2 denotes 2 $\uu$ and 2 $\R$
step values at $n,n+1$ used in the reconstruction.
Following analyses indicate that using
the most recent $\uu$ and $\R$
will yield a scheme with better stability.
Therefore, the current paper does not include the
time step indices in the naming of DITR methods.
This naming convention is continued in the following
methods.

When $c_2=1/2$, quadrature rule in equation \eqref{eq:Quad3} has precision
of degree 3, making the DITR U2R2 method 4th order accurate,
and the stage value $\uu^{n+c_2}$ has a precision of degree 3.
When $c_2\neq1/2$, DITR U2R2 becomes 3rd order accurate.

In order to further examine the accuracy order,
equations \eqref{eq:interpU2R2}, \eqref{eq:Quad3}
can be reformulated into a standard IRK method,
yielding a Butcher tableau shown in Table \ref{tab:U2R2Butcher}.
\begin{table}[htbp]
    \centering
    \begin{tabular}{c|ccc}
        0     & 0              & 0        & 0              \\
        $c_2$ & $d_1 + a_2b_1$ & $a_2b_2$ & $d_2 + a_2b_3$ \\
        1     & $b_1$          & $b_2$    & $b_3$          \\ \hline
              & $b_1$          & $b_2$    & $b_3$
    \end{tabular}
    \caption{Butcher tableau of DITR U2R2}
    \label{tab:U2R2Butcher}
\end{table}

According to table \ref{tab:U2R2Butcher} with the coefficients
determined by \eqref{eq:interpU2R2} and \eqref{eq:integ0},
one can find that the 4th order accurate
DITR U2R2 method with $c_2=1/2$
is actually the Lobatto IIIA method
of order 4 \cite{wanner1996solving}.
The classic order and stage order of DITR U2R2 could
also be verified using Table \ref{tab:U2R2Butcher} via
the simplifying assumptions, which is a trivial procedure
given the formulae provided in \cite{wanner1996solving}.

Following standard analysis based on Dahlquist's equation
$\frac{dy}{dt} = \lambda y$ \cite{wanner1996solving},
the stability function giving by $y^{1}=R(h\lambda)y^0$
applied to DITR U2R2 is in the form:
\begin{equation}
    \label{eq:stabilityFuncU2R2}
    R_{U2R2}(z) = -\frac{4\,z-2\,c_{2}\,z-c_{2}\,z^2+z^2+6}{2\,z+2\,c_{2}\,z-c_{2}\,z^2-6}
\end{equation}
which becomes the (2,2)-Pad{\'e} approximation when $c_2=1/2$ and
DITR U2R2 becomes Lobatto IIIA.
Analysis on equation \eqref{eq:stabilityFuncU2R2}
would show that $c_2\in[1/2,1)$ is a sufficient and necessary
condition of DITR U2R2 being $A$-stable given $c_2\in(0,1)$.
The limit at infinity
\begin{equation}
    \lim_{z\rightarrow\infty}R_{U2R2}(z) = \frac{1-c_2}{c_2}
\end{equation}
confirm that DITR U2R2 is unable to achieve $L$-stability
by adjusting $c_2$.


DITR U2R2 ($c_2=1/2$) or Lobatto IIIA method is symmetric,
which is a preferable property when integrating
reversible systems including orbital motion and particle
systems.
However,
the symmetry in this ODE method could be considered harmful in CFD application.
Most CFD systems of interest are physically dissipative,
while for a symmetric method
$R(z) \rightarrow 1$ when $z \rightarrow \infty$,
which means the method is more likely to preserve
spurious modes arising from spatial discretization.
Although DITR U2R2 cannot achieve $L$-stability,
using a value of $c_2 > 1/2$ would still
produce $\lim_{z\rightarrow\infty}R(z)\in(0,1)$, which
would be a useful property in simulation of dissipative systems.
With  $c_2 > 1/2$, stiff modes could vanish faster over the time
steps, while $c_2 = 1/2$ tends to preserve them.

\subsubsection{The DITR U2R1 method}

Giving up one interpolation condition in DITR U2R2
forces the scheme to have 3rd order accuracy.
The current paper removes $\R^{n}$ from U2R2, namely
using $\uu^{n},\uu^{n+1}$, $\R^{n+1}$ for interpolation,
which is able to produce an $L$-stable DITR scheme.

Similar with U2R2, DITR U2R1 has the interpolation
written as:
\begin{equation}
    \begin{aligned}
        \label{eq:TRU2R1}
        \uu^{n+c_2} & =
        a_{1,U2R1}\uu^{n} +
        a_{2,U2R1}\uu^{n + 1}
        \\ & +
        \inc t^n d_{2,U2R1}\R^{n + 1}
    \end{aligned}
\end{equation}
with $\uu^{n+c_2}$ being
the numerical approximation
of $\uu(t^{n+c_2})$
and the interpolation bases at $c_2$ node being:
\begin{equation}
    \begin{aligned}
        a_{1,U2R1} & = 1 - (2c_2 - {c_2}^2), \\
        a_{2,U2R1} & = 2c_2 - {c_2}^2,       \\
        d_{2,U2R1} & = {c_2}^2 - {c_2}.      \\
    \end{aligned}
    \label{eq:interpU2R1}
\end{equation}

The temporal reconstruction equation \eqref{eq:TRU2R1}
combined with direct integration equation \eqref{eq:Quad3}
forms the DITR U2R1 method.

The interpolation bases shown in \eqref{eq:interpU2R1}
are quadratic.
Therefore, the scheme yields 3rd order accuracy
and the choice of $c_2$ does not affect the order of accuracy.
Similar to U2R2, DITR U2R1's order of accuracy can
be examined using
standard procedures for
Runge-Kutta methods\cite{wanner1996solving}.

The linear stability function for DITR U2R1 is
\begin{equation}
    \label{eq:stabilityFuncU2R1}
    R_{U2R1}(z) = \frac{2\,z+6}{z^2-4\,z+6},
\end{equation}
which is (1,2)-Pad{\'e} approximation and not affected by $c_2$.
It can be found $|R_{U2R1}(z)| < 1, \forall \real(z) < 0$,
and obviously $\lim_{z\rightarrow\infty}R_{U2R1}(z) = 0$.
Therefore, DITR U2R1 method is $L$-stable.

The stability function also
shows that when $\R$ is a linear function of $\uu$,
the solution is not
affected by $c_2$.
However, for nonlinear $\R$,
$c_2$ changes the behavior of U2R1.

\subsubsection{The DITR U3R1 method}

To exploit the information when multiple previous
steps are available,
using conditions from $t^{n-1}$ would be
preferable.
The current paper chooses $\uu^{n-1},\uu^{n}$, $\uu^{n+1}$
and $\R^{n+1}$ as U3R1's interpolation conditions,
as other choices do not produce sufficient linear stability.
The interpolation at $c_2$ node becomes:
\begin{equation}
    \begin{aligned}
        \label{eq:TRU3R1}
        \uu^{n+c_2} & =
        a_{0,U3R1}\uu^{n} +
        a_{1,U3R1}\uu^{n} +
        a_{2,U3R1}\uu^{n + 1}
        \\ & +
        \inc t^n d_{2,U3R1}\R^{n + 1}
    \end{aligned}
\end{equation}
with $\uu^{n+c_2}$ being
the numerical approximation
of $\uu(t^{n+c_2})$
and the interpolation bases at $c_2$ node being:
\begin{equation}
    \begin{aligned}
        a_{0,U3R1} & = -\frac{c_{2}\,{\left(c_{2}-1\right)}^2}{\Theta\,{\left(\Theta+1\right)}^2}       ,                                                      \\
        a_{1,U3R1} & = \frac{\left(\Theta+c_{2}\right)\,{\left(c_{2}-1\right)}^2}{\Theta}                     ,                                                \\
        a_{2,U3R1} & =  \frac{c_{2}\,\left(-\Theta^2\,c_{2}+2\,\Theta^2-\Theta\,{c_{2}}^2+3\,\Theta-2\,{c_{2}}^2+3\,c_{2}\right)}{{\left(\Theta+1\right)}^2} , \\
        d_{2,U3R1} & =   \frac{c_{2}\,\left(\Theta+c_{2}\right)\,\left(c_{2}-1\right)}{\Theta+1}.
    \end{aligned}
    \label{eq:interpU3R1}
\end{equation}
where $\Theta = \inc t^{n-1}/\inc t^{n}$.

The temporal reconstruction equation \eqref{eq:TRU3R1}
combined with direct integration equation \eqref{eq:Quad3}
forms the DITR U3R1 method.

The method is 4th order when $c_2=1/2$, when
both the interpolation and integration has precision
of degree 3.
For the special case of $c_2=1/2$, simplified
coefficients are given:
\begin{equation}
    \begin{aligned}
        a_{0,U3R1} & = -\frac{1}{8\,\Theta\,{\left(\Theta+1\right)}^2}  ,              \\
        a_{1,U3R1} & = \frac{\Theta+\frac{1}{2}}{4\,\Theta}        ,                   \\
        a_{2,U3R1} & =  \frac{6\,\Theta^2+11\,\Theta+4}{8\,{\left(\Theta+1\right)}^2}, \\
        d_{2,U3R1} & = -\frac{\Theta+\frac{1}{2}}{4\,\left(\Theta+1\right)}.
    \end{aligned}
    \label{eq:interpU3R1-S}
\end{equation}

Linear stability is analyzed when $\Theta =1$ and $c_2 = 1/2$.
The solution to the test problem
produces two solutions:
\begin{equation}
    \left(
    \begin{matrix}
        R_{U3R1}^{(1)}(z) \\
        R_{U3R1}^{(2)}(z)
    \end{matrix}
    \right)=\left(\begin{array}{c} \frac{10\,z-\sqrt{-6\,z^3+129\,z^2+432\,z+576}+24}{6\,z^2-29\,z+48}\\ \frac{10\,z+\sqrt{-6\,z^3+129\,z^2+432\,z+576}+24}{6\,z^2-29\,z+48} \end{array}\right)
    .
\end{equation}
It is obvious $R_{U3R1}^{(1)}(z)\rightarrow 0, R_{U3R1}^{(2)}(z)\rightarrow0$ when
$z\rightarrow\infty$.
When observed numerically, it is found both $|R_{U3R1}^{(1)}(z)|$
and $|R_{U3R1}^{(2)}(z)|$ are less than 1 in the left
half plane of $z$.
The current paper therefore believes %! not a linear multistep method
the DITR U3R1 method is indeed $L$-stable.
Note that DITR U3R1 is not a linear multistep method,
and the root locus curve analysis used in those methods
can not be directly applied here.
To the author's knowledge, no
such implicit time marching method
has been reported in the literature.

With 4th order accuracy and observed $L$-stability, the
DITR U3R1 method is potentially more favorable than
U2R2 and U2R1.

\subsubsection{Summary of DITR methods}
\label{sssec:sumDITRs}

The U2R2, U2R1 and U3R1 variants of the
DITR method can be written is a unified form.
The first equation is the direct integration:
\begin{equation}
    \uu^{n+1} = \uu^{n} + \inc t^n\left(
    b_1 \R^n +
    b_2 \R^{n+c_2} +
    b_3 \R^{n+1}
    \right)
    \label{eq:DISum}
\end{equation}
with weights decided by \eqref{eq:integ0},
and also listed Table \ref{tab:integ0Tab}.
For simplicity, in numerical expressions,
notations such as $\R^{n+c_2}=\R(t^{n+c_2}, \uu^{n+c_2})$
are used from now on.
\begin{table}[htbp]
    \centering
    \begin{tabular}{|c|c|c|}
        \hline
        $b_1$                            & $b_2$ & $b_3$ \\
        \hline
        $\frac{1}{2} - \frac{1}{6{c_2}}$ &
        $\frac{1}{6{c_2}(1-{c_2})}$      &
        $\frac{1}{2} - \frac{1}{6(1-{c_2})} $            \\
        \hline
    \end{tabular}
    \caption{Butcher tableau of DITR U2R2}
    \label{tab:integ0Tab}
\end{table}

The second equation is the temporal reconstruction:
\begin{equation}
    \label{eq:TRSum}
    \uu^{n+c_2}  =
    a^n_0\uu^{n - 1} +
    a^n_1\uu^{n} +
    a^n_2\uu^{n + 1}
    +
    \inc t^n
    \left(
    d^n_1\R^{n} +
    d^n_2\R^{n + 1}
    \right)
\end{equation}
where the coefficients vary in different schemes
following Table \ref{tab:inter0Tab}.
The time step ratio for the two-step U3R1 method is
$\Theta = \inc t^{n-1} / \inc t^{n}$.

\begin{table}[htbp]
    \centering
    \footnotesize
    \begin{tabular}{|c|c|c|c|}
        \hline
        DITR Method & U2R2                          & U2R1                 & U3R1                                                                                                                                   \\
        \hline
        $a^n_0$     & 0                             & 0                    & $-\frac{c_{2}\,{\left(c_{2}-1\right)}^2}{\Theta\,{\left(\Theta+1\right)}^2}$                                                           \\
        \hline
        $a^n_1$     & $1-(3{c_2}^2 - 2 {c_2}^3)$    & $1-(2c_2 - {c_2}^2)$ & $\frac{\left(\Theta+c_{2}\right)\,{\left(c_{2}-1\right)}^2}{\Theta}$                                                                   \\
        \hline
        $a^n_2$     & $3{c_2}^2 - 2 {c_2}^3$        & $2c_2 - {c_2}^2$     & $\frac{c_{2}\,\left(-\Theta^2\,c_{2}+2\,\Theta^2-\Theta\,{c_{2}}^2+3\,\Theta-2\,{c_{2}}^2+3\,c_{2}\right)}{{\left(\Theta+1\right)}^2}$ \\
        \hline
        $d^n_1$     & ${c_2} - 2 {c_2}^2 + {c_2}^3$ & 0                    & 0                                                                                                                                      \\
        \hline
        $d^n_2$     & $- {c_2}^2 + {c_2}^3$         & ${c_2}^2 - {c_2}$    & $\frac{c_{2}\,\left(\Theta+c_{2}\right)\,\left(c_{2}-1\right)}{\Theta+1}$                                                              \\
        \hline
    \end{tabular}
    \caption{Interpolation coefficients for different DITR methods}
    \label{tab:inter0Tab}
\end{table}

In practice, the current research chooses $c_2=0.5,0.55$ for U2R2.
U2R2 $c_2=0.5$ has optimal order of accuracy (4th order), but it is symmetric.
U2R2 $c_2=0.55$ is 3rd order accurate but breaks symmetry and gains some stability.

For U2R1, $c_2$ does not affect order of accuracy and stability.
We empirically choose $c_2=0.25$ for U2R1, the reason roots in
the convergence of solving PDEs using
dual time stepping, which will be explained in \ref{sssec:numScan}.

For U3R1, since any $c_2$ gives $L$-stability, so $c_2=0.5$ which
offers optimal 4th order of accuracy is always used.




% \subsection{Time Marching Based on Interpolation}

% The acquiring of an NIRK method with
% certain order of accuracy is usually based on
% a certain integration method and
% corresponding taylor expansion analysis,
% as described in \cite{kulikov2006familyNIRKOrig}.
% However, the present paper proposes a single step method
% derived with Hermite interpolation
% (thus called HIRK),
% which simplifies the acquiring of high
% stage order and provides an approach to
% modify the scheme's stability.

% Considering the first order ODE arising from high-order finite volume method
% with
% $t\in[0,\infty)$ and $\uu\in \mathbb{R}^N$:
% \begin{equation*}
%     \frac{d\uu}{dt} = \R(t, \uu)
% \end{equation*}
% which is directly taken from equation \eqref{eq:FVODE}.
% A direct integration of equation \eqref{eq:FVODE}
% leads to a time-marching relation
% \begin{equation}
%     \uu^{n+1} = \uu^{n} + \int_{t^n}^{t^{n+1}}{
%     \R(t, \uu)
%     }
% \end{equation}

% With $\uu^n$ as the known numerical solution at $t=t^n$ and
% $t^{n+1} = t^{n} + \inc t$,
% we first consider the construction of implicit second order
% time marching methods as an introduction.
% First, form a linear interpolation for $t\in[t^n, t^{n+1}]$,
% which gives an approximate time-distribution of
% $\uu(t)=\frac{t^{n+1} - t}{\inc t}\uu^n + \frac{t - t^n}{\inc t}\uu^{n+1}$.
% Second, use a second order accurate quadrature rule on
% the linear time distribution. Evidently a trapezoid rule yields
% trapezoid rule method or Crank-Nicolson method, and
% midpoint rule yields the second order midpoint method.
% Inspired by this procedure, now we proceed to construct
% a high-order version.

% For a single step implicit method,
% if the implicitness is already solved,
% the endpoint values $\uu^n, \uu^{n+1}$ are
% given. Moreover,  $\R^n, \R^{n+1}$ can be
% directly evaluated, where $\R^n = \R(t^n, \uu^n)$.
% Given the endpoint values and corresponding first derivatives,
% an Hermite interpolation can be given:
% \newcommand{\ttt}{t^*}
% \begin{equation}
%     \label{eq:Hermite}
%     \begin{aligned}
%         \uu(t) & \approx (1-3{\ttt}^2 - 2 {\ttt}^3)\uu^n
%         + (3{\ttt}^2 - 2 {\ttt}^3)\uu^{n+1}                     \\
%                & + \inc t ({\ttt} - 2 {\ttt}^2 + {\ttt}^3) \R^n
%         + \inc t (- {\ttt}^2 + {\ttt}^3) \R^{n+1}
%     \end{aligned}
% \end{equation}
% where $\ttt = (t - t^n)/\inc t$ and $\ttt \in [0,1]$.
% The Hermite interpolation \eqref{eq:Hermite}
% has $O(\inc t^4)$ truncation error.
% Next, a quadrature rule is applied.
% To utilize the endpoints and reduce memory requirement,
% we only insert one middle stage at $\ttt = c_2$.
% Using polynomial interpolation based
% numerical integration, with nodes $\ttt = 0,c_2,1$,
% the numerical integration of some function $r(t)$ would be
% \begin{equation}
%     \label{eq:Quad}
%     \begin{aligned}
%         \int_{t^n}^{t^{n+1}}r(t)dt & \approx \\
%         \left(\frac{1}{2} - \frac{1}{6{c_2}}\right)r(t^n)
%                                    & +
%         \left(\frac{1}{6{c_2}(1-{c_2})}\right)r(t^n + c_2\inc t )
%         +
%         \left(\frac{1}{2} - \frac{1}{6(1-{c_2})}\right)r(t^{n+1})
%     \end{aligned}
% \end{equation}
% using the relation $\int_{t^n}^{t^{n+1}}\R dt = \uu^{n+1} -\uu^{n}$,
% there are two algebraic relations and two unknown vectors
% in total, which is able to form a solvable system.
% The quadrature rule \eqref{eq:Quad} has truncation error
% $O(\inc t^3)$ at least,
% for it is based on second degree polynomial interpolation.


% Summarizing the discussions above, the HIRK method has the form:
% \begin{subequations}
%     \label{eq:HM3}
%     \begin{align}
%         \uu^{n+1} & = \uu^{n} +
%         \inc t
%         \left[
%             b_1\R(t^{n,1}, \uu^n) +
%             b_2\R(t^{n+c_2}, \uu^*) +
%             b_3\R(t^{n,3}, \uu^{n+1})
%         \right]   \label{eq:HM3-1} \\
%         \uu^{*}   & =
%         a_1\uu^{n} +
%         a_2\uu^{n+1} +
%         \inc t
%         \left[
%             d_1\R(t^{n,1}, \uu^n) +
%             d_2\R(t^{n,3}, \uu^{n+1})
%             \right] \label{eq:HM3-2}
%     \end{align}
% \end{subequations}
% where $t^{n,i}=t^n+c_i\inc t$, and $c_1 = 0, c_3 = 1, c_2\in(0,1)$.

% The only stage value besides endpoints is $\uu^*$.
% The internal stage $\uu^*$ can be explicitly
% calculated with \eqref{eq:HM3-2}, which is derived
% directly from the cubic Hermite interpolation of $\uu$
% on interval $[t^n,t^{n+1}]$ evaluated at
% $t^{n+c_2}$, which result in the interpolation relation:
% \begin{equation}
%     \begin{aligned}
%         a_2 & = 1 - a_1 = 3{c_2}^2 - 2 {c_2}^3 \\
%         d_1 & = {c_2} - 2 {c_2}^2 + {c_2}^3    \\
%         d_2 & = - {c_2}^2 + {c_2}^3            \\
%     \end{aligned}
%     \label{eq:interp}
% \end{equation}
% The first equation \eqref{eq:HM3-1} is
% a numeric integration on interval $[t^n,t^{n+1}]$ with
% polynomial nodes
% $t^{n,1}=t^n,t^{n+c_2}=t^n+c_2\inc t,t^{n,3}=t^{n+1}$, which demands:
% \begin{equation}
%     \begin{aligned}
%         b_1 & = \frac{1}{2} - \frac{1}{6{c_2}}     \\
%         b_2 & = \frac{1}{6{c_2}(1-{c_2})}          \\
%         b_3 & = \frac{1}{2} - \frac{1}{6(1-{c_2})} \\
%     \end{aligned}
%     \label{eq:integ}
% \end{equation}
% The Hermite interpolation gives \eqref{eq:HM3-2}
% local truncation error $O(\inc t^4)$,
% and the numeric integration gives \eqref{eq:HM3-1}
% local truncation error $O(\inc t^3)$, therefore the
% classic order of accuracy of HIRK \eqref{eq:HM3} is
% 3. If $c_2=1/2$, then the numeric integration \eqref{eq:HM3-2}
% becomes a Gauss-Lobatto quadrature rule, which yields
% a local truncation error of $O(\inc t^4)$, making
% the scheme 4th order accurate.
% Also, from the interpolation, no matter
% the choice of $c_2$, HIRK has a stage order of
% 3, making it stiffly accurate.



% \subsection{Order of Accuracy}
% To discuss the order of accuracy and stage order
% formally, HIRK can be considered as a standard IRK.
% Reformulating \eqref{eq:HM3} into
% a standard Runge-Kutta form yields a Butcher
% tableau:
% \begin{table}[htbp]
%     \centering
%     \begin{tabular}{c|ccc}
%         0     & 0              & 0        & 0              \\
%         $c_2$ & $d_1 + a_2b_1$ & $a_2b_2$ & $d_2 + a_2b_3$ \\
%         1     & $b_1$          & $b_2$    & $b_3$          \\ \hline
%               & $b_1$          & $b_2$    & $b_3$
%     \end{tabular}
%     \caption{Butcher tableau of \eqref{eq:HM3}}
%     \label{tab:HM3Butcher}
% \end{table}

% From table \ref{tab:HM3Butcher} with the coefficients
% decided with \eqref{eq:interp} and \eqref{eq:integ},
% one can find that the 4th order accurate
% HIRK $c_2=1/2$ method is indeed the Lobatto IIIA method
% of order 4 \cite{wanner1996solving}.
% The classic order and stage order of HIRK could
% also be evaluated from table \ref{tab:HM3Butcher} via
% the simplifying assumptions, which is a trivial procedure
% given the formulae provided in \cite{wanner1996solving}.

% If the quadrature rule in HIRK is
% replaced with Gauss-Legendre rule,
% the method immediately becomes a special case of the
% Gauss type NIRK method of order 4
% described in \cite{kulikov2006familyNIRKOrig}.
% For large scale CFD application,
% using 2 point Gauss-Legendre rule
% would mean deriving the right hand side at
% 3 different stages iteratively.
% Thus, the current HIRK method only considers
% a 3 point Gauss-Lobatto type quadrature
% (with the middle abscissa moved and order of accuracy reduced),
% which would provide sufficient accuracy and only require 2 unknown stages.



% \begin{equation}
%     \label{eq:HM3R}
%     \begin{aligned}
%         \uu^{n+1} & = \uu^{n} +
%         \inc t
%         \left(
%         b_1\R(t^{n,1}, \uu^n) +
%         b_2\R(t^{n+c_2}, \uu^*) +
%         b_3\R(t^{n,3}, \uu^{n+1})
%         \right) \\
%         \uu^{*}   & =
%         \uu^{n}  + 
%         \inc t
%         \left(
%         (d_1 + a_2b_1)\R(t^{n,1}, \uu^n) +
%         a_2b_2\R(t^{n+c_2}, \uu^*) +
%         (d_2 + a_2b_3)\R(t^{n,3}, \uu^{n+1})
%         \right)
%     \end{aligned}
% \end{equation}

% \subsection{Linear Stability}
% \label{ssec:linStab}

% Following standard analysis based on Dahlquist's equation
% $\frac{dy}{dt} = \lambda y$ \cite{wanner1996solving},
% the stability function giving $y^{1}=R(h\lambda)y^0$
% when applying HIRK is in the form:
% \begin{equation}
%     \label{eq:stabilityFunc}
%     R(z) = -\frac{4\,z-2\,c_{2}\,z-c_{2}\,z^2+z^2+6}{2\,z+2\,c_{2}\,z-c_{2}\,z^2-6}
% \end{equation}
% which becomes the (2,2)-Pad{\'e} approximation when $c_2=1/2$ and HIRK
% becomes Lobatto IIIA. The limit
% \begin{equation}
%     \lim_{z\rightarrow\infty}R(z) = \frac{1-c_2}{c_2}
% \end{equation}
% gives that a necessary condition for $A$-stability of
% HIRK is $c_2 \in [1/2,1)$, and shows that HIRK
% is unable to achieve $L$-stability.
% Further analysis on \eqref{eq:stabilityFunc}
% would confirm $c_2 \in [1/2,1)$ is a sufficient
% condition for $A$-stability.

% HIRK($1/2$) or Lobatto IIIA method is symmetric,
% which is a preferable property when integrating
% reversible systems,
% but the symmetry could be considered harmful in CFD application.
% Most CFD systems of interest are physically dissipative,
% while for symmetric RK methods
% $R(z) \rightarrow 1$ when $z \rightarrow \infty$,
% which is more likely to preserve
% spurious modes arising from spatial  discretization.
% Although HIRK cannot achieve $L$-stability
% by simply adjusting $c_2$,
% using a non-trivial value $c_2 > 1/2$ would still
% produce $\lim_{z\rightarrow\infty}R(z)\in(0,1)$, which
% is a useful property in simulation of dissipative systems.
% With  $c_2 > 1/2$, stiff modes could better vanish over the time
% steps, while $c_2 = 1/2$ tends to preserve them.


\section{Iteration methods for solving DITR}
\label{sec:Solving}

The DITR methods discussed in the current paper
consist of two implicit equations \eqref{eq:DISum} and
\eqref{eq:TRSum}, which are results of direct integration
and temporal reconstruction respectively.
There are two unknown vectors $\uu^{n+1},\uu^{n+c_2}$ in the equations.
In order to numerically conduct time marching with DITR, the equations
need to be iteratively solved.
Although the DITR schemes can be used as the ODE integrator
directly, we are mainly interested in using these schemes 
as the time marching schemes for solving nonlinear time-dependent
PDEs, specifically the compressible Euler and Navier-Stokes equations.



\subsection{Dual Time Stepping}
\label{ssec:dualTime} 

Dual time stepping technique \cite{jameson1991time,jameson2017evaluation}
is a standard procedure of solving nonlinear
implicit time stepping equations
in flow problems.
The nonlinear equations are written in the form of
\begin{equation}
    \FF^*(\uu^*) = 0
    \label{eq:abstractImplicit}
\end{equation}
where $\uu^*$ represent one of the unknowns
in DIRK, BDF or DITR methods.
For example, for trapezoid rule (Crank-Nicolson method),
equation \eqref{eq:abstractImplicit} reads
\begin{equation}
    \label{eq:trapF}
    \FF(\uu^{n+1}) =
    -\frac{\uu^{n+1}-\uu^{n}}{\inc t^n} + \frac{\R^{n} + \R^{n+1}}{2}
    =0
\end{equation}
Generally, $\FF^*$ should be arranged so that
$\R(\uu^*)$ in $\FF^*$ has coefficient being $O(1)$ and positive,
while $\uu^*$ has its coefficient being $O(\inc t^{-1})$ and negative.
With the idea of dual time stepping, namely
\begin{equation}
    \frac{\dd\uu^{*}}{\dd \tau} = \FF^*(\uu^{*})
    \label{eq:exactDualTime}
\end{equation}
combined with a backward Euler method on the
pseudo time $\tau$, equation \eqref{eq:abstractImplicit}
is solved with
\begin{equation}
    \frac{\uu^{*,m+1} - \uu^{*,m}}{\inc \tau} = \FF^*(\uu^{*,m+1})
\end{equation}
where $m, m+1$ are pseudo time steps.
Clearly $\tau$ has a dimension of time.
Further linearizing would produce
\begin{equation}
    \left(\frac{\eye}{\inc\tau} -
    \partialderivative{\FF^*}{\uu^*}  \right)
    \inc\uu^{*,m} = \FF^*(\uu^{*,m})
    \label{eq:abstractNewton}
\end{equation}
where $\inc\uu^{*,m} = \uu^{*,m+1} - \uu^{*,m}$ is the increment
of the iteration.
The Jacobian of residual $\partialderivative{\FF^*}{\uu^*}$
is a non-trivial matrix, thus each update of $\uu^*$ requires
the solution of a sparse linear problem.
When $\inc\tau\rightarrow0$, Equation \eqref{eq:abstractNewton}
tends to reflect the exact evolution of dual time stepping
equation \eqref{eq:exactDualTime}.
Equation \eqref{eq:abstractNewton} becomes the
standard Newton-Raphson method when $\inc\tau\rightarrow\infty$.

In this paper, the dual time stepping technique is considered to be crucial
in helping the numerical system overcome the convergence difficulties
arising from strong nonlinearity and discontinuity
in certain compressible flows.
The pseudo time term in dual time stepping
serves as a stabilizing term in the iteration,
which is necessary when the flow has strong transient features.
Therefore, the following sections will focus on different
methods of applying dual time iteration to DITR.

Dual time stepping \eeqref{eq:exactDualTime} is an artificial
pseudo-time-dependent ODE system, and linear stability features with
respect to $\tau$ can be performed \cite{jameson1991time}.
The current paper considers using Fourier analysis
to gain a perception of the stability features in pseudo time
when the form of $\FF^*$ varies.
Therefore, the current section will introduce the
Fourier analysis in pseudo time using standard BDF and DIRK
methods as an example.


\subsubsection{Fourier analysis in pseudo time}
\label{sssec:fourier}



To start the Fourier analysis, consider the linear advection problem
solved on an infinitely large 1-D uniform
grid.
Suppose the 1-D linear advection problem reads
\begin{equation}
    \partialderivative{u}{t} = -a \partialderivative{u}{x}
\end{equation}
with positive convection speed $a$.
Some difference scheme on uniform grid is used as spatial  discretization, which
yields the ODE system:
\begin{equation}
    R_i=\derivative{u_i}{t} = -\frac{a}{\inc x} \sum_{j\in S_i}u_j\alpha_j
    \label{eq:fd}
\end{equation}
where $\alpha_j$ is the coefficients for finite difference approximation
and $S_i$ is the stencil set. $R_i$ denotes the right-hand-side of the ODE,
or components of $\R$ in other words.
As the problem is linear,
a simple wave solution
\begin{equation}
    u = A(t)\exp(\imagUnit \frac{\kappa}{\inc x} x)
\end{equation}
can be substituted in and produces
a relation of
\begin{equation}
    \frac{\dd A(t)}{\dd t} = -\imagUnit a\frac{\kappa'}{\inc x}  A(t)
\end{equation}
where $\kappa'$ is the modified non-dimensional wave number,
and $\imagUnit$ is the imaginary unit.
For a finite difference scheme in equation \eqref{eq:fd},
$\kappa'$ can be found as
\begin{equation}
    \kappa' = \frac{1}{\imagUnit} \sum_{j\in S_i}\exp(\imagUnit (j-i)\kappa)\alpha_j
\end{equation}
The modified wave number $\kappa'$ is commonly used to evaluate the performance of
spatial  discretization schemes.
A stable discretization would require $\imag(\kappa) \leq 0$ so that
$A(t)$ does not exponentially increase in magnitude.

Next, the classic linear analysis on dissipation and dispersion is
extended to dual time stepping. The dual time
stepping equation \eqref{eq:exactDualTime} can be
considered the evolution equation of the solution in pseudo time.
Thus, for the linear advection problem,
a simple wave pseudo time solution
\begin{equation}
    u^*(\tau,x) = A^*(\tau)\exp(\imagUnit \frac{\kappa}{\inc x} x)
    \label{eq:simpleWaveTestTau}
\end{equation}
where $*$ represent a certain unknown stage or step value.
For standard BDF and ESDIRK methods, the residual $\FF^*$
can be written as
\begin{equation}
    \FF^*(\uu^*) = \mathbf{C} - \frac{\mathcal{A}}{\inc t} \uu^* + \mathcal{B}\R(t^*, \uu^*)
    \label{eq:canonicalFF}
\end{equation}
where $\mathbf{C}$ is the part of vectors irrelevant with $\uu^*$,
and $\mathcal{A},\mathcal{B}$ are positive real coefficients.
$\mathbf{C}$ contains the transient terms of ODE integrator,
and stays constant in the pseudo time iteration and is
therefore irrelevant to linear stability.
With  spatial discretization \eeqref{eq:fd},
time integrator \eeqref{eq:canonicalFF},
and
the simple wave solution \eeqref{eq:simpleWaveTestTau}
substituted into the exact pseudo time evolution \eeqref{eq:exactDualTime},
the wave amplitudes are found to have the form
\begin{equation}
    \frac{\dd A^*(\tau)}{\dd \tau} = -\imagUnit a\frac{\kappa^*}{\inc x}  A^*(\tau) + C^*
    \label{eq:kappaSDefTau}
\end{equation}
where $\kappa^*$ represents the modified wave number for pseudo time.
The $C^*$ term in the above equation is a constant coefficient arising
from the $\mathbf{C}$ term in the time stepping method.
Clearly only the linear term with $\kappa^*$ could induce an exponentially diverging
solution, while the constant term only changes the asymptotic solution
when the pseudo time evolution is stable.
The pseudo time modified wave number $\kappa^*$ is found to be
\begin{equation}
    \begin{aligned}
        \kappa^* & = -\imagUnit\frac{\mathcal{A}}{\CFLt} +
        \frac{\mathcal{B}}{\imagUnit} \sum_{j\in S_i}\exp(\imagUnit (j-i)\kappa)\alpha_j \\
                 & =-\imagUnit\frac{\mathcal{A}}{\CFLt} + \mathcal{B}\kappa'
    \end{aligned}
\end{equation}
where $\CFLt = \frac{a\inc t }{\inc x}$ is the physical CFL number.
When the finite difference scheme is stable,
the physical time modified wave number satisfies $\imag (\kappa') \leq 0$ for
all $\kappa$.
Consequently, $\imag (\kappa^*) < 0$, and the exact pseudo time marching
is stable.

To summarize, for stage residual function $\FF^*$ in the form of
equation \eqref{eq:canonicalFF}, it is guaranteed $\imag (\kappa^*) < 0$,
and the pseudo time evolution is stable in the sense of a continuous $\tau$.

This result clearly demonstrates the benefit of using the dual time stepping
technique,
that it is automatically stable when applied to common ODE methods including
BDF and ESDIRK.


\subsection{The fully coupled iteration method for solving DITR}

For linear multistep methods (including BDF) and DIRK type methods (including ESDIRK),
at each step or stage, only one $\uu^{*}=\uu^{n,s}$
($s$ is the stage index) or $\uu^{*}=\uu^{n+1}$ is unknown,
thus equation \eqref{eq:abstractNewton} can be applied directly.
However, in fully implicit RK methods and the current DITR methods,
more than one unknown $\uu^{*}$ are present at each stage.
For example, in DITR methods, both $\uu^{n+1},\uu^{n+c_2}$
are present in both equations \eqref{eq:DISum} and \eqref{eq:TRSum}.
Note that in equations \eqref{eq:DISum} and \eqref{eq:TRSum},
$\R^{n+1},\R^{n+c_2}$ are also dependent on $\uu^{n+1},\uu^{n+c_2}$.
Assuming $\R$ to be nonlinear, linear combinations of
equations \eqref{eq:DISum} and \eqref{eq:TRSum} are
impossible to completely eliminate
the dependence on one of $\uu^{n+1},\uu^{n+c_2}$, and
fully implicit RK methods share the same issue.

One of the most obvious solutions to the multiple unknowns
in DITR and fully implicit RK methods is to apply
equation \eqref{eq:abstractNewton} to an enlarged
system.
The enlarged system concatenates different $\uu$
stages as one vector and uses the concatenated vector as
$\uu^*$ in \eqref{eq:abstractNewton}.
The residual function $\FF^*$ is therefore a concatenated vector of
the residual vectors at different stages.
Therefore, for methods with $k$ unknown stages, the number of
unknowns solved concurrently is $k$ times of that in BDF or ESDIRK methods.
Applied to DITR methods, there is
\begin{equation}
    \uu^* = \begin{bmatrix}
        \uu^{n+c_2} \\
        \uu^{n+1}
    \end{bmatrix}
\end{equation}
and we denote
\begin{equation}
    \begin{aligned}
        \FF^* & =\begin{bmatrix}
            \FF^{n+c_2} \\
            \FF^{n+1}
        \end{bmatrix} = \Pm\begin{bmatrix}
            \GG^{n+c_2} \\
            \GG^{n+1}
        \end{bmatrix} \\ & = \Pm\begin{bmatrix}
            \frac{a^n_0\uu^{n - 1} +
                a^n_1\uu^{n} +
                a^n_2\uu^{n + 1} - \uu^{n+c_2}}{\inc t^n}
            +
            d^n_1\R^{n} +
            d^n_2\R^{n + 1}
            \\
            \frac{\uu^{n} - \uu^{n+1}}{\inc t^n}  +
            b_1 \R^n +
            b_2 \R^{n+c_2} +
            b_3 \R^{n+1}
        \end{bmatrix}
        \label{eq:DITRFFDef}
    \end{aligned}
\end{equation}
where $\Pm$ is a preconditioning matrix:
\begin{equation}
    \Pm = \begin{bmatrix}
        \Pe_{11}\eye & \Pe_{12}\eye \\
        \Pe_{21}\eye & \Pe_{22}\eye \\
    \end{bmatrix}
\end{equation}
with $\Pe_{11},\Pe_{12} \dots$ being real scalars.
Functions $\GG^{n+c_2},\GG^{n+1}$ are the original residuals
corresponding to temporal reconstruction \eeqref{eq:TRSum} and
direct integration \eeqref{eq:DISum} respectively. Note that
$\FF^{n+c_2},\FF^{n+1}$ and $\GG^{n+c_2},\GG^{n+1}$ are all
dependent on both $\uu^{n+c_2},\uu^{n+1}$.
And consequently \eeqref{eq:exactDualTime} becomes
\begin{equation}
    \begin{bmatrix}
        \derivative{\uu^{n+c_2}}{\tau} \\
        \derivative{\uu^{n+1}}{\tau}
    \end{bmatrix} =
    \begin{bmatrix}
        \FF^{n+c_2} \\
        \FF^{n+1}
    \end{bmatrix}
    \label{eq:DITRNaiveDualTime}
\end{equation}
Applying \eeqref{eq:DITRNaiveDualTime} directly to \eeqref{eq:abstractNewton}
would require the solution of an enlarged sparse linear problem.

For solvers that explicitly implement sparse matrix storage,
such enlargement is trivial at the cost of $k^2$ times
of matrix storage. But for inherently matrix-free flow solvers,
such enlargement could require drastic modification and addition to
the original codebase.
The direct enlargement of the problem is referred to as
the fully coupled iteration method in the current paper.
Due to its complexity of implementation,
the current research does not consider using the
fully coupled iteration method
to solve DITR time stepping schemes.




\subsection{The nested iteration method for solving DITR}

A characteristic of \eeqref{eq:TRSum} is that
$\uu^{n+c_2}$ only exists on the left side, namely the
right side of \eeqref{eq:TRSum} has only $\uu^{n+1}$ as
unknown. Consequently, by substitution, equivalent
to the previous derivations leading to \eeqref{eq:fullLTE},
$\uu^{n+c_2}$ can be eliminated nonlinearly:
\begin{equation}
    \begin{aligned}
        \uu^{n+1} & =                       \\
                  & \uu^{n} + \inc t\biggl[
        b_1 \R^n                            \\
                  & +
            b_2 \R(t^{n+c_2},a^n_0\uu^{n - 1} +
            a^n_1\uu^{n} +
            a^n_2\uu^{n + 1}
            +
            \inc t^n
            \left(
            d^n_1\R^{n} +
            d^n_2\R^{n + 1}
        \right))                            \\
                  & +
            b_3 \R(t^{n+1},\uu^{n+1})
            \biggr]
        \label{eq:nestedDITR}
    \end{aligned}
\end{equation}
which we rewrite into the general residual form required in section \ref{ssec:dualTime}:
\begin{equation}
    \begin{aligned}
        \FF^N(\uu^{n+1}) & = -\frac{\uu^{n+1} - \uu^{n}}{\inc t^n} \\
                         & +
        b_1 \R^n                                                   \\
                         & +
        b_2 \R\left(t^{n+c_2},a^n_0\uu^{n - 1} +
        a^n_1\uu^{n} +
        a^n_2\uu^{n + 1}
        +
        \inc t^n
        \left(
        d^n_1\R^{n} +
        d^n_2\R^{n + 1}
        \right)\right)                                             \\
                         & +
        b_3 \R(t^{n+1},\uu^{n+1}) = 0.
        \label{eq:nestedDITR_F}
    \end{aligned}
\end{equation}
Next, using the linearized implicit backward Euler approximation,
namely equation \eqref{eq:abstractNewton}, DITR methods can be solved:
\begin{equation}
    \left(\frac{\eye}{\inc\tau} -
    \partialderivative{\FF^N}{\uu^{n+1}}  \right)
    \inc\uu^{n+1,m} = \FF^N(\uu^{n+1,m})
    \label{eq:abstractNewtonNested}
\end{equation}
where $\inc\uu^{n+1,m}=\uu^{n+1,m+1}-\uu^{n+1,m}$.

The huge advantage of this iteration method is that only one equation is solved,
with only the next step value as unknown.
This iteration method requires $\R$ to be nested in another $\R$ evaluation and
is referred to as the nested iteration method.

The nested iteration method exploits the
local explicitness of DITR's formulation.
The nested iteration method is similar, and sometimes equivalent with
the iteration methods used in MIRK \cite{cash1975classMIRKOrig,cash1977clasMIRK1,cash1982monoMIRK2}
and NIRK \cite{kulikov2006familyNIRKOrig,kulikov2009adaptive,kulikov2007asymptotic}.
In these literature, the residual function $\FF^N$ is directly
solved using a Newton method:
\begin{equation}
    -\partialderivative{\FF^N}{\uu^{n+1}}
    \inc\uu^{n+1,m} = \FF^N(\uu^{n+1,m})
    \label{eq:abstractNewtonNestedNoTau}
\end{equation}
The Jacobian matrix $\partialderivative{\FF^N}{\uu^{n+1}}$
is a quadratic polynomial of the Jacobian of right-hand-side $\R$.
Kulikov originally proposed to use an approximate factorization \cite{kulikov2006familyNIRKOrig}
of $\partialderivative{\FF^N}{\uu^{n+1}}$ to solve \eeqref{eq:abstractNewtonNestedNoTau},
which allows the direct using of matrix-free solvers.
However, in previous literature, MIRK and NIRK methods are mainly used on small-scale
ODEs rather than as an implicit time marching scheme for compressible flow solvers,
and no pseudo time term was involved.

Our numerical experiments indicate that with small $\inc\tau$
and relatively large $\inc t$,
the nested iteration method \eeqref{eq:abstractNewtonNested}
usually fails to converge even in solving linear problems.
The original iteration methods of NIRK \eeqref{eq:abstractNewtonNestedNoTau}
which does not involve dual time stepping
are found to work on linear problems.
However,  when dual time stepping is abandoned, i.e., $\inc \tau=+\infty$,
for nonlinear flow problems like the isentropic vortex,
when $\inc t$ is relatively large, convergence is almost impossible
to achieve.
Since dealing with large physical CFL number using local
pseudo time stepping is an essential ability of implicit
time stepping methods,
this divergent result denies the practicality of the
nested iteration method.






\subsubsection{Fourier analysis in pseudo time: the nested iteration method}

To explain the failure of the nested iteration method
in solving PDEs uisng the dual time stepping technique, Fourier
analysis based on pseudo time introduced in \ref{sssec:fourier} is considered.

When DITR is solved using the nested iteration method,
residual $\FF^N$ in \eeqref{eq:nestedDITR_F}
cannot be written in the form of  \eeqref{eq:canonicalFF},
as a result of nested $\R$ terms.
In this case, there is an interaction
between the temporal and spatial discretization.
To facilitate the analysis, the spatial discretization
could be specified.
Here we choose the first-order
upwind difference as an example:
\begin{equation}
    \derivative{u_i}{t} = -a \frac{u_i - u_{i-1}}{\inc x}
    \label{eq:fdup1}
\end{equation}
Therefore from \eeqref{eq:nestedDITR_F} and \eeqref{eq:canonicalFF} we have:
\begin{equation}
    \begin{aligned}
        \frac{\dd u_i^{n+1}}{\dd \tau} & =
        -\frac{u_i^{n+1} - u_i^{n}}{\inc t^n}      \\
                                       & +
        b_1 (-a\frac{u_i^{n}-u_{i-1}^{n}}{\inc x}) \\
                                       & +
        b_2 \Bigl[-\frac{a}{\inc x} \nabla_i
        \bigl[
        a^n_0u^{n - 1}_i +
        a^n_1u^{n}_i +
        a^n_2u^{n + 1}_i                           \\
                                       & +
        \inc t^n
        \left(
        d^n_1(-a\frac{u_i^{n}-u_{i-1}^{n}}{\inc x}) +
        d^n_2(-a\frac{u_i^{n+1}-u_{i-1}^{n+1}}{\inc x})
        \right)\bigr]  \Bigr]                      \\
                                       & +
        b_3 (-a\frac{u_i^{n+1}-u_{i-1}^{n+1}}{\inc x})
        \label{eq:nestedDITR_F1}
    \end{aligned}
\end{equation}
where $\nabla_i(u_i) = u_i - u_{u_i-1}$ is the $i$ direction
backward difference operator.
Since only $n+1$ related parts on
the right side is of interest,
the \eeqref{eq:nestedDITR_F1} can be reformulated into:
\begin{equation}
    \begin{aligned}
        \frac{\dd u_i^{n+1}}{\dd \tau}\frac{\inc x}{a} & =
        C_i^n                                              \\
                                                       &
        -  \frac{1}{\CFLt} u_i^{n+1}
        -  b_2a_2^n u_i^{n+1}
        +  b_2a_2^n u_{i-1}^{n+1}
        \\
                                                       &
        + \CFLt b_2 d_2^n u_i^{n+1}
        - \CFLt b_2 d_2^n u_{i-1}^{n+1}
        \\
                                                       &
        - \CFLt b_2 d_2^n u_{i-1}^{n+1}
        + \CFLt b_2 d_2^n u_{i-2}^{n+1}
        \\
                                                       &
        -b_3 u_i^{n+1}
        +b_3 u_{i-1}^{n+1}
        \\
                                                       &
        =C_i^n  +  \CFLt b_2 d_2^n u_{i-2}^{n+1}
        \\
                                                       & +
        (-2\CFLt b_2 d_2^n + b_2a_2^n + b_3)u_{i-1}^{n+1}
        \\
                                                       & +
        (\CFLt b_2 d_2^n - b_3 - b_2a_2^n -  \frac{1}{\CFLt})u_i^{n+1}
        \label{eq:nestedDITR_F2}
    \end{aligned}
\end{equation}
where $C_i^m$ are terms unrelated to $u_i^{n+1}, u_{i-1}^{n+1}\dots$
and remain constant throughout the pseudo time evolution.

Once again simple wave solution from equation \eqref{eq:simpleWaveTestTau}
is applied in \eqref{eq:nestedDITR_F2}, and comparing with
\eqref{eq:kappaSDefTau}, it is found
\begin{equation}
    \begin{aligned}
        \kappa^* = & \imagUnit
        \Biggl[
            \CFLt b_2 d_2^n \exp(-2\imagUnit\kappa)
            +
            (-2\CFLt b_2 d_2^n + b_2a_2^n + b_3)\exp(-\imagUnit\kappa)
        \\ + &
            (\CFLt b_2 d_2^n - b_3 - b_2a_2^n -  \frac{1}{\CFLt})
            \Biggr]
    \end{aligned}
\end{equation}
and therefore the imaginary part:
\begin{equation}
    \begin{aligned}
        \imag(\kappa^*) = &
        \CFLt b_2 d_2^n \cos(2\kappa)
        +
        (-2\CFLt b_2 d_2^n + b_2a_2^n + b_3)\cos(\kappa)
        \\ + &
        (\CFLt b_2 d_2^n - b_3 - b_2a_2^n -  \frac{1}{\CFLt})
        \\
        =                 &
        \CFLt b_2 d_2^n \left[\cos(2\kappa) -2\cos(\kappa) + 1\right]
        + (b_2a_2^n + b_3) (\cos(\kappa) - 1) - \frac{1}{\CFLt}
    \end{aligned}
    \label{eq:imagKsTauDITR}
\end{equation}

In equation \eqref{eq:imagKsTauDITR},
the $\left[\cos(2\kappa) -2\cos(\kappa) + 1\right]$ term
is negative for $\kappa \in (0, \pi/2)$, and for
all current DITR methods, $b_2>0$ and $d_2^n < 0$.
Consequently, there always exists a $\CFLt$ that is large enough
so that $\imag(\kappa^*(\kappa)) \leq 0, \forall \kappa$ is false.
Which means for a large enough physical CFL number, the nested
iteration method is unstable in the sense of continuous $\tau$
evolution.

As an example, DITR U2R2 method with $c_2=1/2$ gives
a specific formulation of $\imag(\kappa^*)$ that only depends on $\kappa$
and $\CFLt$, which is illustrated in figure \ref{fig:HM3_U2R2050_KappaS}.
Figure \ref{fig:HM3_U2R2050_KappaS} indicates for DITR U2R2 $c_2=1/2$,
when using nested iteration method, it is unable to linearly converge
with small $\inc \tau$ and approximately $\CFLt > 8$.
In other words, even with linear advection equation and first order upwind
scheme,
the nested iteration method does not converge when physical CFL reaches 8
and pseudo time step is small.

\begin{figure}[htbp]
    \centering
    \includegraphics[width=0.6\textwidth]{pics/HM3_U2R2050_KappaS.pdf}
    \caption[]{DITR U2R2 $c_2=1/2$ imaginary part of $\kappa^*$ using
        nested solving}
    \label{fig:HM3_U2R2050_KappaS}
\end{figure}

After further analysis,
besides the first-order upwind spatial discretization,
for any linear spatial discretization with modified wave number $\kappa'$,
the pseudo time wave number $\kappa^*$ is
\begin{equation}
    \begin{aligned}
        \kappa^* =
        -i\CFLt b_2 d_2^n {\kappa'}^2
        + (b_2a_2^n + b_3) \kappa' - \frac{i}{\CFLt}
    \end{aligned}
    \label{eq:imagKsTauDITR_General}
\end{equation}
and for $\kappa'$ with finite non-zero real and imaginary parts,
it is always easy to find a large enough $\CFLt$ that makes
$\imag(\kappa^*) > 0$.


According to numerical tests with linear advection problem,
when $\inc \tau \rightarrow \infty$, the nested solving
approach is possibly convergent. However, for strongly nonlinear
flow problems, a small $\inc \tau$ is almost a must.
The current research has tested nested iteration method
with N-S equations discretized using high-order FV method,
and the nested iteration method is indeed unable to successfully
converge in many scenarios.
The failure in both theoretical pseudo time stability and
real life problems makes
the nested iteration method inapplicable.

\subsection{The stage-decoupled iteration method of DITR}

The failure of nested iteration method in dual time stepping
roots in the nesting of right-hand-side $\R$ terms
(i.e. spatial  discretization operators), which
causes an $O(\CFLt)$ term in the modified wave number,
making the wave number unbounded when $\CFLt\rightarrow + \infty$.
Therefore, the non-nested form of dual time stepping in
the fully coupled iteration method \eeqref{eq:DITRFFDef} and \eeqref{eq:DITRNaiveDualTime}
seems more favorable.

In order to circumvent the complexity encountered in solving the enlarged
system via the fully coupled iteration method, the present paper
proposes a simplified iteration method to solve the enlarged dual time marching
\eeqref{eq:DITRNaiveDualTime}:
\begin{equation}
    \begin{aligned}
        \left(\frac{\eye}{\inc\tau} -
        \partialderivative{\FF^{n+c_2}(\uu^{n+c_2}, \uu^{n+1})}{\uu^{n+c_2}}  \right)
        \inc\uu^{n+c_2,m} & = \FF^{n+c_2}(\uu^{n+c_2,m}, \uu^{n+1,m}) \\
        \left(\frac{\eye}{\inc\tau} -
        \partialderivative{\FF^{n+1}(\uu^{n+c_2}, \uu^{n+1})}{\uu^{n+1}}  \right)
        \inc\uu^{n+1,m}   & = \FF^{n+1}(\uu^{n+c_2,m+1}, \uu^{n+1,m}) \\
    \end{aligned}
    \label{eq:DITRSDS}
\end{equation}
where $\inc \uu^{n+c_2,m} = \uu^{n+c_2,m+1} - \uu^{n+c_2,m}$ and
$\inc \uu^{n+1,m} = \uu^{n+1,m+1} - \uu^{n+1,m}$.
In the first updating formula in \eeqref{eq:DITRSDS},
$\uu^{n+1}$ is considered static and the linearizing is only
performed on $\uu^{n+c_2}$. The second updating formula
then makes $\uu^{n+c_2}$ static, and performs linearizing on
$\uu^{n+1}$ only. Note that the right side $\FF$ functions
and Jacobian terms $\partialderivative{\FF}{\uu}$ always
uses the latest versions of $\uu^{n+c_2}$ and $\uu^{n+1}$.
Both $\uu^{n+c_2}$ and $\uu^{n+1}$ are updated immediately
after each increment values are obtained.
The linear systems are in the same form
of those in BDF and ESDIRK methods, and they
can be solved using matrix-free linear solvers like
LU-SGS solvers.

Compared to the updating procedure in fully coupled iteration method,
where both increment values are solved in a coupled system,
the current iteration method decouples the inter-stage linear connections.
So, this iteration method is referred to as the stage-decoupled iteration method.

The convergence of the stage-decoupled iteration method relies on
the preconditioning matrix $\Pm$.
As $\Pm$ is required to be non-singular, and
$\GG$ terms in \eeqref{eq:DITRFFDef} are already
well-formed, the current research limits the form
of $\Pm$ to be:


where $\beta$ is a real parameter.
Note that restricting the form of $\Pm$ does not affect the
generality of the following discussions,
where the analysis methodologies
are applicable to all forms of $\Pm$.

With the specified preconditioning \eeqref{eq:specialP},
the residual functions take the form:
\begin{equation}
    \begin{aligned}
        \FF^{n+c_2} & =
        \frac{a^n_0\uu^{n - 1} +
            (a^n_1+\beta)\uu^{n} +
            (a^n_2-\beta)\uu^{n + 1} - \uu^{n+c_2}}{\inc t^n}
        \\ & +
        (d^n_1 + \beta b_1)\R^{n} +
        \beta b_2 \R^{n+c_2} +
        (d^n_2 + \beta b_3)\R^{n + 1},                             \\
        \FF^{n+1}   & =    \frac{\uu^{n} - \uu^{n+1}}{\inc t^n}  +
        b_1 \R^n +
        b_2 \R^{n+c_2} +
        b_3 \R^{n+1}.
    \end{aligned}
\end{equation}

The current paper uses matrix-free approximate LU-SGS
\cite{luo1998fast,luo2001accurate} to solve the linear problems \eeqref{eq:DITRSDS}.

\subsubsection{Fourier analysis for pseudo time: the stage-decoupled iteration method}
\label{sssec:SDSFourier}

When $\inc \tau \rightarrow 0$, the stage-decoupled pseudo time marching
\eeqref{eq:DITRSDS} is still consistent with the continuous
pseudo time marching \eeqref{eq:DITRNaiveDualTime}, therefore
general Fourier analysis on \eeqref{eq:DITRNaiveDualTime} is performed.
To simplify the analysis, we can rewrite \eeqref{eq:DITRFFDef}
as a linear form:
\begin{equation}
    \begin{bmatrix}
        \FF^{n+c_2} \\
        \FF^{n+1}
    \end{bmatrix}
    =
    \begin{bmatrix}
        \C_1 + \frac{1}{\inc t^n}(A_{11}\uu^{n+c_2} + A_{12}\uu^{n+1})
        + B_{11}\R^{n+c_2} + B_{12}\R^{n+1} \\
        \C_2 + \frac{1}{\inc t^n}(A_{21}\uu^{n+c_2} + A_{22}\uu^{n+1})
        + B_{21}\R^{n+c_2} + B_{22}\R^{n+1}
    \end{bmatrix}
    \label{eq:linearDITRFF}
\end{equation}
where $A_{ij},B_{ij}$ are constant coefficients
decided by the DITR method, and $\C_i$ are
vectors that are constant in the solving process.

Using simple wave modes:
\begin{equation}
    \begin{aligned}
        u^{n+c_2}(\tau,x) & = A^{n+c_2}(\tau)\exp(\imagUnit \frac{\kappa}{\inc x} x) \\
        u^{n+1}(\tau,x)   & = A^{n+1}(\tau)\exp(\imagUnit \frac{\kappa}{\inc x} x)
    \end{aligned}
\end{equation}

By applying the Fourier analysis mentioned before,
with some spatial  discretization with modified having number
being $\kappa'$, the evolution of wave strength is:
\begin{equation}
    \derivative{ }{\tau}
    \begin{bmatrix}
        A^{n+c_2}(\tau) \\
        A^{n+1}(\tau)
    \end{bmatrix}
    =
    -\imagUnit \frac{a}{\inc x}
    \left(\frac{\imagUnit}{\CFLt}\begin{bmatrix}
        A_{11} & A_{12} \\
        A_{21} & A_{22} \\
    \end{bmatrix}
    +\kappa'
    \begin{bmatrix}
        B_{11} & B_{12} \\
        B_{21} & B_{22} \\
    \end{bmatrix}
    \right)
    \begin{bmatrix}
        A^{n+c_2}(\tau) \\
        A^{n+1}(\tau)
    \end{bmatrix}
    +
    \begin{bmatrix}
        C_1 \\C_2
    \end{bmatrix}
\end{equation}
where $C_1,C_2$ are constants during the pseudo time marching.
Therefore, the modified pseudo time marching wave number is
the eigenvalues $\kappa^*_1,\kappa^*_2$ of matrix $\mathbf{K}^*$ and
\begin{equation}
    \mathbf{K}^* = \frac{\imagUnit}{\CFLt}\begin{bmatrix}
        A_{11} & A_{12} \\
        A_{21} & A_{22} \\
    \end{bmatrix}
    +\kappa'
    \begin{bmatrix}
        B_{11} & B_{12} \\
        B_{21} & B_{22} \\
    \end{bmatrix}
\end{equation}

While preconditioner $\Pm$ is in the form of \eeqref{eq:specialP},
for the DITR methods discussed in the current paper,
the coefficient matrices are
\begin{equation}
    \begin{bmatrix}
        A_{11} & A_{12} \\
        A_{21} & A_{22} \\
    \end{bmatrix} = \begin{bmatrix}
        -1 & a_2^n - \beta \\
        0  & -1            \\
    \end{bmatrix},\ \
    \begin{bmatrix}
        B_{11} & B_{12} \\
        B_{21} & B_{22} \\
    \end{bmatrix} = \begin{bmatrix}
        b_2\beta & b_3\beta + d_2^n \\
        b_2      & b_3              \\
    \end{bmatrix}
    \label{eq:ABBetaForm}
\end{equation}
Therefore, $\mathbf{K}^*$ is determined by the DITR coefficients,
$\beta$, $\kappa'$ and physical CFL number.
Since $\CFLt > 0$, $\real(\frac{\CFLt\kappa^*_i}{\imagUnit}) \leq 0, i=1,2$
is equivalent with $\imag(\kappa^*_i) \leq 0, i=1,2$,
we will investigate the eigenvalues of
$\frac{\CFLt}{\imagUnit}\mathbf{K}^*$, which is
determined by DITR coefficients, $\beta$ and the
combined non-dimensional eigenvalue of spatial  discretization $\mu'$
where
\begin{equation}
    \mu' = \frac{\CFLt \kappa'}{\imagUnit}
\end{equation}
and
\begin{equation}
    \frac{\CFLt}{\imagUnit}\mathbf{K}^* = \begin{bmatrix}
        A_{11} & A_{12} \\
        A_{21} & A_{22} \\
    \end{bmatrix} + \mu \begin{bmatrix}
        B_{11} & B_{12} \\
        B_{21} & B_{22} \\
    \end{bmatrix}
\end{equation}

Therefore, to ensure the stability of continuous pseudo time
marching of DITR's stage-decoupled iteration method, it is
required
\begin{equation}iteration method
    \real(\mu^*_i) \leq 0,\  i = 1,2,\  \forall \real(\mu') \leq 0
\end{equation}
where
\begin{equation}
    \mu^*_i = \frac{\CFLt\kappa^*_i}{\imagUnit},\ i=1,2
\end{equation}

As an example, $\max(\real(\mu^*_1),\real(\mu^*_2))$ is plotted for
DITR U2R2 $c_2=1/2$ in Figure \ref{fig:MuU2R2} with varying $\beta$ choices.
When $\beta = 0$ or the preconditioner is identity, the pseudo time marching is
unstable. Stability seems to be achieved when $\beta = 1$, because
the left half plane seems to have $\max(\real(\mu^*_1),\real(\mu^*_2)) < 0$.
\begin{figure}[htbp]
    \centering
    \begin{subfigure}{0.33\textwidth}
        \includegraphics[width=\textwidth]{pics/HM3_Method1_C0.5_Theta1_Beta0_TauMu.pdf}
        \caption[]{$\beta = 0$}
    \end{subfigure}\hfill
    \begin{subfigure}{0.33\textwidth}
        \includegraphics[width=\textwidth]{pics/HM3_Method1_C0.5_Theta1_Beta0.5_TauMu.pdf}
        \caption[]{$\beta = 0.5$}
    \end{subfigure}\hfill
    \begin{subfigure}{0.33\textwidth}
        \includegraphics[width=\textwidth]{pics/HM3_Method1_C0.5_Theta1_Beta1_TauMu.pdf}
        \caption[]{$\beta = 1$}
    \end{subfigure}
    \caption{DITR U2R2 $c_2=1/2$, $\max(\real(\mu^*_1),\real(\mu^*_2))$ distribution, black line is 0}
    \label{fig:MuU2R2}
\end{figure}

To determine the choices of the preconditioning parameter  $\beta$
with Fourier analysis, the
present paper has performed numerical scanning
on $\beta$, which will be discussed later.

\subsubsection{Gauss-Seidel stability of stage-decoupled iteration method}
\label{sssec:GSStability}

The previous section has discussed the ideal
situation of a continuous $\tau$ evolution, where
$\inc \tau \rightarrow 0$. When $\inc \tau$ is finite
or $\inc \tau \rightarrow +\infty$,
the effect of stage-decoupling as described in \eeqref{eq:DITRSDS}
is non-trivial.
If $\R$ is linear, and $\inc \tau \rightarrow +\infty$,
the iterating process of \eeqref{eq:DITRSDS} is
actually a 2-by-2 block Gauss-Seidel iteration that obtains
the linear increment if convergent.
Therefore, the current section is about to discuss the
near Guass-Seidel behavior of the stage-decoupled iteration method.

Assuming $\R$ to be linear and diagonalizable,
a linear scalar ODE
\begin{equation}
    \derivative{y}{t} = \lambda y
\end{equation}
with $\real(\lambda)\leq 0$
is tested in the stage-decoupled iteration method when
$\inc \tau \rightarrow +\infty$.
\eeqref{eq:DITRSDS} therefore becomes
\begin{equation}
    \begin{aligned}
        \inc y^{n+c_2, m} & = -\frac{1}{ E_{11} } (c_1 + E_{11}  y^{n+c_2, m}+ E_{12}  y^{n+1, m})     \\
        \inc y^{n+1, m}   & = -\frac{1}{ E_{22} } (c_2 + E_{21}  y^{n+c_2, m + 1}+ E_{22}  y^{n+1, m})
    \end{aligned}
    \label{eq:scalarGS}
\end{equation}
where $E_{ij}=A_{ij}+zB_{ij},\ i,j=1,2$, and $A_{ij}, B_{ij}$ are
linear coefficients described in the linear form \eeqref{eq:linearDITRFF}
and can be calculated with \eeqref{eq:ABBetaForm} using the $\beta$ form
preconditioning. $z=\lambda\inc t^n$ is the non-dimensional eigenvalue.
$c_1,c_2$ are constant values not dependent on $y^{n+1}, y^{n+c_2}$.

Considering $y^{n+c_2,m} + \inc y^{n+c_2,m} = y^{n+c_2,m+1}$ and
$y^{n+1,m} + \inc y^{n+1,m} = y^{n+1,m+1}$,
\eeqref{eq:scalarGS} has matrix form:
\begin{equation}
    \begin{bmatrix}
        y^{n+c_2, m + 1} \\
        y^{n+1, m + 1}
    \end{bmatrix}
    =
    -\begin{bmatrix}
        E_{11} & 0      \\
        E_{21} & E_{22}
    \end{bmatrix}^{-1} \begin{bmatrix}
        c_1 \\c_2
    \end{bmatrix}
    -
    \begin{bmatrix}
        E_{11} & 0      \\
        E_{21} & E_{22}
    \end{bmatrix}^{-1}
    \begin{bmatrix}
        0 & E_{12} \\
        0 & 0
    \end{bmatrix}
    \begin{bmatrix}
        y^{n+c_2, m} \\
        y^{n+1, m}
    \end{bmatrix}
    \label{eq:scalarGSMat}
\end{equation}
\eeqref{eq:scalarGSMat} clearly demonstrates that the stage-decoupled
iteration method reduces to a $2\times2$ linear Gauss-Seidel iteration.
The recursion transformation
\begin{equation}
    \B_{GS} = -\begin{bmatrix}
        E_{11} & 0      \\
        E_{21} & E_{22}
    \end{bmatrix}^{-1}
    \begin{bmatrix}
        0 & E_{12} \\
        0 & 0
    \end{bmatrix}
\end{equation}
has only one non-zero eigenvalue
\begin{equation}
    G_{GS} = \frac{E_{12}\,E_{21}}{E_{11}\,E_{22}}
\end{equation}
To make the Gauss-Seidel recursion \eeqref{eq:scalarGSMat}
convergent, $|G_{GS}|\leq 1, \forall \real(z) \leq 0$ is required.

As an example, distribution of $|G_{GS}|$ for DITR U2R2 $c_2=1/2$
is illustrated in Figure \ref{fig:GGSU2R2}.
The figure shows when $\beta=0$, the stability region is finite.
When $\beta>0.5$, the figure shows that the stability region includes
the left half plane.
\begin{figure}[htbp]
    \centering
    \begin{subfigure}{0.33\textwidth}
        \includegraphics[width=\textwidth]{pics/HM3_Method1_C0.5_Theta1_Beta0_GGS.pdf}
        \caption[]{$\beta = 0$}
    \end{subfigure}\hfill
    \begin{subfigure}{0.33\textwidth}
        \includegraphics[width=\textwidth]{pics/HM3_Method1_C0.5_Theta1_Beta0.5_GGS.pdf}
        \caption[]{$\beta = 0.5$}
    \end{subfigure}\hfill
    \begin{subfigure}{0.33\textwidth}
        \includegraphics[width=\textwidth]{pics/HM3_Method1_C0.5_Theta1_Beta1_GGS.pdf}
        \caption[]{$\beta = 1$}
    \end{subfigure}
    \caption{DITR U2R2 $c_2=1/2$, $|G_{GS}|$ distribution, black line is 1}
    \label{fig:GGSU2R2}
\end{figure}

Using the Gauss-Seidel stability of the stage-decoupled iteration method,
the choice of $\beta$ can be again narrowed down.
Numerical scanning is also conducted to find optimized $\beta$ value for
DITR methods, which will be discussed in the next section.

\subsubsection{Numerical Scanning for the Preconditioning Parameter}
\label{sssec:numScan}

The preconditioning parameter, $\beta$, as discussed in \ref{sssec:SDSFourier}
and \ref{sssec:GSStability}, affects convergence properties of the
stage-decoupled solving strategy.
When $\inc \tau\rightarrow0$, as discussed in \ref{sssec:SDSFourier},
$\max(\real(\mu^*_1),\real(\mu^*_2))$ represents the linear growth
rate of the $\tau$ evolution. When the DITR method parameters are settled,
$\mu^*_1=\mu^*_1(\beta, \mu')$, $\mu^*_2=\mu^*_2(\beta, \mu')$ and $G_{GS}=G_{GS}(\beta, z)$.
When $\inc \tau\rightarrow+\infty$, $|G_{GS}|$ from \ref{sssec:GSStability}
represents the magnifying factor in the Gauss-Seidel updating.
Similar to $A$-stability of ODE integrators,
the relations:
\begin{equation}
    \max(\real(\mu^*_1),\real(\mu^*_2)) \leq 0,\ \forall \real(\mu') \leq 0
\end{equation}
and
\begin{equation}
    |G_{GS}| \leq 1,\ \forall \real(z) \leq 0
\end{equation}
are referred to as the $A(\tau,0)-$stability and
$A(\tau,\infty)-$stability
of the stage-decoupled iteration method.
The current paper hopes to utilize both extremes of $\inc\tau$ to
narrow down the choice of an optimized $\beta$.


First, $\beta$ is restrained to be no greater than $O(1)$ to make
the preconditioning non-singular.
Meanwhile, $\beta\leq 0$ leads to a negative coefficient of $\R^{n+c_2}$
in $\FF^{n+c_2}$ according to \eeqref{eq:DITRFFDef}, which almost
guarantees the loss of stability. Therefore, empirically, $\beta\in[0,2]$
is to be the numerical scanning interval.

Next, given a certain $\beta$ value and a certain DITR method, the
stability indicators $\max(\real(\mu^*_1),\real(\mu^*_2))$ and $|G_{GS}|$
can be calculated as functions of $\mu'$ and $z$ respectively.
To check on the $A(\tau,0)-$ and $A(\tau,\infty)-$stability of this
certain $\beta$ choice, $\mu'$ and $z$ are numerically sampled
in the left half complex plane. The samples include $101\times101$
uniform points in $[-10,0]\times[0,10]$. Furthermore, radially distributed
points with 101 argument samples in $[\pi/2,\pi]$ and 10 exponentially
distributed radius samples in $[10^1, 10^{10}]$ are included.
Note that the sample points are all above the real axis, because the
distributions of $\max(\real(\mu^*_1),\real(\mu^*_2))$ and $|G_{GS}|$
are symmetric about the real axis with the DITR coefficients being real.
The numeric sample points form the sample
sets $\{\mu'_i\}$ and $\{z_i\}$. Then, the numeric maximums are
calculated:
\begin{equation}
    \begin{aligned}
        \mu^*_{M}(\beta) & = \max_{i}\{\max(\real(\mu^*_1(\beta,\mu'_i)),\real(\mu^*_2(\beta,\mu'_i)))\} \\
        G_{GS,M}(\beta)  & = \max_{i}\{|G_{GS}(\beta,z_i)|\}
    \end{aligned}
\end{equation}
if for some $\beta=\beta_0$, $\mu^*_{M}(\beta_0) \leq 0,G_{GS,M}(\beta_0) \leq 1$,
it is known for  $\beta=\beta_0$, $A(\tau,0)-$stability and
$A(\tau,\infty)-$stability are achieved.


\begin{figure}[htbp]
    \centering
    \begin{subfigure}{0.5\textwidth}
        \includegraphics[width=\textwidth]{pics/HM3_Methods_SearchMu.pdf}
        % \caption[]{$\beta = 0$}
    \end{subfigure}\hfill
    \begin{subfigure}{0.5\textwidth}
        \includegraphics[width=\textwidth]{pics/HM3_Methods_SearchGGS.pdf}
        % \caption[]{$\beta = 0.5$}
    \end{subfigure}
    \caption{Numerical scanning results of $\beta$}
    \label{fig:MuGGSSearch}
\end{figure}
Figure \ref{fig:MuGGSSearch} illustrates the numerical scanning
results of maximal values of
$\max(\real(\mu^*_1),\real(\mu^*_2))$ and $|G_{GS}|$ in the left
half plane. Note that U3R1 methods uses $c_2=0.5$ in all cases, and $Theta$ is
the ratio of time step sizes.
In the sense of satisfying the stability conditions $\mu^*_{M} \leq 0,G_{GS,M} \leq 1$,
$\mu^*_{M}$ offers a more strict restriction.
The approximate results are listed in Table \ref{tab:resrictionBetaSearch}.

\begin{table}[htbp]
    \centering
    \begin{tabular}{|c|c|c|}
        \hline
        Method             & Restriction from $\mu^*_{M}$ & Restriction from $G_{GS,M}$ \\
        \hline
        U2R2 $c_2=0.5$     & $\beta > 0.63$               & $\beta > 0.38$              \\
        \hline
        U2R2 $c_2=0.55$    & $\beta > 0.72$               & $\beta > 0.53$              \\
        \hline
        U2R1 $c_2=0.25$    & $\beta > 0.63$               & $\beta > 0.34$              \\
        \hline
        U2R1 $c_2=0.5$     & $\beta > 1$                  & $\beta > 0.75$              \\
        \hline
        U3R1 $\Theta=0.25$ & $\beta > 0.71$               & $\beta > 0.45$              \\
        \hline
        U3R1 $\Theta=1$    & $\beta > 0.83$               & $\beta > 0.57$              \\
        \hline
        U3R1 $\Theta=4$    & $\beta > 0.94$               & $\beta > 0.68$              \\
        \hline
    \end{tabular}
    \caption{Approximate stability restrictions on $\beta$ obtained in numerical scanning}
    \label{tab:resrictionBetaSearch}
\end{table}

In practice, the current research observed that
when $\beta$ is significantly larger than the approximate
restriction boundaries in Table \ref{tab:resrictionBetaSearch},
convergence is faster and more reliable.
The reason could be that with larger $\beta$, $\mu^*_M$  and $G_{GS,M}$
are allowed to be significantly smaller than $0$ and $1$ respectively,
making $A(\tau,0)-$stability and
$A(\tau,\infty)-$stability stronger.
Meanwhile, it is observed that the shape of the stability region
$|G_{GS}|\leq1$ is greatly affected by the choice of $\beta$,
which is indicated in Figure \ref{fig:HM3_U2R205_GGSRegions}.
In Figure \ref{fig:HM3_U2R205_GGSRegions}, with the
increase of $\beta$, the unstable region is more
compacted around the real axis, making the iteration with $\inc\tau\rightarrow\infty$
converge more reliably.
\begin{figure}[htbp]
    \centering
    \includegraphics[width=0.6\textwidth]{pics/HM3_U2R205_GGSRegions.pdf}
    \caption[]{Boundaries of $|G_{GS}|\leq1$ stability regions of U2R2 $c_2=0.5$ with different $\beta$}
    \label{fig:HM3_U2R205_GGSRegions}
\end{figure}

However, when $\beta>1.5$, in all cases $\mu^*_{M}$ and $G_{GS,M}$ increase
with $\beta$, which means some modes in the linear error will
decay slower. In practice, the current research indeed finds a larger
$\beta$ causes slow convergence.

\begin{figure}[htbp]
    \centering
    \includegraphics[width=0.6\textwidth]{pics/HM3_U2R1_GGSRegions.pdf}
    \caption[]{Boundaries of $|G_{GS}|\leq1$ stability regions of U2R1 with different $\beta$ and $c_2$}
    \label{fig:HM3_U2R1_GGSRegions}
\end{figure}
Another issue that should be addressed here is the
selection of $c_2$ for U2R1 methods. $c_2$ does not
affect order of accuracy and ODE stability of U2R1, but
has influence on the convergence behavior, as is shown
in Figure \ref{fig:MuGGSSearch} and Table \ref{tab:resrictionBetaSearch}.
From Figure \ref{fig:MuGGSSearch} and Table \ref{tab:resrictionBetaSearch},
with $c_2=0.25$, U2R1 has less restriction on $\beta$ than with $c_2=0.5$.
Meanwhile, Figure \ref{fig:HM3_U2R1_GGSRegions} illustrates the
impact of $c_2$ on $|G_{GS}|\leq1$ stability regions.
With $c_2=0.5$, the unstable region expands far from the real axis,
while when $c_2=0.25$, the unstable region is
more concentrated around the real axis similar with U2R2 $c_2=0.5$
shown in Figure \ref{fig:HM3_U2R205_GGSRegions}.
Consequently, the current research chooses $c_2=0.25$ over $c_2=0.5$
for U2R1 due to its better convergence behavior in
the stage-decoupled iteration method.



Combining the discussions above,
for U2R2 $c=0.5$, we empirically choose $\beta = 1$.
For U2R2 $c=0.55$, we empirically choose $\beta = 1.333$.
For U2R2 $c=0.25$, we empirically choose $\beta = 1$.
For U3R1, we empirically use $\beta = 1.333$ for all $\Theta$.
These $\beta$ values are significantly larger than the
stability restrictions and guarantees
$A(\tau,0)-$stability and
$A(\tau,\infty)-$stability.
Also, these  $\beta$ values are large enough to make the stable regions of $|G_{GS}|\leq1$
crudely larger, which produces better reliability of convergence.
Meanwhile, $\beta$ values are not so large that $\mu^*_{M}$ and $G_{GS,M}$
are too close to 1, maintaining sufficient convergence speed.



\section{Numerical tests}
\label{sec :Results}

During numerical tests,
BDF2 and ESDIRK4 methods taken from
\cite{bijl2002implicitBDFvESDIRK,kennedy2003additiveARK}
are chosen to be
baseline time marching methods.
For DITR methods, instances of
U2R2 $c_2 = 0.5$, U2R2 $c_2 = 0.55$, U2R1 and U3R1 are tested.

The isentropic vortex, two dimensional vortex shedding
and  double mach reflection problems use
$P^3$ variational reconstruction finite volume method declared in
section \ref{sec:CFV} as spatial  discretization.
Iterative solution of the implicit reconstruction
is conducted before each right-hand-side evaluation,
which consists of 1 block-Jacobi iteration by default.
Pseudo time continued Newton iteration on mean values are
solved using 5 times of block-Jacobi iteration by default,
which is found both stable and efficient enough for VFV solving
transient problems.
It should be noted that, in other words,
both the reconstruction system and
the time stepping implicit system use linear solvers that
runs for a fixed number of steps without monitoring convergence.
Only the magnitude of the nonlinear time stepping residual
is monitored to ensure convergence.
This linear solving method simplifies the control flow of
the program,
while providing a simple yet reliable way of
comparing total calculation consumption between ODE solvers,
which will be explained in section \ref{ssec:resultIV}.



\subsection{Isentropic vortex}
\label{ssec:resultIV}

The isentropic vortex problem is a classic
accuracy testing problem for Euler equations.
The settings can be found in \cite{hu1999weighted_WENO}.
The free-stream flow is $(\rho,u,v,p)=(1,1,1,1)$,
and a perturbation at initial time:
\begin{equation}
    \left\{
    \begin{array}[2]{ll}
        (\delta u, \delta v) & = \frac{\epsilon}{2\pi}\exp(\frac{1-[(x-x_c)^2+(y-y_c)^2]}{2})(-y+y_c,x-x_c) \\
        \delta T             & = - \frac{(\gamma-1)\epsilon^2}{8\gamma\pi^2}\exp(1-[(x-x_c)^2+(y-y_c)^2])   \\
        \delta S             & = 0                                                                          \\
    \end{array}
    \right.
\end{equation}
with ideal gas setting of $T = p/\rho, S= p/\rho^\gamma, \gamma =1.4$.
Initial vortex center is chosen $(x_c,y_c)=(5,5)$,
and vortex strength is $\epsilon = 5$.
The analytic solution to the isentropic vortex problem is a
translation of initial field with speed $(1,1)$.
The computational domain is $[0,10]\times[0,10]$,
using periodic boundary conditions.

First, the implicit ODE integrators are tested with aggressively large
time steps.
The mesh is $40\times40$ square grid,
and solution is calculated until $t=10$ with $\inc t=1$.
The CFL number based on $\inc t, \inc x$ is roughly $11$,
making the propagation of the vortex hard to
simulate.

\begin{figure}[htbp]
    \centering
    \begin{subfigure}{0.33\textwidth}
        \includegraphics[width=\textwidth]{pics/IV40_10steps_HM3LBT.png}
        \caption[]{DITR U2R2 $c_2=0.5$}
        \label{sfig:IV10Step_HM3LBT}
    \end{subfigure}\hfill
    \begin{subfigure}{0.33\textwidth}
        \includegraphics[width=\textwidth]{pics/IV40_10steps_HM3.png}
        \caption[]{DITR U2R2 $c_2=0.55$}
        \label{sfig:IV10Step_HM3}
    \end{subfigure}\hfill
    \begin{subfigure}{0.33\textwidth}
        \includegraphics[width=\textwidth]{pics/IV40_10steps_HM3U2R1.png}
        \caption[]{DITR U2R1}
        \label{sfig:IV40_10steps_HM3U2R1}
    \end{subfigure}\\
    \begin{subfigure}{0.33\textwidth}
        \includegraphics[width=\textwidth]{pics/IV40_10steps_HM3U3R1.png}
        \caption[]{DITR U3R1}
        \label{sfig:IV40_10steps_HM3U3R1}
    \end{subfigure}\hfill
    \begin{subfigure}{0.33\textwidth}
        \includegraphics[width=\textwidth]{pics/IV40_10steps_ESDIRK4.png}
        \caption[]{ESDIRK4}
        \label{sfig:IV10Step_ESDIRK4}
    \end{subfigure}\hfill
    \begin{subfigure}{0.33\textwidth}
        \includegraphics[width=\textwidth]{pics/IV40_10steps_BDF2.png}
        \caption[]{BDF2}
        \label{sfig:IV10Step_BDF2}
    \end{subfigure}
    \caption{Density of isentropic vortex problem, with aggressively large time step $\inc t = 1$ at $t=10$}
    \label{fig:IV10Step}
\end{figure}

Results of large time step testing are shown in figure \ref{fig:IV10Step}.
Clearly, from figure \ref{sfig:IV10Step_BDF2}, BDF2 almost completely
smears the initial vortex with only 10 steps for one period.
The higher order methods somehow preserve the characteristics of
a vortex. DITR U2R2 $c_2=0.5$ produces significant numerical oscillation along
the propagation direction as shown in figure \ref{sfig:IV10Step_HM3LBT},
while the DITR U2R2 $c_2=0.55$ inhibits them better in figure \ref{sfig:IV10Step_HM3}.
The peak value in the vortex center produced by DITR U2R2 $c_2=0.5$ is comparable with
ESDIRK4, while DITR U2R2 $c_2=0.55$ gives a flatter density peak.
Both U2R1 and U3R1 methods are $L$-stable like ESDIRK4, and they
are indeed better at suppressing non-physical oscillations.
Note that it seems ESDIRK4 outperforms all the DITR methods in
this case, but DITR only needs 2 internal stages to be solved while
ESDIRK4 needs 5.

Next, precision and efficiency of different ODE methods are
qualitatively evaluated with isentropic vortex solved on a $160\times160$
grid until $t=2$.
The fine mesh makes spatial
discretization error negligible compared with
time marching error.
The density error is defined as an $L1$ norm in the form of
\begin{equation}
    \epsilon_\rho = \frac{\int{|\rho-\rho_a| \dd x\dd y}}{
        100
    }
\end{equation}
with  $\rho$ the numeric result of density
and $\rho_a$ the analytic result.



\begin{figure}[htbp]
    \centering
    \begin{subfigure}{0.5\textwidth}
        \includegraphics[width=\textwidth]{pics/HM3_IV160_fig_1.pdf}
        \caption[]{Density error vs. time step size }
        \label{sfig:IVTests_Conv}
    \end{subfigure}\hfill
    \begin{subfigure}{0.5\textwidth}
        \includegraphics[width=\textwidth]{pics/HM3_IV160_fig_2.pdf}
        \caption[]{Density error vs. total CPU time consumption}
        \label{sfig:IVTests_Eff}
    \end{subfigure}
    \caption[]{Convergence and efficiency test with isentropic vortex problem}
    \label{fig:IVTests}
\end{figure}

Results of convergence and efficiency study with isentropic vortex is
shown in figure \ref{fig:IVTests}.

Figure \ref{sfig:IVTests_Conv} shows that with the same time step size,
ESDIRK4 has the best accuracy, while DITR U2R2 $c_2=0.5$ is close to
ESDIRK4 when time step is refined. DITR U2R2 $c_2=0.55$ is less accurate
than DITR U2R2 $c_2=0.5$, but it performs almost as well as
DITR U2R2 $c_2=0.5$ when time step is large.
DITR U3R1 and U2R1 are less accurate than U2R2 methods with
the same time step.

As of order of error, in Figure \ref{sfig:IVTests_Conv},
error of ESDIRK4 is barely able to reach 4th order of convergence,
while DITR U2R2 $c_2=0.5$ and U3R1 methods are also able to reach
4th order convergence with smaller time steps. U2R1 and U2R2 $c_2=0.55$
methods are closer to the 3rd order slope, which conforms with
their theoretical order.

Figure \ref{sfig:IVTests_Eff}
implies that when consuming the same computational resource,
DITR U2R2 methods have the best accuracy and efficiency.
All the high-order
time marching methods have better efficiency than BDF2,
while DITR U2R2 methods have better efficiency than ESDIRK4.
The symmetric DITR U2R2 $c_2=0.5$ has better efficiency
compared with more stable DITR U2R2 $c_2=0.55$.


\subsection{Two dimensional vortex shedding}

Vortex shedding from a circular cylinder and forming a vortex street
is a classic test problem for transient fluid simulation. Due to the refined
mesh near solid wall, such cases usually prefer implicit time marching
over explicit ones whose time steps are bounded by CFL condition.
The current paper studies the 2D laminar case, where
Reynolds number
$Re_d=\rho_\infty u_\infty d / \mu_\infty $ is  $1200$,
with Mach number being $Ma=0.1$.
Parameters are normalized so that freestream speed, density
and diameter of the cylinder are unit values.
Small Mach number makes the flow more incompressible, and
the speed of sound makes time steps in explicit time marching
restricted.
Implicit time marching schemes can automatically omit the restrictions
of the speed of sound, thus being potentially more favorable.

\begin{figure}[htbp]
    \centering
    \begin{subfigure}{0.5\textwidth}
        \includegraphics[width=\textwidth]{pics/CylinderB1_Re1200_Mesh.png}
        \caption[]{Mesh}
        \label{sfig:CylinderRe1200Demo_Mesh}
    \end{subfigure}\hfill
    \begin{subfigure}{0.5\textwidth}
        \includegraphics[width=\textwidth]{pics/CylinderB1_Re1200.png}
        \caption[]{Vorticity distribution}
        \label{sfig:CylinderRe1200Demo_Vort}
    \end{subfigure}
    \caption[]{Mesh and a instance of z-vorticity distribution
        in $Re=1200$ vortex shedding problem}
    \label{fig:CylinderRe1200Demo}
\end{figure}

Figure \ref{sfig:CylinderRe1200Demo_Mesh} demonstrates
the unstructured grid used in the $Re=1200$  vortex
street calculation, and Figure \ref{sfig:CylinderRe1200Demo_Vort}
demonstrates z-vorticity distribution in the 2-D vortex street
after it is fully developed.

In order to quantitatively compare different
time marching schemes,
the time marching error is compared.
Using the same mesh as in Figure \ref{sfig:CylinderRe1200Demo_Mesh}
and the same compact FV spatial  discretization,
a numeric reference solution is calculated with ESDIRK4 using
very fine time step $\inc t = 0.00125$.
In the reference solution, restart information at $t=200$
is stored, in which the vortex street has fully developed.
Next, starting from the $t=200$ flow field, combined with
different time marching schemes and time step sizes
varying from $0.04$ to $0.1$, transient flow is simulated until $t=210$.
CPU consumption and error values are evaluated.
The transient errors are defined with
\begin{equation}
    \begin{aligned}
        \|\epsilon_{My}\|_{t,2}^2
        = & \frac{1}{N_{t}}\sum_{i=1}^{N_t}{(M_{y,i}-M_{y,ref}(t_i))^2} \\
        \|\epsilon_{Mx}\|_{t,2}^2
        = & \frac{1}{N_{t}}\sum_{i=1}^{N_t}{(M_{x,i}-M_{x,ref}(t_i))^2}
    \end{aligned}
    \label{eq:vorstreetErr}
\end{equation}
where $N_{t}$ is the number of time steps for $t\in(200,210]$,
$t_i$ is the time on time steps, and $M_{y,i}$ and $M_{x,i}$ are
the norms of time derivatives of $y$ and $x$ momentum:
\begin{equation}
    M_y = \int_{\Omega}{\left|\derivative{\rho u_y}{t}\right| \dd \Omega},\ \
    M_x = \int_{\Omega}{\left|\derivative{\rho u_x}{t}\right| \dd \Omega}
\end{equation}
Meanwhile, $M_{y,ref}(t_i)$ and $M_{x,ref}(t_i)$ are those values
obtained in the reference solution.

This manner of calculating transient error avoids the difficulty
of preserving all the transient solutions at each time step of
the very fine reference solution.
As the reference solution uses significantly smaller time step,
its temporal error is considered to be negligible.
In other words, the reference solution is a good enough
approximation of the exact solution of the semi-discretized
FV equations \eeqref{eq:FVODE}.
Compared with the reference
solution, the solution using regular time steps
induces major temporal discretization error, which can
be illustrated with the errors constructed in \eeqref{eq:vorstreetErr}.

In order to mitigate the influence of the error induced
by implicit dual time stepping in each step,
each dual time stepping are terminated after
the residual is smaller than $10^{-7}$ of the
starting value.

% \begin{figure}[htbp]
%     \centering
%     \includegraphics[width=0.7\textwidth]{pics/CylinderA1_Re2000_Mesh.png}
%     \caption[]{Part of mesh used in $Re=2000$ vortex shedding problem}
%     \label{fig:CylinderRe2000_Mesh}
% \end{figure}

% \begin{figure}[htbp]
%     \centering
%     \begin{subfigure}{0.5\textwidth}
%         \includegraphics[width=\textwidth]{pics/CylinderA1_Re2000_HM3LBT.png}
%         \caption[]{HIRK $c_2=0.5$}
%         \label{sfig:CylinderRe2000_HM3LBT}
%     \end{subfigure}\hfill
%     \begin{subfigure}{0.5\textwidth}
%         \includegraphics[width=\textwidth]{pics/CylinderA1_Re2000_HM3.png}
%         \caption[]{HIRK $c_2=0.55$}
%         \label{sfig:CylinderRe2000_HM3}
%     \end{subfigure}
%     \caption[]{Comparison of pressure distribution in $Re=2000$ vortex shedding problem}
%     \label{fig:CylinderRe2000}
% \end{figure}

\begin{figure}[htbp]
    \centering
    \begin{subfigure}{0.5\textwidth}
        \includegraphics[width=\textwidth]{pics/Cylinder_fig14.pdf}
        \caption[]{Error vs. time step size}
        \label{sfig:CylinderRe1200_My_C}
    \end{subfigure}\hfill
    \begin{subfigure}{0.5\textwidth}
        \includegraphics[width=\textwidth]{pics/Cylinder_fig4.pdf}
        \caption[]{Error vs. CPU Time}
        \label{sfig:CylinderRe1200_My_E}
    \end{subfigure}
    \caption[]{Convergence and efficiency analysis with $\epsilon_{My}$ in $Re=1200$ vortex shedding problem}
    \label{fig:CylinderRe1200_My}
\end{figure}

\begin{figure}[htbp]
    \centering
    \begin{subfigure}{0.5\textwidth}
        \includegraphics[width=\textwidth]{pics/Cylinder_fig15.pdf}
        \caption[]{Error vs. time step size}
        \label{sfig:CylinderRe1200_Mx_C}
    \end{subfigure}\hfill
    \begin{subfigure}{0.5\textwidth}
        \includegraphics[width=\textwidth]{pics/Cylinder_fig5.pdf}
        \caption[]{Error vs. CPU Time}
        \label{sfig:CylinderRe1200_Mx_E}
    \end{subfigure}
    \caption[]{Convergence and efficiency analysis with $\epsilon_{Mx}$ in $Re=1200$ vortex shedding problem}
    \label{fig:CylinderRe1200_Mx}
\end{figure}

Results of the errors versus time step size and CPU time
are illustrated in
Figure \ref{fig:CylinderRe1200_My} and \ref{fig:CylinderRe1200_Mx}.
Figure \ref{sfig:CylinderRe1200_My_C} and \ref{sfig:CylinderRe1200_Mx_C}
illustrates the results of convergence analysis, where
with the refinement of time step, the temporal
discretization error is reduced.
The ESDIRK4, DITR U2R2 $c_2=0.5$, DITR U3R1 methods
approximately 4th order convergence in
Figure \ref{sfig:CylinderRe1200_My_C} and \ref{sfig:CylinderRe1200_Mx_C},
while U2R1 and U2R2 $c_2=0.55$ are 3rd order.
BDF2 is indeed 2nd order accurate, and is only able
to be comparable with high-order methods when $\inc t$
is very small.
Among the DITR methods, U2R2 $c_2=0.5$ has the smallest error.
U3R1 and U2R2 $c_2=0.55$ are close but U3R1 has higher order of convergence.
Although better than BDF2,
U2R1 has the worst error of vortex street simulation among DITR methods.

Figure \ref{sfig:CylinderRe1200_My_E} and \ref{sfig:CylinderRe1200_Mx_E}
use CPU Time as the horizontal axis, therefore comparison of efficiency is illustrated.
In Figure \ref{sfig:CylinderRe1200_My_E} and \ref{sfig:CylinderRe1200_Mx_E},
in order to achieve an error with magnitude of $10^{-3}$, the
high-order methods are significantly more economic than the
2nd order BDF2.
Among the high order methods, DITR U2R2 $c_2=0.55$ is
close to ESDIRK4 in efficiency, while
U2R2 $c_2=0.5$ and U3R1 are more efficient than ESDIRK4.
The most efficient U2R2 $c_2=0.5$ takes less than 70\% of
the time used in ESDIRK4 when the error is $10^-4$.



% \begin{figure}[htbp]
%     \centering
%     \begin{subfigure}{0.5\textwidth}
%         \includegraphics[width=\textwidth]{pics/Cylinder_fig13.pdf}
%         \caption[]{Error vs. time step size}
%         \label{sfig:CylinderRe1200_St_C}
%     \end{subfigure}\hfill
%     \begin{subfigure}{0.5\textwidth}
%         \includegraphics[width=\textwidth]{pics/Cylinder_fig3.pdf}
%         \caption[]{Error vs. effective iterations}
%         \label{sfig:CylinderRe1200_St_E}
%     \end{subfigure}
%     \caption[]{Convergence and efficiency analysis with $\epsilon_{St}$ in $Re=1200$ vortex shedding problem}
%     \label{fig:CylinderRe1200_St}
% \end{figure}

\subsection{Double mach reflection}

The double Mach reflection problem \cite{woodward1984dmr} is tested to
compare the resolution capabilities
of different time marching schemes.
The double Mach reflection computes
inviscid ideal gas in $[0,4]\times[0,1]$,
initialized by a Ma 10 moving shock located
at $x = 1/6 + \cot(60^\circ) y$.
The boundary of $y=0, x\in[1/6,4]$ is
inviscid wall, and all other boundaries
conforms with the Ma 10 moving shock.
Details about the initial and boundary settings may be found in \cite{woodward1984dmr}.
The compact FV scheme is additionally equipped with
an accuracy preserving CWBAP limiter \cite{wu2023cwbap},
which grants the ability to capture spatial  discontinuities.
To handle strong discontinuities, the local Lax-Friedrichs flux
is used here.
The computations are conducted on a uniform quadrilateral mesh with
mesh size $h=1/480$.
Physical time step is set to a relatively large $\inc t= 2\times10^{-4}$,
and the solutions at $t=0.25$ are compared.

\begin{figure}[htbp]
    \centering
    \begin{subfigure}{0.5\textwidth}
        \includegraphics[width=\textwidth]{pics/DM480_HM3LBT.png}
        \caption[]{DITR U2R2 $c_2=0.5$}
        \label{sfig:DM480_HM3LBT}
    \end{subfigure}\hfill
    \begin{subfigure}{0.5\textwidth}
        \includegraphics[width=\textwidth]{pics/DM480_HM3.png}
        \caption[]{DITR U2R2 $c_2=0.55$}
        \label{sfig:DM480_HM3}
    \end{subfigure}
    \caption{Density in double mach reflection problem, DITR U2R2}
    \label{fig:DM480-1}
\end{figure}

\begin{figure}[htbp]
    \centering
    \begin{subfigure}{0.5\textwidth}
        \includegraphics[width=\textwidth]{pics/DM480-1_HM3U2R1.png}
        \caption[]{DITR U2R1}
        \label{sfig:DM480_HM3U2R1}
    \end{subfigure}\hfill
    \begin{subfigure}{0.5\textwidth}
        \includegraphics[width=\textwidth]{pics/DM480-1_HM3U3R1.png}
        \caption[]{DITR U3R1}
        \label{sfig:DM480_HM3U3R1}
    \end{subfigure}
    \caption{Density in double mach reflection problem, DITR U2R1 and U3R1}
    \label{fig:DM480-2}
\end{figure}

\begin{figure}[htbp]
    \centering
    \begin{subfigure}{0.5\textwidth}
        \includegraphics[width=\textwidth]{pics/DM480-1_ESDIRK4-T2.png}
        \caption[]{ESDIRK4}
        \label{sfig:DM480_ESDIRK4}
    \end{subfigure}\hfill
    \begin{subfigure}{0.5\textwidth}
        \includegraphics[width=\textwidth]{pics/DM480-1_BDF-T2.png}
        \caption[]{BDF2}
        \label{sfig:DM480_BDF2}
    \end{subfigure}
    \caption{Density in double mach reflection problem, baseline methods}
    \label{fig:DM480-3}
\end{figure}

Figure \ref{fig:DM480-1}, \ref{fig:DM480-2} and \ref{fig:DM480-3}
illustrate a zoomed view of density distribution.
From Figure \ref{fig:DM480-3},
it is observed BDF2 produces
a deformed Mach stem, which, according
to further tests, can be
corrected by using smaller $\inc t$.
Using the same $\inc t= 2\times10^{-4}$,
all high-order methods produce a
normal Mach stem.
Meanwhile, the second order BDF2
completely smears the K-H instability
in the shear layer, despite that the spatial
discretization is 4th order accurate.
All high-order time march methods methods can successfully
simulate K-H instability and resolve the
small structures induced.
The results of DITR U2R2 $c_2=0.5$, U2R2  $c_2=0.55$
and U3R1 are very similar with ESDIRK4, while U2R1
appears to produce less vortices in the shear layer.

Like the results from isentropic vortex and vortex shedding simulation,
using the same time step, DITR methods
use less time than ESDIRK4 due to the reduction of stage numbers.
The U2R2 $c_2=0.5$ method consumes around $63\%$ of ESDIRK4's time,
and U3R1 uses $79\%$.



\section{Conclusion}
\label{sec:Conc}

The current paper has described a method of
developing time marching schemes using a
direct integration and a temporal reconstruction.
Using one form of quadrature rule and several
forms of temporal reconstruction, a series of
specific DITR methods are discovered and analyzed.
The DITR U2R2 method is $A$-stable, and with $c_2=0.5$, U2R2 is 4th order accurate.
The DITR U2R1 method is $L$-stable with 3rd order accuracy, and
the DITR U3R1 method is $L$-stable with 4th order accuracy.
In order to apply these time marching methods to flow problems,
stability in dual time stepping is analyzed.
It is discovered that using the nested iteration method, pseudo time marching is unstable.
Using the stage-decoupled iteration method with preconditioning,
the implicit time steps of DITR can steadily converge with proper choices of
the preconditioning parameter $\beta$.
The stage-decoupled iteration method is easy to be implemented in a
matrix-free manner.

After analyzing the results of numerical tests, it is confirmed
all the DITR methods can exhibit their theoretical order of accuracy.
With the same time step, DITR methods are much more accurate than
BDF2 and comparable with ESDIRK4.
Due to having only 2 stages, DITR methods takes less time than ESDIRK4
each step.
When reaching the same accuracy, some DITR methods are
distinctively faster than ESDIRK4.
In summary, the DITR methods are steady implicit time marching methods
that are easy to implement and more efficient.

The iteration methods discussed in the current paper is still crude.
For the successful stage-decoupled iteration method,
more variants of the preconditioning matrix and details including
the choice of pseudo time step, the influence of approximate
Jacobian and the influence of linear solvers are not discussed here.
Meanwhile, more delicate analysis on the choosing of preconditioning parameter
$\beta$ could still be conducted.
For the unsteady nested iteration method,
tricks to stabilize the method including modifying the matrices and
using methods other than dual-time stepping are not considered.
Therefore, more delicate and efficient iteration methods of DITR could
still be found.