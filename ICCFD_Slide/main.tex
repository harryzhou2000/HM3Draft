%%=======================================================================
% !Mode:: "TeX:UTF-8"
% !TEX program  = xeLatex
%%=======================================================================
% 模板名称:thubeamer
% 模板版本:V1.0.0
% 模板作者:杨敬轩(Jingxuan Yang)
% 联系作者:yangjx20@mails.tsinghua.edu.cn & yanglatex2e@gmail.com
% 模板适用:清华大学风格 Beamer 模板
% 模板编译:手动编译方法参看 README.md 或 thubeamer.pdf
%          编译 beamer 之前必须编译说明文档:make doc 或双击 makedoc.bat
%          编译说明文档同时分离出四个样式文件 *thubeamer.sty
%          GNU make 工具:make beamer
%          Windows 批处理脚本:双击 makebeamer.bat 自动编译 beamer
%          更多编译细节详见说明文档:thubeamer.pdf
% 更新时间:2021/07/07
% 模板帮助:请**务必务必务必**阅读 thubeamer.pdf 说明文档,文档查看方法:
%          下载模板文件夹里就有,如果是从 CTAN 上安装更新本模板,则通过
%          cmd 命令行:texdoc thubeamer 查看文档
%          推荐前往模板的 GitHub 仓库获取最新文件,地址:
%          https://github.com/YangLaTeX/thubeamer
%%=======================================================================

% 设置文档类别为 <beamer>
\documentclass[aspectratio=169,serif]{beamer} % 设置长宽比为 16:9
% \documentclass{beamer}

% 使用 <thubeamer> 主题
% 模板选项如下
% (a.1) smoothbars: 页面顶端单行显示目录,默认选项
% \usetheme{thubeamer}

% (a.2) sidebar: 页面左侧分栏显示目录
\usetheme[sidebar]{thubeamer}

% (b) sectiontoc: 在每节(section)前显示目录,并高亮显示当前节,默认不显示

% (c) subsectiontoc: 在每小节(subsection)前显示目录,并高亮显示当前节和当前小节,默认不显示

% 图片存放路径
\graphicspath{{figures/}{../}}


% % About:  Macros for Vector, Matrix, Tensor, Math Operator and Misc
% Author: Jingxuan Yang

% vectors
\newcommand{\va}{\bm{a}}       \newcommand{\vah}{\hat{\bm{a}}}        \newcommand{\ah}{\hat{a}}    \newcommand{\vat}{\tilde{\bm{a}}}       \newcommand{\at}{\tilde{a}}
\newcommand{\vb}{\bm{b}}       \newcommand{\vbh}{\hat{\bm{b}}}        \newcommand{\bh}{\hat{b}}    \newcommand{\vbt}{\tilde{\bm{b}}}       \newcommand{\bt}{\tilde{b}}
\newcommand{\vc}{\bm{c}}       \newcommand{\vch}{\hat{\bm{c}}}        \newcommand{\ch}{\hat{c}}    \newcommand{\vct}{\tilde{\bm{c}}}       \newcommand{\ct}{\tilde{c}}
\newcommand{\vd}{\bm{d}}       \newcommand{\vdh}{\hat{\bm{d}}}        \newcommand{\dhat}{\hat{d}}  \newcommand{\vdt}{\tilde{\bm{d}}}       \newcommand{\dt}{\tilde{d}}
\newcommand{\ve}{\bm{e}}       \newcommand{\veh}{\hat{\bm{e}}}        \newcommand{\eh}{\hat{e}}    \newcommand{\vet}{\tilde{\bm{e}}}       \newcommand{\et}{\tilde{e}}
\newcommand{\vf}{\bm{f}}       \newcommand{\vfh}{\hat{\bm{f}}}        \newcommand{\fh}{\hat{f}}    \newcommand{\vft}{\tilde{\bm{f}}}       \newcommand{\ft}{\tilde{f}}
\newcommand{\vg}{\bm{g}}       \newcommand{\vgh}{\hat{\bm{g}}}        \newcommand{\gh}{\hat{g}}    \newcommand{\vgt}{\tilde{\bm{g}}}       \newcommand{\gt}{\tilde{g}}
\newcommand{\vh}{\bm{h}}     \newcommand{\vhh}{\hat{\bm{h}}}        \newcommand{\hh}{\hat{h}}    \newcommand{\vht}{\tilde{\bm{h}}}       \newcommand{\htild}{\tilde{h}}
\newcommand{\vi}{\bm{i}}       \newcommand{\vih}{\hat{\bm{i}}}        \newcommand{\ih}{\hat{i}}    \newcommand{\vit}{\tilde{\bm{i}}}       \newcommand{\itild}{\tilde{i}}
\newcommand{\vj}{\bm{j}}       \newcommand{\vjh}{\hat{\bm{j}}}        \newcommand{\jh}{\hat{j}}    \newcommand{\vjt}{\tilde{\bm{j}}}       \newcommand{\jt}{\tilde{j}}
\newcommand{\vk}{\bm{k}}       \newcommand{\vkh}{\hat{\bm{k}}}        \newcommand{\kh}{\hat{k}}    \newcommand{\vkt}{\tilde{\bm{k}}}       \newcommand{\kt}{\tilde{k}}
\newcommand{\vl}{\bm{l}}       \newcommand{\vlh}{\hat{\bm{l}}}        \newcommand{\lh}{\hat{l}}    \newcommand{\vlt}{\tilde{\bm{l}}}       \newcommand{\lt}{\tilde{l}}
\newcommand{\vm}{\bm{m}}       \newcommand{\vmh}{\hat{\bm{m}}}        \newcommand{\mh}{\hat{m}}    \newcommand{\vmt}{\tilde{\bm{m}}}       \newcommand{\mt}{\tilde{m}}
\newcommand{\vn}{\bm{n}}       \newcommand{\vnh}{\hat{\bm{n}}}        \newcommand{\nh}{\hat{n}}    \newcommand{\vnt}{\tilde{\bm{n}}}       \newcommand{\nt}{\tilde{n}}
\newcommand{\vo}{\bm{o}}       \newcommand{\voh}{\hat{\bm{o}}}        \newcommand{\oh}{\hat{o}}    \newcommand{\vot}{\tilde{\bm{o}}}       \newcommand{\ot}{\tilde{o}}
\newcommand{\vp}{\bm{p}}       \newcommand{\vph}{\hat{\bm{p}}}        \newcommand{\ph}{\hat{p}}    \newcommand{\vpt}{\tilde{\bm{p}}}       \newcommand{\pt}{\tilde{p}}
\newcommand{\vq}{\bm{q}}       \newcommand{\vqh}{\hat{\bm{q}}}        \newcommand{\qh}{\hat{q}}    \newcommand{\vqt}{\tilde{\bm{q}}}       \newcommand{\qt}{\tilde{q}}
\newcommand{\vr}{\bm{r}}       \newcommand{\vrh}{\hat{\bm{r}}}        \newcommand{\rh}{\hat{r}}    \newcommand{\vrt}{\tilde{\bm{r}}}       \newcommand{\rt}{\tilde{r}}
\newcommand{\vs}{\bm{s}}       \newcommand{\vsh}{\hat{\bm{s}}}        \newcommand{\sh}{\hat{s}}    \newcommand{\vst}{\tilde{\bm{s}}}       \newcommand{\st}{\tilde{s}}
\newcommand{\vt}{\bm{t}}       \newcommand{\vth}{\hat{\bm{t}}}        \newcommand{\that}{\hat{t}}  \newcommand{\vtt}{\tilde{\bm{t}}}       \newcommand{\ttil}{\tilde{t}}
\newcommand{\vu}{\bm{u}}       \newcommand{\vuh}{\hat{\bm{u}}}        \newcommand{\uh}{\hat{u}}    \newcommand{\vut}{\tilde{\bm{u}}}       \newcommand{\ut}{\tilde{u}}
\newcommand{\vv}{\bm{v}}       \newcommand{\vvh}{\hat{\bm{v}}}        \newcommand{\vhat}{\hat{v}}    \newcommand{\vvt}{\tilde{\bm{v}}}       \newcommand{\vtild}{\tilde{v}}
\newcommand{\vw}{\bm{w}}       \newcommand{\vwh}{\hat{\bm{w}}}        \newcommand{\wh}{\hat{w}}    \newcommand{\vwt}{\tilde{\bm{w}}}       \newcommand{\wt}{\tilde{w}}
\newcommand{\vx}{\bm{x}}       \newcommand{\vxh}{\hat{\bm{x}}}        \newcommand{\xh}{\hat{x}}    \newcommand{\vxt}{\tilde{\bm{x}}}       \newcommand{\xt}{\tilde{x}}
\newcommand{\vy}{\bm{y}}       \newcommand{\vyh}{\hat{\bm{y}}}        \newcommand{\yh}{\hat{y}}    \newcommand{\vyt}{\tilde{\bm{y}}}       \newcommand{\yt}{\tilde{y}}
\newcommand{\vz}{\bm{z}}       \newcommand{\vzh}{\hat{\bm{z}}}        \newcommand{\zh}{\hat{z}}    \newcommand{\vzt}{\tilde{\bm{z}}}       \newcommand{\zt}{\tilde{z}}

\newcommand{\valpha}{\bm{\alpha}}
\newcommand{\vbeta}{\bm{\beta}}
\newcommand{\vgamma}{\bm{\gamma}}
\newcommand{\vtheta}{\bm{\theta}}
\newcommand{\vlambda}{\bm{\lambda}}
\newcommand{\vmu}{\bm{\mu}}
\newcommand{\vomega}{\bm{\omega}}

\newcommand{\mSigma}{\bm{\Sigma}}

\newcommand{\Fc}{\mathcal{F}}
\newcommand{\Xc}{\mathcal{X}}
\newcommand{\Yc}{\mathcal{Y}}
\newcommand{\Zc}{\mathcal{Z}}
\newcommand{\Gc}{\mathcal{G}}
\newcommand{\Hc}{\mathcal{H}}
\newcommand{\Dc}{\mathcal{D}}
\newcommand{\Cc}{\mathcal{C}}
\newcommand{\Rc}{\mathcal{R}}
\newcommand{\Lc}{\mathcal{L}}
\newcommand{\Nc}{\mathcal{N}}
\newcommand{\Tc}{\mathcal{T}}

% matrices
\newcommand{\ma}{\bm{A}}
\newcommand{\mb}{\bm{B}}
\newcommand{\md}{\bm{D}}
\newcommand{\mH}{\bm{H}}
\newcommand{\mE}{\bm{E}}
\newcommand{\mi}{\bm{I}}
\newcommand{\mk}{\bm{K}}
\newcommand{\ml}{\bm{L}}
\newcommand{\mn}{\bm{N}}
\newcommand{\mP}{\bm{P}}
\newcommand{\mq}{\bm{Q}}
\newcommand{\mr}{\bm{R}}
\newcommand{\mU}{\bm{U}}
\newcommand{\mv}{\bm{V}}
\newcommand{\mw}{\bm{W}}
\newcommand{\mx}{\bm{X}}
\newcommand{\my}{\bm{Y}}
\newcommand{\mz}{\bm{Z}}

% tensors
\newcommand{\tp}{\mathsf{P}}
\newcommand{\tu}{\mathsf{U}}
\newcommand{\tx}{\mathsf{X}}
\newcommand{\ty}{\mathsf{Y}}
\newcommand{\tz}{\mathsf{Z}}
\newcommand{\tw}{\mathsf{W}}
\newcommand{\tf}{\mathsf{F}}
\newcommand{\ta}{\mathsf{A}}
\renewcommand{\th}{\mathsf{H}}

% norms
\newcommand{\mynorm}[2]{\| {#1} \|_{#2}}
\newcommand{\norm}[2]{\mynorm{#1}{#2}}
\newcommand{\bignorm}[2]{\left\| {#1} \right\|_{#2}}
\newcommand{\norml}[1]{\mynorm{#1}{1}}
\newcommand{\bignorml}[1]{\bignorm{#1}{1}}
\newcommand{\infnorm}[1]{\mynorm{#1}{\infty}}
\newcommand{\biginfnorm}[1]{\bignorm{#1}{\infty}}
\newcommand{\oneinf}{\ell_{1,\infty}}
\newcommand{\onetwo}{\ell_{1,2}}
\newcommand{\oneinfnorm}[1]{\mynorm{#1}{1,\infty}}
\newcommand{\bigoneinf}[1]{\bignorm{#1}{1,\infty}}
\newcommand{\onetwonorm}[1]{\mynorm{#1}{1,2}}
\newcommand{\bigonetwo}[1]{\bignorm{#1}{1,2}}
\newcommand{\enorm}[1]{\mynorm{#1}{2}}
\newcommand{\bigenorm}[1]{\bignorm{#1}{2}}
\newcommand{\znorm}[1]{\mynorm{#1}{0}}
\newcommand{\bigznorm}[1]{\bignorm{#1}{0}}
\newcommand{\frob}[1]{\|{#1}\|_{\text{F}}}
\newcommand{\bigfrob}[1]{\bignorm{#1}{\text{F}}}
\newcommand{\grpnorm}[2]{\norm{#1}{\text{Gr}(#2)}}

% math operators
\DeclareMathOperator*{\argmin}{argmin}
\DeclareMathOperator*{\argmax}{argmax}
\DeclareMathOperator{\divg}{div}
\DeclareMathOperator{\dom}{dom}
\DeclareMathOperator{\interior}{int}
\DeclareMathOperator{\ri}{ri}
\DeclareMathOperator{\sgn}{sgn}
\DeclareMathOperator{\trace}{Tr}
\DeclareMathOperator{\diag}{diag}
\DeclareMathOperator{\rank}{rank}
\DeclareMathOperator{\range}{range}
\DeclareMathOperator{\vect}{vec}
\DeclareMathOperator{\prox}{prox}
\DeclareMathOperator{\intr}{int}
\DeclareMathOperator{\relint}{ri}
\DeclareMathOperator{\hard}{hard}
\DeclareMathOperator{\id}{id}
% \DeclareMathOperator{\tf}{tf}
% \DeclareMathOperator{\idf}{idf}

% misc
\newcommand{\gs}{\geqslant}
\newcommand{\ls}{\leqslant}
\newcommand{\set}[1]{\left\{ {#1}\right\}}

\newcommand{\defeq}{\ \stackrel{\text{def}}{=}\ }
\newcommand{\ip}[2]{\left\langle#1, #2\right\rangle}
\newcommand{\reals}{\mathbb{R}}
\newcommand{\complex}{\mathbb{C}}
\newcommand{\half}{\frac{1}{2}}


% \renewcommand\mathfamilydefault{\mathde}
% \usefonttheme[onlymath]{serif}
\usepackage[nodayofweek]{datetime}
\newdateformat{mydate}{\twodigit{\THEDAY}{ }\shortmonthname[\THEMONTH], \THEYEAR}

% \usefonttheme{professionalfonts} % using non standard fonts for beamer
% \usefonttheme{serif} % default family is serif
\usepackage{fontspec}
% \setsansfont{Times New Roman} % this works

\usepackage{amssymb}
\usepackage{amsmath}
\usepackage{physics}
% \usepackage{cmbright}
% \fontencoding{OT1}\fontfamily{cmbr}\selectfont %to load ot1cmbr.fd
% \DeclareFontShape{OT1}{cmbr}{bx}{n}{% change bx definition
% <->cmbrbx10%
% }{}
% \normalfont % back to normalfont
\usepackage{caption}
\usepackage{subcaption}
\renewcommand{\tablename}{Table}
\renewcommand{\figurename}{Figure}

% 封面信息,方括号内容是显示在左侧边栏的内容(当选择 sidebar 主题时有效)
\title[DITR]{High-order Accurate Implicit Scheme Based on 
Temporal Reconstruction for Solving Compressible 
Navier-Stokes Equations}
\author[Hanyu Zhou, Yu-Xin Ren]{Hanyu Zhou, Yu-Xin Ren}
\institute[Tsinghua University]{\small  Tsinghhua University}
\date{\small \vskip -10pt \today}



% 开始写文章
\begin{document}


% !TeX root = ./HM3Draft.tex

\newcommand{\trans}{^\mathrm{T}}

\newcommand{\U}{\mathbf{U}}
\newcommand{\F}{\mathbf{F}}
\newcommand{\x}{\mathbf{x}}

\newcommand{\OO}{\mathbf{\Omega}}
\newcommand{\UM}{\overline{\U}}
\newcommand{\Fn}{\tilde{\F}}
\newcommand{\n}{\mathbf{n}}
\newcommand{\uu}{\overline{\mathbf{U}}}
\newcommand{\R}{\mathbf{R}}
\newcommand{\inc}{{\Delta}} % beamer can't take \mathbf{delta}
\newcommand{\Tau}{\mathrm{T}}
\renewcommand{\real}{\mathrm{Re}}
\newcommand{\imag}{\mathrm{Im}}

\newcommand{\CFLt}{\text{CFL}_t}
\newcommand{\CFLtau}{\text{CFL}_\tau}

\newcommand{\eeqref}[1]{Eq.\eqref{#1}}
\newcommand{\us}{\mathbf{u}}


\renewcommand{\Res}{\mathcal{R}}
\newcommand{\Jres}{\mathcal{J}}
\newcommand{\eye}{\mathbf{I}}
\newcommand{\J}{\mathbf{J}}
\newcommand{\FF}{\mathcal{F}}
\newcommand{\Pm}{\mathbf{P}}
\newcommand{\Pe}{\mathrm{P}}

\newcommand{\imagUnit}{\mathrm{i}}

\newcommand{\GG}{\mathcal{G}}


\newcommand{\A}{\mathbf{A}}
\newcommand{\B}{\mathbf{B}}
\newcommand{\C}{\mathbf{C}}

% 标题页
\begin{frame}
  \maketitle
\end{frame}

% 目录页
\section*{Contents}
\frame{
  \frametitle{\secname}
  \tableofcontents[hideallsubsections]
}

\section{Introduction}

% \subsection{课题主要研究内容}

\begin{frame}{\secname: Implicit Time Marching}
  \begin{itemize}
    \item  A widely applied practice in CFD:
          spacial discretization + ODE integrator
          (Time Marching Scheme/Method)
    \item Variants of RK3, RK4 are popular
          explicit schemes.
    \item Implicit time marching can deal with stiffness and
          use desirable time step size:
          \begin{itemize}
            \item Wall-bounded viscous flow, especially when applying a RANS model in BL
            \item Low Ma or incompressible flow, propagation of sound wave ignored
            \item Mesh overly fine in some regions / directions
            \item Additional stiffness caused by source term
                  (chemical, physical...)
          \end{itemize}
    \item Implicit time marching can be easily applied with additional algebraic
          equations (see section \ref{sec:CFV}).
  \end{itemize}
\end{frame}




\begin{frame}{\secname: Stability}
  \begin{itemize}
    \item Not any implicit scheme is stable,
          even in a linear context
          \begin{block}{}
            When solving the linear ODE with $\inc t$
            $$
              \dv{y}{t} = \lambda y = (\frac{z}{\inc t})y
            $$
            and $y^{n+1}=R(z) y^{n}$, then $|R(z)|\leq1, \forall \real(z)\leq 0$ is
            called $A$-stability, added with $|R(\infty)|=0$ is called $L$-stability
            \cite{wanner1996solving}.
          \end{block}
          \begin{block}{Dahlquist's second barrier
              \cite{dahlquist1963special}}
            A linear multistep method that has $A$-stability
            can only be {\bf 2nd order accurate at most}.
            Specially, BDF2 and Trapezoid Rule (Crank-Nicolson)
            are $A$-stable, and BDF2 is $L$-stable.
          \end{block}
  \end{itemize}
\end{frame}

\begin{frame}{\secname: High-order Implicit Time Marching}
  \begin{itemize}
    \item High order CFD method is an active field of research
          \cite{cheng2009highHOReview}.
    \item A lot of high order spacial discretization
          methods in literature can be combined
          with {\bf arbitrary} time marching, but higher-order
          time marching schemes greatly reduces the need of time refinement.
    \item Schemes being both implicit and high-order (>2): mostly implicit RK schemes
          \begin{itemize}
            \item Fully implicit RK (IRK): hard to implement,
                  needs careful treatment \cite{pazner2017stage,jameson2017evaluation}.
            \item Diagonally implicit RK
                  (including SDIRK, ESDIRK) can be solved stage-by-stage.
                  The popular 4th order ESDIRK \cite{kennedy2003additiveARK,kvaerno2004singly}
                  has {\bf 5 implicit stages}, and is $L$-stable.
            \item Other variations of implicit RKs, like MIRK and NIRK
                  \cite{cash1975classMIRKOrig,kulikov2006familyNIRKOrig}
                  are seldom tested in CFD.
          \end{itemize}
  \end{itemize}
\end{frame}

\begin{frame}{\secname: Our Work}
  \begin{itemize}
    \item Our recent exploration indicates a novel
          way of obtaining stable, high-order accurate and
          efficient implicit time marching schemes:
          \begin{block}{DITR Time Marching Scheme}
            \begin{itemize}
              \item {\bf Direct Integration}
                    integrate the ODE using a quadrature rule
              \item {\bf Temporal Reconstruction}
                    approximate the nodal values using a polynomial
                    interpolation
            \end{itemize}
          \end{block}
    \item Selecting different DI and TR affects
          the accuracy and stability of DITR methods.
    \item Applied to high-order Compact Finite Volume
          (Variational Reconstruction), tested in compressible
          flow problems.
  \end{itemize}
\end{frame}

\section{Compact Finite Volume}
\label{sec:CFV}

\begin{frame}{\secname: Finite Volume}
  \begin{itemize}
    \item High-order spacial discretization are able to demonstrate
          high spacial resolution and low spacial discretization error.
    \item
          Briefly introduce the N-S equations' spacial discretization
          \begin{block}{Compressible Navier-Stokes Equation}
            \begin{equation}
              \label{eq:NS}
              \frac{\partial \U}{\partial t} +
              \mathbf\nabla \cdot (\F - \F_v)  = 0
            \end{equation}
          \end{block}
    \item Using Finite Volume (FV) discretization, each
          cell records its mean value as physical DOF:
          \begin{block}{}
            \begin{equation}
              \label{eq:FVMean}
              \UM_i = \frac{1}{\overline{\Omega}_i}\int_{\OO_i}\U(\x)\dd \Omega
            \end{equation}
          \end{block}
  \end{itemize}
\end{frame}

\begin{frame}{\secname: Finite Volume Method}
  \begin{itemize}
    \item Next, one can use polynomial reconstruction per cell:
          \begin{block}{}
            \begin{equation}
              \label{eq:FVRec}
              \U_i(\x) = \UM_i + \sum_{l=1}^{\mathrm{NDOF}(k)}{\U_i^l\varphi_{i,l}(\x) }
            \end{equation}
          \end{block}
    \item All the flux terms are discretized on the surface due to Gauss-Green formula
          \begin{block}{}
            \begin{equation}
              \label{eq:FVInt}
              \frac{\dd\UM_i}{\dd t}
              +\sum_{j\in S_i, j\neq i}{\left(
                \int_{f_{i,j}}{
                  [\Fn(\U_i,\U_j) - \Fn_v(\U_i,\U_j, \nabla \U_i, \nabla \U_j)] \cdot \n  \dd A
                }\right)
              }
              = 0
            \end{equation}
          \end{block}
  \end{itemize}
\end{frame}

\begin{frame}{\secname: Reconstruction}
  \begin{itemize}
    \item Reconstruct polynomial coefficients using compact
            {\bf variational reconstruction}\cite{wang2017compact_VR},
          by minimizing a global
          functional:
          \begin{block}{}
            \begin{equation}
              I = \sum{I_f},
            \end{equation}
            \begin{equation}
              I_f = \omega_f^G\int_{\OO_i \cap \OO_j}{
                \sum_{p+q=0}^{p + q\leq k}\left[
                  \omega_f^D(p,q)
                  d_{ij}^{p+q}
                  \left(
                  \partialderivative{^{p+q}u_i(x,y)}{x^p\partial y^q}
                  -
                  \partialderivative{^{p+q}u_j(x,y)}{x^p\partial y^q}
                  \right)
                  \right]^2
                \dd f
              }
            \end{equation}
          \end{block}
    \item The functional measures interface jump of polynomials
  \end{itemize}
\end{frame}

\section{DITR Methods}
\subsection{Construction}


\begin{frame}[allowframebreaks]{\secname: \subsecname}
  The spacial discretization yields an ODE $\dv{\U}{t}=\R(t,\U)$.
  DI and TR are:
  \begin{block}{Direct Integration}
    \begin{itemize}
      \item We only use a 3-point quadrature
            rule on $[t^n,t^{n+1}]$:
            \begin{equation}
              \label{eq:Quad3}
              \begin{aligned}
                \frac{\uu^{n+1} - \uu^{n}}{\inc t^n} = & \frac{1}{\inc t^n}
                \int_{t^n}^{t^{n+1}}{
                \R(t, \uu) \dd t
                }                                                           \\ \approx &
                b_1\R(t^n, \uu^n)
                +
                b_2\R(t^{n+c_2}, \uu^{n+c_2})
                +
                b_3\R(t^{n+1}, \uu^{n+1})
              \end{aligned}
            \end{equation}
      \item The ${n+c_2}$ value is an intermediate stage value, and $t^{n+1}$ is the desired new step value.
            $\uu^{n+c_2},\uu^{n+1}$ are unknown values.
            $c_2\in (0,1)$
    \end{itemize}
  \end{block}


  \begin{block}{Temporal Reconstruction}
    \begin{itemize}
      \item Currently we use only step values for TR (no stage values):
            \begin{equation}
              \begin{aligned}
                \label{eq:TR}
                \uu(t) & \approx
                A^n_0(t)\uu^{n - 1} +
                A^n_1(t)\uu^{n} +
                A^n_2(t)\uu^{n + 1}
                \\ & +
                \inc t^n D^n_0(t)\R^{n - 1} +
                \inc t^n D^n_1(t)\R^{n} +
                \inc t^n D^n_2(t)\R^{n + 1}
              \end{aligned}
            \end{equation}
      \item $A^n_i(t), D^n_i(t), i=0,1,2$
            are polynomial base functions, similar to Hermite
            interpolation bases, as $\R=\dv{\uu}{t}$ time derivative values are
            used as interpolation conditions.
    \end{itemize}
  \end{block}


  \begin{itemize}
    \item DI \eeqref{eq:Quad3} and TR \eeqref{eq:TR} combined can form
          a closed system.
    \item TR can be used to evaluate the $n+c_2$
          stage value, and produces an additional equation.
    \item More stage values need more TR equations.
  \end{itemize}
\end{frame}

\subsection{Order of Accuracy}

\begin{frame}[allowframebreaks]{\secname: \subsecname}
  \begin{itemize}
    \item DI \eeqref{eq:Quad3} can be precise at 2nd or 3rd ($c_2=0.5$) degree.
    \item TR can reach 3rd degree accurate using 4 of the interpolation conditions.
    \item Suppose DI has precision degree of $m$ the truncation error is:
          \begin{equation}
            \label{eq:Quad3Err}
            \begin{aligned}
              \uu^{n+1} - \uu^{n} = & \int_{t^n}^{t^{n+1}}{
              \R(t, \uu)\dd t}                              \\  = &
              {\inc t}^{n}
              \left[
                b_1\R(t^n, \uu^n)
                +
                b_2\R(t^{n+c_2}, \uu^{n+c_2})
                +
                b_3\R(t^{n+1}, \uu^{n+1})
                \right]
              \\ + &
              O((\inc t^{n})^{m+2} )
            \end{aligned}
          \end{equation}
    \item and TR has precision degree of $l$
          \begin{equation}
            \begin{aligned}
              \label{eq:TRErr}
              \uu(t^{n+c_2}) & =
              A^n_0(t^{n+c_2})\uu^{n - 1} +
              A^n_1(t^{n+c_2})\uu^{n} +
              A^n_2(t^{n+c_2})\uu^{n + 1}
              \\ & +
              \inc t^n D^n_0(t^{n+c_2})\R^{n - 1} +
              \inc t^n D^n_1(t^{n+c_2})\R^{n} +
              \inc t^n D^n_2(t^{n+c_2})\R^{n + 1}
              \\ & +
              O((\inc t^{n})^{l+1} )
            \end{aligned}
          \end{equation}
    \item Substitute \eeqref{eq:TRErr} into \eeqref{eq:Quad3Err} (assuming $\R$ is non-singular):
          \begin{equation}
            \begin{aligned}
              \uu^{n+1}
              = &
              \uu^{n} + {\inc t}^{n}
              \left[
                b_1\R(t^n, \uu^n)
                +
                b_2\R(t^{n+c_2}, \uu^{n+c_2})
                +
                b_3\R(t^{n+1}, \uu^{n+1})
                \right]
              \\ + &
              O((\inc t^{n})^{m+2}  + O((\inc t^{n})^{l+2} ))
            \end{aligned}
            \label{eq:fullLTE}
          \end{equation}
    \item The theoretical order of accuracy: $\min(m,l) + 1$
  \end{itemize}
\end{frame}


\subsection{Variants}

\begin{frame}{\secname: \subsecname}
  \begin{itemize}
    \item TR \eeqref{eq:TR} could use any subset of those interpolation conditions,
          resulting in different DITR methods
    \item To exploit all information in a single step:
          \begin{block}{DITR U2R2 Method}
            \begin{itemize}
              \item Using $\uu^{n},\uu^{n+1},\R^{n},\R^{n+1}$ for TR
              \item 4th order accurate if $c_2=0.5$, 3rd order otherwise
              \item $A$-stable if $c_2 \geq 0.5$, and more stable as $c_2$ increases
            \end{itemize}
          \end{block}
  \end{itemize}
\end{frame}

\begin{frame}{\secname: \subsecname}
  \begin{itemize}
    \small
    \item Using 2nd degree interpolation
          \begin{block}{DITR U2R1 Method}
            \begin{itemize}
              \item Using $\uu^{n},\uu^{n+1},\R^{n+1}$ for TR
              \item 3rd order accurate
              \item Proved to be $L$-stable
            \end{itemize}
          \end{block}
    \item Using multistep information (not linear multistep)
          \begin{block}{DITR U3R1 Method}
            \begin{itemize}
              \item Using $\uu^{n-1},\uu^{n},\uu^{n+1},\R^{n+1}$ for TR
              \item 4th order accurate if $c_2=0.5$
              \item Proved to be $L$-stable
            \end{itemize}
          \end{block}
  \end{itemize}
\end{frame}

\begin{frame}{\secname: \subsecname}
  \begin{itemize}
    \item The variants have unified form:
          \begin{block}{DITR Schemes}
            \begin{equation}
              \uu^{n+1} = \uu^{n} + \inc t^n\left(
              b_1 \R^n +
              b_2 \R^{n+c_2} +
              b_3 \R^{n+1}
              \right)
              \label{eq:DISum}
            \end{equation}
            \begin{equation}
              \label{eq:TRSum}
              \uu^{n+c_2}  =
              a^n_0\uu^{n - 1} +
              a^n_1\uu^{n} +
              a^n_2\uu^{n + 1}
              +
              \inc t^n
              \left(
              d^n_1\R^{n} +
              d^n_2\R^{n + 1}
              \right)
            \end{equation}
          \end{block}
    \item The quadrature coefficients are the same:
          \begin{table}[htbp]
            \centering
            \begin{tabular}{|c|c|c|}
              \hline
              $b_1$                            & $b_2$ & $b_3$ \\
              \hline
              $\frac{1}{2} - \frac{1}{6{c_2}}$ &
              $\frac{1}{6{c_2}(1-{c_2})}$      &
              $\frac{1}{2} - \frac{1}{6(1-{c_2})} $            \\
              \hline
            \end{tabular}
            % \caption{Butcher tableau of DITR U2R2}
            \label{tab:integ0Tab}
          \end{table}
  \end{itemize}
\end{frame}

\begin{frame}{\secname: \subsecname}
  \begin{itemize}
    \item The TR coefficients are different for each variant:
          \begin{table}[htbp]
            \centering
            \footnotesize
            \begin{tabular}{|c|c|c|c|}
              \hline
              Variant & U2R2                          & U2R1                 & U3R1                                                                                                                                   \\
              \hline
              $a^n_0$ & 0                             & 0                    & $-\frac{c_{2}\,{\left(c_{2}-1\right)}^2}{\Theta\,{\left(\Theta+1\right)}^2}$                                                           \\
              \hline
              $a^n_1$ & $1-(3{c_2}^2 - 2 {c_2}^3)$    & $1-(2c_2 - {c_2}^2)$ & $\frac{\left(\Theta+c_{2}\right)\,{\left(c_{2}-1\right)}^2}{\Theta}$                                                                   \\
              \hline
              $a^n_2$ & $3{c_2}^2 - 2 {c_2}^3$        & $2c_2 - {c_2}^2$     & $\frac{c_{2}\,\left(-\Theta^2\,c_{2}+2\,\Theta^2-\Theta\,{c_{2}}^2+3\,\Theta-2\,{c_{2}}^2+3\,c_{2}\right)}{{\left(\Theta+1\right)}^2}$ \\
              \hline
              $d^n_1$ & ${c_2} - 2 {c_2}^2 + {c_2}^3$ & 0                    & 0                                                                                                                                      \\
              \hline
              $d^n_2$ & $- {c_2}^2 + {c_2}^3$         & ${c_2}^2 - {c_2}$    & $\frac{c_{2}\,\left(\Theta+c_{2}\right)\,\left(c_{2}-1\right)}{\Theta+1}$                                                              \\
              \hline
            \end{tabular}
            % \caption{Interpolation coefficients for different DITR methods}
            \label{tab:inter0Tab}
          \end{table}
    \item $\Theta=\frac{\inc t^{n-1}}{\inc t^n}$ is the
          time step ratio for U3R1
  \end{itemize}
\end{frame}

\section{Iteration Methods}

\subsection{Nested Iteration}


\begin{frame}{\secname: \subsecname}
  \begin{itemize}
    \item Substituting TR into DI, the intermediate stage $n+c_2$ can be eliminated!
          \begin{equation}
            \begin{aligned}
              \uu^{n+1} & =                       \\
                        & \uu^{n} + \inc t\biggl[
              b_1 \R^n                            \\
                        & +
                b_2 \R(t^{n+c_2},a^n_0\uu^{n - 1} +
                a^n_1\uu^{n} +
                a^n_2\uu^{n + 1}
                +
                \inc t^n
                \left(
                d^n_1\R^{n} +
                d^n_2\R^{n + 1}
              \right))                            \\
                        & +
                b_3 \R(t^{n+1},\uu^{n+1})
                \biggr]
              \label{eq:nestedDITR}
            \end{aligned}
          \end{equation}
    \item Only one unknown; $\R$ is nested in $\R$
  \end{itemize}
\end{frame}

\begin{frame}{\secname: \subsecname}
  \begin{itemize}
    \item The implicit equation is formed as $\FF^N=0$, where residual $\FF^N$ is
          \small
          \begin{equation}
            \begin{aligned}
              \FF^N(\uu^{n+1}) & = -\frac{\uu^{n+1} - \uu^{n}}{\inc t^n} \\
                               & +
              b_1 \R^n                                                   \\
                               & +
              b_2 \R\left(t^{n+c_2},a^n_0\uu^{n - 1} +
              a^n_1\uu^{n} +
              a^n_2\uu^{n + 1}
              +
              \inc t^n
              \left(
              d^n_1\R^{n} +
              d^n_2\R^{n + 1}
              \right)\right)                                             \\
                               & +
              b_3 \R(t^{n+1},\uu^{n+1}) = 0.
              \label{eq:nestedDITR_F}
            \end{aligned}
          \end{equation}
          \normalsize
    \item We use a dual time stepping technique \cite{jameson1991time} along
          with linearized backward Euler in pseudo time marching:
          \small
          \begin{equation}
            \left(\frac{\eye}{\inc\tau} -
            \pdv{\FF^N}{\uu^{n+1}}  \right)
            \inc\uu^{n+1,m} = \FF^N(\uu^{n+1,m})
            \label{eq:abstractNewtonNested}
          \end{equation}
  \end{itemize}
\end{frame}


\begin{frame}{\secname: \subsecname}
  \begin{itemize}
    \item The Nested Iteration seems good:
          \begin{itemize}
            \item Only one unknown stage ($n+1$ step)
            \item Solve one linear problem each updating
          \end{itemize}
    \item But there are problems:
          \begin{itemize}
            \item Jacobian term $\pdv{\FF^N}{\uu^{n+1}}$
                  is a 2nd degree polynomial of $\pdv{\R}{\uu}$ terms
            \item Matrix-free solvers cannot be applied directly
            \item {\bf Pseudo time evolution is unstable}
          \end{itemize}
    \item Fourier analysis on pseudo time (1-D linear advection):
          \begin{equation}
            \begin{aligned}
              \kappa^* =
              -i\CFLt b_2 d_2^n {\kappa'}^2
              + (b_2a_2^n + b_3) \kappa' - \frac{i}{\CFLt}
            \end{aligned}
            \label{eq:imagKsTauDITR_General}
          \end{equation}
          $\kappa^*$ is modified wave number in pseudo time, $\kappa'=\kappa'(\kappa)$
          is the modified wave number of spacial discretization.
  \end{itemize}
\end{frame}

\begin{frame}{\secname: \subsecname}
  \begin{itemize}
    \item As $b_2d_2^n < 0$, when $CFL_t$ is large enough,
          $\kappa*$ is expected to have positive imaginary part (unstable).
    \item Example: using first order upwind difference scheme:
          \begin{figure}[htbp]
            \centering
            \includegraphics[width=0.5\textwidth]{../pics/HM3_U2R2050_KappaS.pdf}
            \caption[]{DITR U2R2 $c_2=1/2$ imaginary part of $\kappa^*$ using
              nested solving}
            \label{fig:HM3_U2R2050_KappaS}
          \end{figure}
  \end{itemize}
\end{frame}

\begin{frame}{\secname: \subsecname}
  \begin{itemize}
    \item Nested Iteration indeed diverges for large $\CFLt$, in linear advection experiments
    \item Nested Iteration is indeed able to converge at low $\CFLt$
    \item We aim to solve at larger $\CFLt$, so Nested Iteration is unacceptable
  \end{itemize}
\end{frame}

\subsection{Stage-decoupled Iteration}

\begin{frame}{\secname: \subsecname}
  \begin{itemize}
    \item Similar to full-implicit RK, we consider both $\uu^{n+c_2}$ and
          $\uu^{n+1}$ as unknown values.
    \item The implicit system $\FF^*=0$ is enlarged, with residual function $\FF^*$ defined as
          \begin{equation}
            \begin{aligned}
              \FF^* & =\begin{bmatrix}
                \FF^{n+c_2} \\
                \FF^{n+1}
              \end{bmatrix} = \Pm\begin{bmatrix}
                \GG^{n+c_2} \\
                \GG^{n+1}
              \end{bmatrix} \\ & = \Pm\begin{bmatrix}
                \frac{a^n_0\uu^{n - 1} +
                  a^n_1\uu^{n} +
                  a^n_2\uu^{n + 1} - \uu^{n+c_2}}{\inc t^n}
                +
                d^n_1\R^{n} +
                d^n_2\R^{n + 1}
                \\
                \frac{\uu^{n} - \uu^{n+1}}{\inc t^n}  +
                b_1 \R^n +
                b_2 \R^{n+c_2} +
                b_3 \R^{n+1}
              \end{bmatrix}
              \label{eq:DITRFFDef}
            \end{aligned}
          \end{equation}
    \item $\Pm$ is a preconditioning matrix ($2\time 2$ scalar parameters)
  \end{itemize}
\end{frame}


\begin{frame}{\secname: \subsecname}
  \begin{itemize}
    \item The current paper only uses $\Pm$ in a special form:
          \begin{equation}
            \Pm = \begin{bmatrix}
              \eye & \beta\eye \\
              0    & \eye
            \end{bmatrix}
            \label{eq:specialP}
          \end{equation}
    \item Then the residual functions for each stage is
          \begin{equation}
            \begin{aligned}
              \FF^{n+c_2} & =
              \frac{a^n_0\uu^{n - 1} +
                (a^n_1+\beta)\uu^{n} +
                (a^n_2-\beta)\uu^{n + 1} - \uu^{n+c_2}}{\inc t^n}
              \\ & +
              (d^n_1 + \beta b_1)\R^{n} +
              \beta b_2 \R^{n+c_2} +
              (d^n_2 + \beta b_3)\R^{n + 1},                             \\
              \FF^{n+1}   & =    \frac{\uu^{n} - \uu^{n+1}}{\inc t^n}  +
              b_1 \R^n +
              b_2 \R^{n+c_2} +
              b_3 \R^{n+1}.
            \end{aligned}
          \end{equation}
  \end{itemize}
\end{frame}

\begin{frame}{\secname: \subsecname}
  \begin{itemize}
    \item Instead of updating $\uu^{n+c_2}, \uu^{n+1}$ simultaneously,
          which requires the solution of an {\bf enlarged linear system}
    \item we decouple the stage values and update them separately with dual time stepping
          \begin{equation}
            \begin{aligned}
              \left(\frac{\eye}{\inc\tau} -
              \partialderivative{\FF^{n+c_2}(\uu^{n+c_2}, \uu^{n+1})}{\uu^{n+c_2}}  \right)
              \inc\uu^{n+c_2,m} & = \FF^{n+c_2}(\uu^{n+c_2,m}, \uu^{n+1,m}) \\
              \left(\frac{\eye}{\inc\tau} -
              \partialderivative{\FF^{n+1}(\uu^{n+c_2}, \uu^{n+1})}{\uu^{n+1}}  \right)
              \inc\uu^{n+1,m}   & = \FF^{n+1}(\uu^{n+c_2,m+1}, \uu^{n+1,m}) \\
            \end{aligned}
            \label{eq:DITRSDS}
          \end{equation}
  \end{itemize}
\end{frame}


\begin{frame}{\secname: \subsecname}
  \begin{itemize}
    \item Stage-decoupled iteration uses almost the same programs used in
          ESDIRK and BDF solving (easy implementation)
    \item Compatible with old matrix-free solvers (LU-SGS...)
    \item Convergence is affected by preconditioning parameter $\beta$
  \end{itemize}
\end{frame}

\begin{frame}{\secname: \subsecname}
  \small
  \begin{itemize}
    \item To choose parameter $\beta$, results of two analyses are combined:
          \begin{itemize}
            \item Fourier analysis in pseudo time ($\inc\tau \rightarrow 0$)
          \end{itemize}
          \begin{itemize}
            \item Gauss-Seidel stability ($\inc\tau \rightarrow \infty$, induced
                  when decoupling the stages)
          \end{itemize}
    \item Fourier analysis yields a linear ODE system of wave amplitudes:
          \footnotesize
          \begin{equation}
            \derivative{ }{\tau}
            \begin{bmatrix}
              A^{n+c_2}(\tau) \\
              A^{n+1}(\tau)
            \end{bmatrix}
            =
            -\imagUnit \frac{a}{\inc x}
            \left(\frac{\imagUnit}{\CFLt}\begin{bmatrix}
              A_{11} & A_{12} \\
              A_{21} & A_{22} \\
            \end{bmatrix}
            +\kappa'
            \begin{bmatrix}
              B_{11} & B_{12} \\
              B_{21} & B_{22} \\
            \end{bmatrix}
            \right)
            \begin{bmatrix}
              A^{n+c_2}(\tau) \\
              A^{n+1}(\tau)
            \end{bmatrix}
            +
            \begin{bmatrix}
              C_1 \\C_2
            \end{bmatrix}
          \end{equation}
          \normalsize
    \item Then stability relies on the modified pseudo time wave number, or equivalently
          the eigenvalues of Jacobian
          $\mu^*_i = \frac{\CFLt\kappa^*_i}{\imagUnit},\ i=1,2$ to have negative
          real parts (exponentially decaying).
  \end{itemize}
\end{frame}

\begin{frame}{\secname: \subsecname}
  \begin{itemize}
    \item There are two eigenvalues in Fourier analysis,
          we plot the maximum of their real part $\max(\real(\mu^*_1),\real(\mu^*_2))$
          as a function of $\mu'$, eigenvalue of the spacial discretization.
  \end{itemize}
  \begin{figure}[htbp]
    \centering
    \begin{subfigure}{0.33\textwidth}
      \includegraphics[width=\textwidth]{../pics/HM3_Method1_C0.5_Theta1_Beta0_TauMu.pdf}
      \caption[]{$\beta = 0$}
    \end{subfigure}\hfill
    \begin{subfigure}{0.33\textwidth}
      \includegraphics[width=\textwidth]{../pics/HM3_Method1_C0.5_Theta1_Beta0.5_TauMu.pdf}
      \caption[]{$\beta = 0.5$}
    \end{subfigure}\hfill
    \begin{subfigure}{0.33\textwidth}
      \includegraphics[width=\textwidth]{../pics/HM3_Method1_C0.5_Theta1_Beta1_TauMu.pdf}
      \caption[]{$\beta = 1$}
    \end{subfigure}
    \caption{DITR U2R2 $c_2=1/2$, $\max(\real(\mu^*_1),\real(\mu^*_2))$ distribution, black line is 0}
    \label{fig:MuU2R2}
  \end{figure}
\end{frame}

\begin{frame}{\secname: \subsecname}
  \begin{itemize}
    \small
    \item Gauss-Seidel stability analysis takes decoupling into account
    \item Solving linear ODE $y'=\frac{z}{\inc t}y$, when $\inc \tau\rightarrow\infty$, the updating becomes
          Gauss-Seidel iteration of a $2\times 2$ linear system.
    \item Magnifying factor between steps $G_{GS}$
  \end{itemize}
  \begin{figure}[htbp]
    \centering
    \begin{subfigure}{0.33\textwidth}
      \includegraphics[width=\textwidth]{../pics/HM3_Method1_C0.5_Theta1_Beta0_GGS.pdf}
      \caption[]{$\beta = 0$}
    \end{subfigure}\hfill
    \begin{subfigure}{0.33\textwidth}
      \includegraphics[width=\textwidth]{../pics/HM3_Method1_C0.5_Theta1_Beta0.5_GGS.pdf}
      \caption[]{$\beta = 0.5$}
    \end{subfigure}\hfill
    \begin{subfigure}{0.33\textwidth}
      \includegraphics[width=\textwidth]{../pics/HM3_Method1_C0.5_Theta1_Beta1_GGS.pdf}
      \caption[]{$\beta = 1$}
    \end{subfigure}
    \caption{DITR U2R2 $c_2=1/2$, $|G_{GS}|$ distribution, black line is 1}
    \label{fig:GGSU2R2}
  \end{figure}
\end{frame}

\begin{frame}{\secname: \subsecname}
  \footnotesize
  \begin{itemize}
    \item Stability requires, $\mu^*_{M}(\beta) \leq 0,G_{GS,M}(\beta) \leq 1$.
  \end{itemize}
  \begin{figure}[htbp]
    \centering
    \begin{subfigure}{0.45\textwidth}
      \includegraphics[width=\textwidth]{../pics/HM3_Methods_SearchMu.pdf}
      % \caption[]{$\beta = 0$}
    \end{subfigure}
    \begin{subfigure}{0.45\textwidth}
      \includegraphics[width=\textwidth]{../pics/HM3_Methods_SearchGGS.pdf}
      % \caption[]{$\beta = 0.5$}
    \end{subfigure}
    \caption{Numerical scanning results of $\beta$}
    \label{fig:MuGGSSearch}
  \end{figure}
\end{frame}

\begin{frame}{\secname: \subsecname}
  \small
  \begin{itemize}
    \item Stability restrictions $\mu^*_{M}(\beta) \leq 0,G_{GS,M}(\beta) \leq 1$ are listed
  \end{itemize}
  \footnotesize
  \begin{table}[htbp]
    \centering
    \begin{tabular}{|c|c|c|}
      \hline
      Method             & Restriction from $\mu^*_{M}$ & Restriction from $G_{GS,M}$ \\
      \hline
      U2R2 $c_2=0.5$     & $\beta > 0.63$               & $\beta > 0.38$              \\
      \hline
      U2R2 $c_2=0.55$    & $\beta > 0.72$               & $\beta > 0.53$              \\
      \hline
      U2R1 $c_2=0.25$    & $\beta > 0.63$               & $\beta > 0.34$              \\
      \hline
      U2R1 $c_2=0.5$     & $\beta > 1$                  & $\beta > 0.75$              \\
      \hline
      U3R1 $\Theta=0.25$ & $\beta > 0.71$               & $\beta > 0.45$              \\
      \hline
      U3R1 $\Theta=1$    & $\beta > 0.83$               & $\beta > 0.57$              \\
      \hline
      U3R1 $\Theta=4$    & $\beta > 0.94$               & $\beta > 0.68$              \\
      \hline
    \end{tabular}
    \caption{Approximate stability restrictions on $\beta$ obtained in numerical scanning}
    \label{tab:resrictionBetaSearch}
  \end{table}
\end{frame}



\begin{frame}{\secname: \subsecname}
  \small
  \begin{itemize}
    \item From Figure \ref{fig:MuGGSSearch}, it seems minimizing
          the searched maximums could help improve convergence.
    \item Empirical tests show that $\beta$ being slightly larger
          could reduce the region of instability, thus enhancing
          convergence in nonlinear systems.
    \item Our recommended empirical choice:
          \begin{itemize}
            \item For U2R2 $c=0.5$, $\beta = 1$.
            \item For U2R2 $c=0.55$, $\beta = 1.333$.
            \item For U2R1 $c=0.25$, $\beta = 1$.
            \item For U3R1 with any $\Theta$, $\beta = 1.333$
          \end{itemize}
    \item The choices meet the stability restrictions,
          and have reasonable converge in various test problems.
  \end{itemize}
\end{frame}

\section{Numerical Tests}

\begin{frame}{\secname}
  \begin{itemize}
    \item We have tested 4 DITR variants:
          \begin{itemize}
            \item U2R2 $c_2=0.5$, 4th order, $A$-stable.
            \item U2R2 $c_2=0.55$, 3rd order, $A$-stable.
            \item U2R1 $c_2=0.25$, 3rd order, $L$-stable.
            \item U3R1 $c_2=0.5$, 4th order, $L$-stable, multistep.
          \end{itemize}
    \item Transient numerical cases with compressible N-S equations:
          \begin{itemize}
            \item Isentropic vortex
            \item Vortex shedding of a circular cylinder
            \item Double Mach reflection
          \end{itemize}
    \item 4th order variational reconstruction CFV discretization is used.
    \item Details of the test cases are referred to the proceedings article.
  \end{itemize}
\end{frame}


\begin{frame}{\secname: Isentropic Vortex}
  % \begin{itemize}
  %   \item The 
  % \end{itemize}
  \scriptsize
  \begin{figure}[htbp]
    \centering
    \begin{subfigure}{0.5\textwidth}
        \includegraphics[width=\textwidth]{pics/HM3_IV160_fig_1.pdf}
        \caption[]{Density error vs. time step size }
        \label{sfig:IVTests_Conv}
    \end{subfigure}\hfill
    \begin{subfigure}{0.5\textwidth}
        \includegraphics[width=\textwidth]{pics/HM3_IV160_fig_2.pdf}
        \caption[]{Density error vs. total CPU time}
        \label{sfig:IVTests_Eff}
    \end{subfigure}
    \caption[]{Convergence and efficiency test with isentropic vortex problem}
    \label{fig:IVTests}
  \end{figure}
\end{frame}


\begin{frame}{\secname: Vortex Shedding}
  % \begin{itemize}
  %   \item The 
  % \end{itemize}
  \scriptsize
  \begin{figure}[htbp]
    \centering
    \begin{subfigure}{0.4\textwidth}
        \includegraphics[width=\textwidth]{pics/CylinderB1_Re1200_Mesh.png}
        \caption[]{Mesh}
        \label{sfig:CylinderRe1200Demo_Mesh}
    \end{subfigure}
    \begin{subfigure}{0.4\textwidth}
        \includegraphics[width=\textwidth]{pics/CylinderB1_Re1200.png}
        \caption[]{Vorticity distribution}
        \label{sfig:CylinderRe1200Demo_Vort}
    \end{subfigure}
    \caption[]{Mesh and a instance of z-vorticity distribution
        in $Re=1200$ vortex shedding problem}
    \label{fig:CylinderRe1200Demo}
  \end{figure}
\end{frame}



\begin{frame}{\secname: Vortex Shedding}
  % \begin{itemize}
  %   \item The 
  % \end{itemize}
  \scriptsize
  \begin{figure}[htbp]
    \centering
    \begin{subfigure}{0.5\textwidth}
        \includegraphics[width=\textwidth]{pics/Cylinder_fig14.pdf}
        \caption[]{Error vs. time step size}
        \label{sfig:CylinderRe1200_My_C}
    \end{subfigure}\hfill
    \begin{subfigure}{0.5\textwidth}
        \includegraphics[width=\textwidth]{pics/Cylinder_fig4.pdf}
        \caption[]{Error vs. CPU Time}
        \label{sfig:CylinderRe1200_My_E}
    \end{subfigure}
    \caption[]{Convergence and efficiency analysis with $\epsilon_{My}$ in $Re=1200$ vortex shedding problem}
    \label{fig:CylinderRe1200_My}
  \end{figure}
\end{frame}

\begin{frame}{\secname: DMR}
  \scriptsize
  \begin{figure}[htbp]
    \centering
    \begin{subfigure}{0.5\textwidth}
        \includegraphics[width=\textwidth]{pics/DM480_HM3LBT.png}
        \caption[]{DITR U2R2 $c_2=0.5$}
        \label{sfig:DM480_HM3LBT}
    \end{subfigure}\hfill
    \begin{subfigure}{0.5\textwidth}
        \includegraphics[width=\textwidth]{pics/DM480_HM3.png}
        \caption[]{DITR U2R2 $c_2=0.55$}
        \label{sfig:DM480_HM3}
    \end{subfigure}
    \caption{Density in double Mach reflection problem, DITR U2R2}
    \label{fig:DM480-1}
  \end{figure}
\end{frame}

\begin{frame}{\secname: DMR}
  \scriptsize
  \begin{figure}[htbp]
    \centering
    \begin{subfigure}{0.5\textwidth}
        \includegraphics[width=\textwidth]{pics/DM480-1_HM3U2R1.png}
        \caption[]{DITR U2R1}
        \label{sfig:DM480_HM3U2R1}
    \end{subfigure}\hfill
    \begin{subfigure}{0.5\textwidth}
        \includegraphics[width=\textwidth]{pics/DM480-1_HM3U3R1.png}
        \caption[]{DITR U3R1}
        \label{sfig:DM480_HM3U3R1}
    \end{subfigure}
    \caption{Density in double Mach reflection problem, DITR U2R1 and U3R1}
    \label{fig:DM480-2}
  \end{figure}
\end{frame}

\begin{frame}{\secname: DMR}
  \scriptsize
  \begin{figure}[htbp]
    \centering
    \begin{subfigure}{0.5\textwidth}
        \includegraphics[width=\textwidth]{pics/DM480-1_ESDIRK4-T2.png}
        \caption[]{ESDIRK4}
        \label{sfig:DM480_ESDIRK4}
    \end{subfigure}\hfill
    \begin{subfigure}{0.5\textwidth}
        \includegraphics[width=\textwidth]{pics/DM480-1_BDF-T2.png}
        \caption[]{BDF2}
        \label{sfig:DM480_BDF2}
    \end{subfigure}
    \caption{Density in double Mach reflection problem, baseline methods}
    \label{fig:DM480-3}
  \end{figure}
\end{frame}

% \subsection{进度介绍}

% \begin{frame}{进度介绍}
%   \begin{figure}
%     \includegraphics[width=0.4\linewidth]{thulogo}
%     \caption{进度介绍}
%   \end{figure}
% \end{frame}

% \section{研究工作及成果}

% \begin{frame}{已完成的研究工作及成果}
%   \begin{block}{已完成的研究工作简介}
%     \begin{itemize}
%       \setlength{\itemsep}{6pt}
%       \item XXXX
%       \item XXXX
%       \item XXXX
%       \item XXXX
%       \item XXXX
%       \item XXXX
%     \end{itemize}
%   \end{block}
% \end{frame}

% \subsection{研究工作一}

% \begin{frame}{研究工作一}
%   \begin{block}{无编号公式}
%     无编号公式示例:
%     $$
%       k:[-\pi,\pi] \rightarrow [0,1]
%     $$
%   \end{block}
% \end{frame}

% \subsection{研究工作二}

% \begin{frame}{研究工作二}
%   \begin{block}{有编号公式}
%     \begin{itemize}
%       \item 有编号公式示例:输入为图像
%             \begin{equation}
%               % \vx\in\reals^{C_{\text{in}}\times H\times W}
%               a
%             \end{equation}
%             其中 $C_{\text{in}}$ 表示通道, $H$ 表示图像高度, $W$ 表示图像深度.
%     \end{itemize}
%   \end{block}
% \end{frame}

% \subsection{研究工作三}

% \begin{frame}{研究工作三}
%   \begin{block}{表格}
%     表格示例, 如表 \ref{tab:unique_values} 所示.

%     \begin{table}[htbp]
%       \small
%       \centering
%       \caption{train.csv 每列非重复元素个数}
%       \label{tab:unique_values}
%       \begin{tabular}{lc}
%         \toprule
%         column       & \# unique values \\
%         \midrule
%         posting\_id  & 34250            \\
%         image        & 32412            \\
%         image\_phash & 28735            \\
%         title        & 33117            \\
%         label\_group & 11014            \\
%         \bottomrule
%       \end{tabular}
%     \end{table}
%   \end{block}
% \end{frame}

% \subsection{研究工作四}

% \begin{frame}{研究工作四}
%   \begin{block}{并排图片}
%     并排图片示例.
%   \end{block}
%   \begin{figure}[htbp]
%     \centering
%     \begin{minipage}[t]{0.48\textwidth}
%       \centering
%       \includegraphics[width=3cm]{thulogo.pdf}
%       \caption{并排图片1}
%       \label{fig:left_side}
%     \end{minipage}
%     \begin{minipage}[t]{0.48\textwidth}
%       \centering
%       \includegraphics[width=3cm]{thulogo.pdf}
%       \caption{并排图片2}
%       \label{fig:right_side}
%     \end{minipage}
%   \end{figure}
% \end{frame}

% \section{后期工作与安排}

% \subsection{后期研究工作}

% \begin{frame}{后期研究工作}
%   \begin{block}{后期研究工作}
%     \begin{itemize}
%       \setlength{\itemsep}{6pt}
%       \item XXXX
%       \item XXXX
%       \item XXXX
%       \item XXXX
%     \end{itemize}
%   \end{block}
% \end{frame}

% \subsection{进度安排}

% \begin{frame}{进度安排}
%   \begin{block}{进度安排}
%     \begin{itemize}
%       \setlength{\itemsep}{6pt}
%       \item XXXX\cite{liu2016}
%       \item XXXX\cite{ren2010}
%       \item XXXX\cite{Chen1992}
%       \item XXXX\cite{Gravagne2003}
%     \end{itemize}
%   \end{block}
% \end{frame}

% \section{问题与解决方案}

% \begin{frame}{问题与解决方案}
%   \begin{block}{问题}
%     \begin{itemize}
%       \item XXXX
%     \end{itemize}
%   \end{block}
%   \begin{block}{对应解决方案}
%     \begin{itemize}
%       \item XXXX
%     \end{itemize}
%   \end{block}
% \end{frame}

% \section{按时完成可能性}

% \begin{frame}{按时完成可能性}
%   \begin{block}{按时完成可能性}
%     \begin{itemize}
%       \setlength{\itemsep}{6pt}
%       \item XXXX\cite{xin1994}
%       \item XXXX\cite{zhai2015}
%       \item XXXX\cite{jones2004}
%       \item XXXX\cite{mcmahan2005}
%     \end{itemize}
%   \end{block}
% \end{frame}


\renewcommand\refname{References} 
\begin{frame}[allowframebreaks]{References}
  \bibliographystyle{unsrt}
  \bibliography{../HM3DraftRefs}
\end{frame}

\begin{frame}
  \begin{center}
    {\Huge\calligra Thanks for your attention!}
    \vspace{1cm}

    {\Huge Q \& A}
  \end{center}
\end{frame}

% 结束文档撰写
\end{document}
